%
% File nodalida2015.tex
%
% Contact beata.megyesi@lingfil.uu.se
%
% Based on the instruction file for EACL 2014
% which in turn was based on the instruction files for previous 
% ACL and EACL conferences.


\documentclass[11pt]{article}
\usepackage[T2A,T1]{fontenc}
\usepackage[utf8]{inputenc}
\usepackage[russian,english]{babel}

\usepackage{times}
\usepackage{latexsym}
\usepackage{fixltx2e} %allows subscripts
%\usepackage{mathptmx}
%\usepackage{txfonts}
\usepackage{url}
\special{papersize=210mm,297mm} % to avoid having to use "-t a4" with dvips 
%\setlength\titlebox{6.5cm}  % You can expand the title box if you really have to

\usepackage{nodalida2015}

\usepackage{linguex}

\newcommand{\rus}[1]{\foreignlanguage{russian}{#1}}
\newcommand{\ft}[1]{\marginpar{\scriptsize F: #1}}
\newcommand{\rr}[1]{\marginpar{\scriptsize R: #1}}

\title{Automatic word stress annotation of Russian running text (max 8 pages)}

\author{Author1 \\
  Affiliation / Address line 1 \\
  Affiliation / Address line 2 \\
  Affiliation / Address line 3 \\
  {\tt email@domain.com} \\\And
  Author2 \\
  Affiliation / Address line 1 \\
  Affiliation / Address line 2 \\
  Affiliation / Address line 3 \\
  {\tt email@domain.com} \\}

\date{2015}

\begin{document}
\maketitle
\begin{abstract}
  We evaluate the effectiveness of new tools that we developed for automatically annotating 
  word stress in Russian running text. Given an adequate lexicon with specified stress, the
  primary obstacle for correct stress placement is disambiguating homographic forms. Using a
  Constraint Grammar, we achieve a X\% improvement over the baseline undisambiguated performance.
\end{abstract}

%\section{Credits}
%
%The template had a credits section for the history of the template, but we don't need
%a Credits section, right?

\section{Introduction}

Russian word stress is variable. The inflecting word classes include complex patterns of
shifting stress, and a lexeme's stress pattern cannot be predicted from surface forms.
Although lexical stress and its attendant vowel reduction are a prominent feature of
spoken Russian, standard written Russian does not typically mark word stress.\footnote{
Texts intended for native speakers sometimes mark stress on words that cannot be 
disambiguated through context, although this is somewhat rare.} Without information about lexical 
stress position, 
correctly converting written Russian text to speech is impossible. This is true both for humans
(e.g. foreign language students) and computers (e.g. text-to-speech).

We identify three different types of word stress ambiguity. \emph{Intraparadigmatic} ambiguity 
refers
to homographic wordforms belonging to the same lexeme, as shown in \ref{ex:intrahom}. 

\ex. Intraparadigmatic homographs \label{ex:intrahom}
\a. \rus{т\'{е}ла} \emph{t\'{e}la} `body.\textsubscript{SG-GEN}' \label{ex:bodySGGEN}
\b. \rus{тел\'{а}} \emph{tel\'{a}} `body.\textsubscript{PL-NOM}' \label{ex:bodyPLNOM}

The remaining two types of stress ambiguity occur between lexemes. \emph{Morphosyntactically
congruent homographs} are homographs that belong to separate lexemes, and whose morphosyntactic 
values are identical, as shown in Table~\ref{table:MScongruent}. 

\begin{table}
\begin{center}
\begin{tabular}{c|ll}
\bf Number-Case & \bf `castle' & \bf `lock' \\
\hline
\texttt{SG-NOM} & \rus{з\'{a}мок} & \rus{зам\'{о}к} \\
\texttt{SG-GEN} & \rus{з\'{a}мка} & \rus{замк\'{а}} \\
\texttt{PL-NOM} & \rus{з\'{a}мки} & \rus{замк\'{и}} \\
\texttt{PL-GEN} & \rus{з\'{a}мков} & \rus{замк\'{о}в} \\
... & \hskip 1em ... & \hskip 1em ... \\
\end{tabular}
\end{center}
\caption{ \label{table:MScongruent} Morphosyntactically congruent homographs. }
\end{table}

\emph{Morphosyntactically incongruent homographs} are homographs that belong to separate
lexemes, and whose morphosyntactic values are different, as shown in 
\ref{ex:MSincongruent}.

\ex. Morphosyntactically incongruent homographs \label{ex:MSincongruent}
\a. \rus{попад\'{а}ть} \emph{popadát'} `fall.\textsubscript{INF-IPFV}' \\
    \rus{поп\'{а}дать} \emph{popádat'} `fall.\textsubscript{INF-PFV}'
\b. \rus{узнаёт} \emph{uznaët} `finds out.\textsubscript{PRS-IPFV-3SG}' \\
    \rus{узн\'{а}ет} \emph{uznáet} `finds out.\textsubscript{FUT-PFV-3SG}'
\c. \rus{н\'{а}шей} \emph{nášej} `our.\textsubscript{F-SG-GEN/DAT/LOC...}'\\
    \rus{наш\'{е}й} \emph{našéj} `sew on.\textsubscript{IMP-2SG}'
\d. \rus{дор\'{о}га} \emph{doróga} `road.\textsubscript{N-F-SG-NOM}'\\
    \rus{дорог\'{а}} \emph{dorogá} `dear.\textsubscript{ADJ-F-SG-PRED}'

\rr{another relevant issue is whether the ambiguity can be resolved at all. Maybe this is 
something we should bring up in the discussion section. For example, when is stress marked in 
authentic by-native-for-native texts? A computer should not be expected to predict the stress on 
such words (although a computer could possibly identify words that should be marked for 
natives.)} It should be noted that not all morphosyntactic ambiguity leads to stress ambiguity. For example,
all inflecting modifiers in Russian have the same form for F-SG-GEN, F-SG-LOC, F-SG-DAT and 
F-SG-INS. We refer to this as \emph{stress-irrelevant} ambiguity, since all readings have
the same stress placement. Stress-irrelevant ambiguity can be intraparadigmatic and/or interlexical. 

Different types of stress ambiguity are overcome by different information. Detailed 
part-of-speech tagging can determine the stress of intraparadigmatic homographs and
morphosyntactically incongruent homographs, whereas word sense disambiguation can 
determine the stress position of morphosyntactically congruent homographs. In the case of
running text in Russian, most stress placement ambiguity is rooted in morphosyntactic
ambiguity, so in this study we focus on the role of part-of-speech tagging in improving
automatic stress placement, leaving word sense disambiguation to future work.

\subsection{Background and Task Definition}

Automatic stress placement in Russian is similar to Diacritic Restoration, a task
which has received increasing interest over the last 20 years. Generally speaking, diacritics
disambiguate otherwise homographic wordforms, so missing diacritics can
complicate many NLP tasks, such as text-to-speech and speech-to-text. For example, speakers 
of Czech may type emails and other communications
without standard diacritics. In order to generate speech from these texts, they must first be 
normalized by restoring diacritics. A slightly different
situation arises with Arabic and Hebrew, since standard orthography lacks vowel diacritics
altogether. For such languages, the `restoration' of vowel diacritics results in less
ambiguity than in standard orthography. For languages with inherently ambiguous orthography,
it may be more precise to refer to this as `diacritic enhancement', since it produces
text that is less ambiguous than the standard language. In this sense, Russian orthography
is similar to Arabic and Hebrew, since its vowel qualities are underspecified in
standard orthography.

Many studies of Russian text-to-speech and automatic speech recognition make note of the 
difficulties caused by the shortcomings of their stress-marking resources (e.g. CITE Krivnova 
1998), but very few studies have targeted stress-marking itself, and they tend to
focus on place stress on unknown words, with almost no attention to the problems that
arise with known words. For example, 
\cite{Xomicevic_2008} developed a set of heuristics for guessing stress placement on 
unknown words in Russian. More recently, CITE Hall \& Sproat (2013) trained a maximum entropy
model on a dictionary of Russian words, achieving 83.9\% accuracy for stress on unknown words 
and 98.7\% on known words. One major shortcoming of evaluating on wordlists is 
that they do not reflect wordform frequency.
Many of the most problematic stress ambiguities in Russian occur in 
high-frequency wordforms\rr{might it be worth computing the difference in stress
ambiguity between wordform lists and running text? i.e. what percentage of tokens have ambiguous
stress.}, so evaluations based on wordlists - as opposed to running
text - have inflated results. Furthermore, working with running text allows for the
possibility of disambiguating homographs based on syntactic context.

So far, the implicit target application the few studies related to automatic stress placement
in Russian has been text-to-speech and speech recognition technology. However, the target 
application of our stress
annotator is a different domain: language learning. Since standard Russian does not contain 
stress-markings, learners are frequently unable to pronounce unknown words correctly without
referencing a dictionary or similar resources. In the context
of language learning, marking stress incorrectly is arguably worse than not marking it
at all. Because of this, we want our stress annotator to be able to abstain from marking
stress on words that it is unable to resolve with high confidence.

\section{Automatic Stress Placement}

State-of-the-art part-of-speech and morphological analysis in Russian is based on finite-state
technology.\footnote{add footnote or citations for Dialog, mystem, etc.} We developed free
and open-source finite-state tools capable of analyzing and
generating stressed wordforms. A finite-state transducer\footnote{a two-level 
morphology, xfst/hfst, blah, blah, blah} generates all possible readings
of each wordform, and a constraint grammar\footnote{vislcg3, blah blah} then removes
some extra readings based on syntactic context. The ultimate success of our stress placement
system depends on the performance of the Constraint Grammar. Ideally, the Constraint Grammar would
successfully remove all but the correct reading for each token, but in practice some
tokens still have more than one reading remaining. Therefore,
we also evaluate various approaches to deal with the remaining ambiguity,
as described in the following paragraph and Table~\ref{tab:conditions}.

\begin{table*}[t]
  \centering
  \begin{tabular}{r|c|c}
    & \rus{кость}-N-F-SG-GEN\hskip 1em\rus{к\'{о}сти} & \rus{кость}-N-F-SG-GEN\hskip 1em\rus{к\'{о}сти} \\
    & \rus{кость}-N-F-SG-DAT\hskip 1em\rus{к\'{о}сти} & \rus{кость}-N-F-SG-DAT\hskip 1em\rus{к\'{о}сти} \\
    & \rus{костить}-V-IPFV-IMP\hskip 1em\rus{кост\'{и}} & \\
    \hline
    {\small {\tt bare}} & \rus{кости} & \rus{кости} \\
    {\small {\tt safe}} & \rus{кости} & \rus{к\'{о}сти} \\
    {\small {\tt guess}} & \rus{к\'{о}сти} or \rus{кост\'{и}} & \rus{к\'{о}сти}\\
    {\small {\tt guessFreq}} & \rus{к\'{о}сти} & \rus{к\'{о}сти}
  \end{tabular}
  \caption{Example output of each condition, given a particular set of readings}
  \label{tab:conditions}
\end{table*}

The {\small {\tt bare}} approach is to not mark stress on words with more than one reading.
The {\small {\tt safe}} approach is to mark stress only on
tokens whose morphosyntactic ambiguity is stress-irrelevant. The {\small {\tt guess}}
approach is to randomly select one of the available readings.
The {\small {\tt guessFreq}} approach is to select the reading that is most frequent
in our corpus. If none of the readings is found in the corpus, then it selects the reading
with the tagset that is most frequent in our corpus. Note that for tokens with
stress-irrelevant ambiguity, {\small {\tt guess}} and {\small {\tt guessFreq}} produce
the same result as the {\small {\tt safe}} method.

For our baseline, we take the output of our morphological analyzer (without the Constraint 
Grammar) in combination with {\small {\tt bare}}, {\small {\tt safe}}, {\small {\tt guess}} and {\small {\tt guessFreq}}. The purpose of this study is to compare the 




\subsection{Electronically-available resources}

Nodalida-2015 provides this description in \LaTeX2e ({\small {\tt
    nodalida2015.tex}}) and PDF format ({\small {\tt nodalida2015.pdf}}),
along with the \LaTeX2e style file used to format it ({\small {\tt
    nodalida2015.sty}}) and an ACL bibliography style ({\small {\tt
    acl.bst}}) in case you want to use the Bib\TeX\ reference management
software. These files are all available at
\url{http://stp.lingfil.uu.se/~bea/nodalida15/}. A Microsoft Word template file ({\small
  {\tt nodalida2015.docx}}) is also available at the same URL. We strongly
recommend the use of these style files, which have been appropriately
tailored for the Nodalida-2015 proceedings. If you have an option, we
recommend that you use the \LaTeX2e version. \textbf{If you will be
  using the Microsoft Word template, we suggest that you anonymize
  your source file so that the pdf produced does not retain your
  identity.}  This can be done by removing any personal information
from your source document properties.



\subsection{Format of Electronic Manuscript}
\label{sect:pdf}

For the production of the electronic manuscript you must use Adobe's
Portable Document Format (PDF). 
%% Add instructions about using pdflatex
This format can be generated directly from \LaTeX2e files or from
postscript ones. On Linux/Unix-like systems, you can use {\tt
  pdflatex} to generate a PDF file from \LaTeX2e files, or {\tt
  ps2pdf} to convert from postscript to PDF. 
%%
In Microsoft Windows, you can use Adobe's Distiller, or if you have
\texttt{cygwin} installed, you can use \texttt{dvipdf} or
\texttt{ps2pdf}. Note that some word processing programs generate PDF
which may not include all the necessary fonts (esp. tree diagrams,
symbols). When you print or create the PDF file, there is usually an
option in your printer setup to include none, all or just non-standard
fonts.  Please make sure that you select the option of including ALL
the fonts. {\em Before sending it, test your PDF by printing it from a
  computer different from the one where it was created.} Moreover,
some word processors may generate very large postscript/PDF files,
where each page is rendered as an image. Such images may reproduce
poorly. In this case, try alternative ways to obtain the postscript
and/or PDF. One way on some systems is to install a driver for a
postscript printer, send your document to the printer specifying
``Output to a file'', then convert the file to PDF.
Please keep in mind that it is of utmost importance to use \textbf{A4 format}.

%%Removed by YP (use of the \special command above instead):
%% It is of utmost importance to specify the \textbf{A4 format} (21 cm x
%% 29.7 cm) when formatting the paper. 
%% When working with {\tt dvips}, for instance, one should specify {\tt -t a4}.

Print-outs of the PDF file on A4 paper should be identical to the
hardcopy version. If you cannot meet the above requirements about the
production of your electronic submission, please contact the
program chair above as soon as possible.


\subsection{Layout}
\label{ssec:layout}

Format manuscripts two columns to a page, in the manner these
instructions are formatted. The exact dimensions for a page on A4
paper are:

\begin{itemize}
\item Left and right margins: 2.5 cm
\item Top margin: 2.5 cm
\item Bottom margin: 2.5 cm
\item Column width: 7.7 cm
\item Column height: 24.7 cm
\item Gap between columns: 0.6 cm
\end{itemize}

\noindent Papers should not be submitted on any other paper size.
 If you cannot meet the above requirements about the production of your electronic submission, 
please contact the program chair above as soon as possible.

\subsection{Fonts}
\label{ssec:fonts}
For reasons of uniformity, Adobe's {\bf Times Roman} font should be
used. In \LaTeX2e{} this is accomplished by putting

\begin{quote}
\begin{verbatim}
\usepackage{times}
\usepackage{latexsym}
\end{verbatim}
\end{quote}
in the preamble. If Times Roman is unavailable, use {\bf Computer
  Modern Roman} (\LaTeX2e{}'s default).  Note that the latter is about
  10\% less dense than Adobe's Times Roman font.


\begin{table}[h]
\begin{center}
\begin{tabular}{|l|rl|}
\hline \bf Type of Text & \bf Font Size & \bf Style \\ \hline
paper title & 15 pt & bold \\
author names & 12 pt & bold \\
author affiliation & 12 pt & \\
the word ``Abstract'' & 12 pt & bold \\
section titles & 12 pt & bold \\
document text & 11 pt  &\\
captions & 11 pt & \\
abstract text & 10 pt & \\
bibliography & 10 pt & \\
footnotes & 9 pt & \\
\hline
\end{tabular}
\end{center}
\caption{\label{font-table} Font guide. }
\end{table}

\subsection{The First Page}
\label{ssec:first}

Center the title, author's name(s) and affiliation(s) across both
columns. Do not use footnotes for affiliations. Do not include the
paper ID number assigned during the submission process. Use the
two-column format only when you begin the abstract.

{\bf Title}: Place the title centered at the top of the first page, in
a 15-point bold font. (For a complete guide to font sizes and styles, 
see Table~\ref{font-table}.) Long titles should be typed on two lines without
a blank line intervening. Approximately, put the title at 2.5 cm from
the top of the page, followed by a blank line, then the author's
names(s), and the affiliation on the following line. Do not use only
initials for given names (middle initials are allowed). Do not format surnames
in all capitals (e.g., use ``Schlangen'' not ``SCHLANGEN'').
Do not format title and section headings in all capitals as well
except for proper names (such as ``BLEU'') that are conventionally
in all capitals.
The affiliation should contain the author's complete address, and if
possible, an electronic mail address. Leave about 2 cm between the
affiliation and the body of the first page.
The title, author names and addresses should be completely
identical to those entered to the electronical paper submission
website in order to maintain the consistency of author information
among all publications of the conference.

{\bf Abstract}: Type the abstract at the beginning of the first
column. The width of the abstract text should be smaller than the
width of the columns for the text in the body of the paper by about
0.6 cm on each side. Center the word {\bf Abstract} in a 12 point bold
font above the body of the abstract. The abstract should be a concise
summary of the general thesis and conclusions of the paper. It should
be no longer than 200 words. The abstract text should be in 10 point font.

{\bf Text}: Begin typing the main body of the text immediately after
the abstract, observing the two-column format as shown in 
the present document. Do not include page numbers.

{\bf Indent} when starting a new paragraph. Use 11 points for text and 
subsection headings, 12 points for section headings and 15 points for
the title. 

\subsection{Sections}

{\bf Headings}: Type and label section and subsection headings in the
style shown on the present document.  Use numbered sections (Arabic
numerals) in order to facilitate cross references. Number subsections
with the section number and the subsection number separated by a dot,
in Arabic numerals. Do not number subsubsections.

{\bf Citations}: Citations within the text appear
in parentheses as~\cite{Gusfield:97} or, if the author's name appears in
the text itself, as Gusfield~\shortcite{Gusfield:97}. 
Append lowercase letters to the year in cases of ambiguity.  
Treat double authors as in~\cite{Aho:72}, but write as
 in~\cite{Chandra:81} when more than two authors are involved. Collapse multiple citations as
in~\cite{Gusfield:97,Aho:72}. Also refrain from using full citations as sentence constituents. We
suggest that instead of
\begin{quote}
  ``\cite{Gusfield:97} showed that ...''
\end{quote}
you use
\begin{quote}
``Gusfield \shortcite{Gusfield:97}   showed that ...''
\end{quote}

If you are using the provided \LaTeX{} and Bib\TeX{} style files, you
can use the command \verb|\newcite| to get ``author (year)'' citations.

As reviewing will be double-blind, the submitted version of the papers should not include the
authors' names and affiliations. Furthermore, self-references that
reveal the author's identity, e.g.,
\begin{quote}
``We previously showed \cite{Gusfield:97} ...''  
\end{quote}
should be avoided. Instead, use citations such as 
\begin{quote}
``Gusfield \shortcite{Gusfield:97}
previously showed ... ''
\end{quote}

\textbf{Please do not  use anonymous citations} and  do not include acknowledgements 
when submitting your paper. Papers that do not conform
to these requirements may be rejected without review. 

\textbf{References}: Gather the full set of references together under
the heading {\bf References}; place the section before any Appendices,
unless they contain references. Arrange the references alphabetically
by first author, rather than by order of occurrence in the text.
Provide as complete a citation as possible, using a consistent format,
such as the one for {\em Computational Linguistics\/} or the one in the 
{\em Publication Manual of the American 
Psychological Association\/}~\cite{APA:83}.  Use of full names for
authors rather than initials is preferred.  A list of abbreviations
for common computer science journals can be found in the ACM 
{\em Computing Reviews\/}~\cite{ACM:83}.

The \LaTeX{} and Bib\TeX{} style files provided roughly fit the
American Psychological Association format, allowing regular citations, 
short citations and multiple citations as described above.

{\bf Appendices}: Appendices, if any, directly follow the text and the
references (but see above).  Letter them in sequence and provide an
informative title: {\bf Appendix A. Title of Appendix}.

\textbf{Acknowledgement} section should appear in accepted manuscripts
only. It should go as a last section immediately
before the references.  Do not number the acknowledgement section.

\subsection{Footnotes}

{\bf Footnotes}: Put footnotes at the bottom of the page and use 9
points text. They may be numbered or referred to by asterisks or other
symbols.\footnote{This is how a footnote should appear.} Footnotes
should be separated from the text by a line.\footnote{Note the line
separating the footnotes from the text.}

\subsection{Graphics}

{\bf Illustrations}: Place figures, tables, and photographs in the
paper near where they are first discussed, rather than at the end, if
possible.  Wide illustrations may run across both columns.  Color
illustrations are allowed, provided you have verified that  
they will be understandable when printed in black ink.

{\bf Captions}: Provide a caption for every illustration; number each one
sequentially in the form:  ``Figure 1. Caption of the Figure.'' ``Table 1.
Caption of the Table.''  Type the captions of the figures and 
tables below the body, using 11 point text.  

\section{Translation of non-English Terms}

It is also advised to supplement non-English characters and terms
with appropriate transliterations and/or translations
since not all readers understand all such characters and terms.
Inline transliteration or translation can be represented in
the order of: original-form transliteration ``translation''.

\section{Length of Submission}
\label{sec:length}

Long papers may consist of up to 8 pages of content (excluding
references) and short papers may consist of up to 
four (4) pages plus 2 pages for references in the proceedings.  Papers that do not conform to the
specified length and formatting requirements are subject to be
rejected without review.

%\section{Other Issues}

%Papers that had software and/or dataset submitted for the review
%process should also submit it with the camera-ready paper. Besides,
%the software and/or dataset should not be anonymous.

% Please note that the publications of Nodalida-2015 will be publicly
% available at ACL Anthology (http://aclweb.org/anthology-new/) in May,
% 2015. Since some of the authors may have plans to file patents related to their papers in
% the conference, we are reminding authors that May 12th, 2015 may be
% considered to be the official publication date, i.e., the opening
% day of the conference.

\section*{Acknowledgments}

Do not number the acknowledgment section. Do not include this section
when submitting your paper for review.

% If you use BibTeX with a bib file named eacl2014.bib, 
% you should add the following two lines:
\bibliographystyle{acl}
\bibliography{rusaccent}

\end{document}
