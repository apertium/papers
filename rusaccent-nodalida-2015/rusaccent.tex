%
% File nodalida2015.tex
%
% Contact beata.megyesi@lingfil.uu.se
%
% Based on the instruction file for EACL 2014
% which in turn was based on the instruction files for previous 
% ACL and EACL conferences.


\documentclass[11pt]{article}
\usepackage[T2A,T1]{fontenc}
\usepackage[utf8]{inputenc}
\usepackage[russian,english]{babel}

\usepackage{times}
\usepackage{latexsym}
\usepackage{fixltx2e} %allows subscripts
%\usepackage{mathptmx}
%\usepackage{txfonts}
\usepackage{url}
\special{papersize=210mm,297mm} % to avoid having to use "-t a4" with dvips 
%\setlength\titlebox{6.5cm}  % You can expand the title box if you really have to

\usepackage{nodalida2015}

\usepackage{linguex}
\usepackage{needspace}

\newcommand{\rus}[1]{\foreignlanguage{russian}{#1}}

\newcommand{\ft}[1]{\marginpar{\scriptsize F: #1}} % Fran's comments
\newcommand{\rr}[1]{\marginpar{\scriptsize R: #1}} % Rob's comments
\newcommand{\dm}[1]{\marginpar{\scriptsize D: #1}} % Detmar's comments
\newcommand{\lj}[1]{\marginpar{\scriptsize L: #1}} % Laura's comments

\title{Automatic word stress annotation of Russian running text (max 8 pages)}

\author{Author1 \\
  Affiliation / Address line 1 \\
  Affiliation / Address line 2 \\
  Affiliation / Address line 3 \\
  {\tt email@domain.com} \\\And
  Author2 \\
  Affiliation / Address line 1 \\
  Affiliation / Address line 2 \\
  Affiliation / Address line 3 \\
  {\tt email@domain.com} \\}

\date{2015}

\begin{document}
\maketitle
\begin{abstract}
  We evaluate the effectiveness of finite-state tools we developed for 
  automatically annotating word stress in Russian running text. To our knowledge,
  this is the first study to empirically evaluate the results of this task. 
  Given an adequate 
  lexicon with specified stress, the primary obstacle for correct stress 
  placement is disambiguating homographic wordforms. Using a Constraint Grammar, 
  we achieve 93.21\% accuracy, a 3.14\% improvement over the baseline
  undisambiguated performance. 
  Combining our Constraint Grammar with frequency data and a simple
  algorithm for unknown word stress position guessing, we achieve 96.15\%
  accuracy. These results highlight the need morphosyntactic disambiguation
  in the wordstress placement task for Russian, and set a standard for future
  research on this task.
\end{abstract}

%\section{Credits}
%
%The template had a credits section for the history of the template, but we don't need
%a Credits section, right?

\section{Introduction}

Russian word stress is variable and phonemic. The inflecting word classes include complex 
patterns of shifting stress, and a lexeme's stress pattern cannot be predicted 
from surface forms. Lexical stress and its attendant vowel reduction are 
a prominent feature of spoken Russian, and the incorrect placement of stress
can render speech almost incomprehensible. However, standard written Russian does not 
typically mark word stress.\footnote{Texts intended for native speakers sometimes 
mark stress on words that cannot be disambiguated through context, although this 
is rare.} Without information about lexical stress position, 
correctly converting written Russian text to speech is impossible. This is true 
both for humans (e.g. foreign language students) and computers (e.g. text-to-
speech).

We identify three different types of word stress ambiguity.
\emph{Intraparadigmatic} stress ambiguity refers
to homographic wordforms belonging to the same lexeme, as shown in 
\ref{ex:intrahom}. 

\needspace{4\baselineskip}
\ex. Intraparadigmatic homographs \label{ex:intrahom}
\a. \rus{т\'{е}ла} \emph{t\'{e}la} `body.\textsubscript{SG-GEN}' 
    \label{ex:bodySGGEN}
\b. \rus{тел\'{а}} \emph{tel\'{a}} `body.\textsubscript{PL-NOM}' 
    \label{ex:bodyPLNOM}

The remaining two types of stress ambiguity occur between lexemes. 
\emph{Morphosyntactically congruent} stress ambiguity occurs between homographs 
that belong to separate lexemes, 
and whose morphosyntactic values are identical, as shown in
Table~\ref{table:MScongruent}. This kind of stress ambiguity is relatively rare.
Resolving this ambiguity would require the use of technologies such as Word Sense 
Disambiguation.

\begin{table}
\begin{center}
\begin{tabular}{c|ll}
\bf Number-Case & \bf `castle' & \bf `lock' \\
\hline
\texttt{SG-NOM} & \rus{з\'{a}мок} & \rus{зам\'{о}к} \\
\texttt{SG-GEN} & \rus{з\'{a}мка} & \rus{замк\'{а}} \\
\texttt{PL-NOM} & \rus{з\'{a}мки} & \rus{замк\'{и}} \\
\texttt{PL-GEN} & \rus{з\'{a}мков} & \rus{замк\'{о}в} \\
... & \hskip 1em ... & \hskip 1em ... \\
\end{tabular}
\end{center}
\caption{ \label{table:MScongruent} Morphosyntactically congruent homographs. }
\end{table}

\emph{Morphosyntactically incongruent} stress ambiguity occurs between homographs
that belong to separate lexemes, and whose morphosyntactic values are different, 
as shown in \ref{ex:MSincongruent}.

\ex. Morphosyntactically incongruent homographs \label{ex:MSincongruent}
\a. \rus{попад\'{а}ть} \emph{popadát'} `fall.\textsubscript{INF-IPFV}' \\
    \rus{поп\'{а}дать} \emph{popádat'} `fall.\textsubscript{INF-PFV}'
\b. \rus{узнаёт} \emph{uznaët} `finds out.\textsubscript{PRS-IPFV-3SG}' \\
    \rus{узн\'{а}ет} \emph{uznáet} `finds out.\textsubscript{FUT-PFV-3SG}'
\c. \rus{н\'{а}шей} \emph{nášej} `our.\textsubscript{F-SG-GEN/DAT/LOC...}'\\
    \rus{наш\'{е}й} \emph{našéj} `sew on.\textsubscript{IMP-2SG}'
\d. \rus{дор\'{о}га} \emph{doróga} `road.\textsubscript{N-F-SG-NOM}'\\
    \rus{дорог\'{а}} \emph{dorogá} `dear.\textsubscript{ADJ-F-SG-PRED}'

It should be noted that not all morphosyntactic ambiguity leads to stress 
ambiguity. For example, all inflecting modifiers have the same form for F-SG-GEN, 
F-SG-LOC, F-SG-DAT and F-SG-INS. We refer to this as \emph{stress-irrelevant} 
ambiguity, since all readings have
the same stress placement. Stress-irrelevant ambiguity can be intraparadigmatic 
and/or interlexical. 

In the case of running 
text in Russian, most stress placement 
ambiguity is rooted in intraparadigmatic and morphosyntactically incongruent 
ambiguity. Detailed part-of-speech tagging can help determine the stress of 
these forms, since each alternative stress placement is tied to a different
tag sequence. In this study we focus on 
the role of part-of-speech tagging in improving
automatic stress placement, leaving morphosyntactically congruent stress 
ambiguity to future work. \rr{this paragraph is choppy.}

\subsection{Background and Task Definition}

Automatic stress placement in Russian is similar to Diacritic Restoration, a task
which has received increasing interest over the last 20 years. Generally 
speaking, diacritics disambiguate otherwise homographic wordforms, so missing 
diacritics can complicate many NLP tasks, such as text-to-speech. For example, 
speakers of Czech may type emails and other communications
without standard diacritics. In order to generate speech from these texts, they 
must first be normalized by restoring diacritics. A slightly different
situation arises with Arabic and Hebrew, since standard orthography lacks vowel 
diacritics altogether. For such languages, the `restoration' of vowel diacritics 
results in less ambiguity than in standard orthography. For languages with 
inherently ambiguous orthography,
it may be more precise to refer to this as `diacritic enhancement', since it 
produces text that is less ambiguous than the standard language. In this sense, 
Russian orthography is similar to Arabic and Hebrew, since its vowel qualities 
are underspecified in standard orthography.

Many studies of Russian text-to-speech and automatic speech recognition make note of the 
difficulties caused by the shortcomings of their stress-marking resources (e.g. \cite{krivnova_1998}), but very few studies have targeted stress-marking itself, and they tend to
focus on placing stress on unknown words, with almost no attention to the problems that
arise with known words. For example, 
\cite{Xomicevic_2008} developed a set of heuristics for guessing stress placement on 
unknown words in Russian. More recently, \cite{hall_sproat_russian_2013} trained 
a maximum entropy model on a dictionary of Russian words, and evaluated on
wordlists containing `known' and `unknown' wordforms.\footnote{ADD FOOTNOTE
EXPLAINING THEIR DEFINITION OF KNOWN AND UNKNOWN.} Their model 
achieved 98.7\% accuracy on known words, and 83.9\% accuracy on unknown words. 
One major shortcoming of evaluating on wordlists is 
that they do not reflect wordform frequency.
Many of the most problematic stress ambiguities in Russian occur in 
high-frequency wordforms, so evaluations based on wordlists - as opposed to 
running text - have inflated results (see discussion in 
Section~\ref{discussion}). Furthermore, working with running text allows for the
possibility of disambiguating homographs based on syntactic context.

So far, the implicit target application of the few studies related to automatic 
stress placement in Russian has been text-to-speech and automatic speech 
recognition. However, the target application of our stress
annotator is a different domain: language learning. Since standard Russian does 
not contain stress-markings, learners are frequently unable to pronounce unknown 
words correctly without referencing a dictionary or similar resources. In the 
context of language learning, marking stress incorrectly is arguably worse than 
not marking it at all. Because of this, we want our stress annotator to be able 
to abstain from marking stress on words that it is unable to resolve with high 
confidence.

\begin{table*}[t]
  \centering
  \begin{tabular}{r|c|c}
    & \rus{кость}-N-F-SG-GEN\hskip 1em\rus{к\'{о}сти} & \rus{кость}-N-F-SG-GEN\hskip 1em\rus{к\'{о}сти} \\
    & \rus{кость}-N-F-SG-DAT\hskip 1em\rus{к\'{о}сти} & \rus{кость}-N-F-SG-DAT\hskip 1em\rus{к\'{о}сти} \\
    & \rus{костить}-V-IPFV-IMP\hskip 1em\rus{кост\'{и}} & \\
    \hline
    {\small {\tt bare}} & \rus{кости} & \rus{кости} \\
    {\small {\tt safe}} & \rus{кости} & \rus{к\'{о}сти} \\
    {\small {\tt guess}} & \rus{к\'{о}сти} ($p$=0.67) or \rus{кост\'{и}} ($p$=0.33) & \rus{к\'{о}сти}\\
    {\small {\tt guessFreq}} & \rus{к\'{о}сти} & \rus{к\'{о}сти}
  \end{tabular}
  \caption{Example output of each stress placement approach, given a particular set of readings for the token \rus{кости} \emph{kosti}.}
  \label{tab:conditions}
\end{table*}

\subsection{Stress Corpus}

Russian texts with marked word stress are relatively rare, except in materials
for language learners, which are predominantly proprietary. Our corpus was 
collected from
free texts on Russian language-learning websites.\footnote{URLs can be found in
the TROLLing data repository (ADD URL).} This small corpus (N=7689) is 
representative of texts 
that learners of Russian are likely to encounter in their studies. These texts 
include excerpts from well-known literary works, as well as dialogs and 
prose and individual sentences that were written for learners.

Unfortunately, the general practice for marking stress in Russian is to
\emph{not} mark stress on monosyllabic tokens, effectively assuming that
all monosyllabics are stressed. However, this approach 
is not well-motivated. Many words -- both monosyllabic and multisyllabic -- are 
unstressed, especially among prepositions, conjunctions, and particles.
\rr{multisyllabics that can be unstressed: \rus{через, между}} Furthermore, there 
are many
high-frequency monosyllabic homographs that can be either stressed or unstressed, 
depending on their part of speech, or particular collocations. Clearly, for
such words, one cannot assume that they are stressed on the basis of their 
syllable count.

Based on these considerations, we built our tools to mark stress on every
word, both monosyllabic and multisyllabic. However, because our gold corpus texts 
do not mark stress on monosyllabic words, we cannot evaluate our annotation of 
those words.

\section{Automatic Stress Placement}

State-of-the-art morphological analysis in Russian is based on finite-state
technology.\footnote{add footnote or citations for Dialog, mystem, etc.} To our 
knowledge, no open-source resources are available for analyzing and generating 
stressed wordforms. Therefore, we developed free and open-source finite-state 
tools capable of analyzing and generating stressed wordforms, based on the
well-known \emph{Grammatical Dictionary of Russian} \cite{Zaliznjak-77}.
A Finite-State Transducer\footnote{Using two-level 
morphology, xfst/hfst, blah, blah, blah} (FST) generates all possible 
morphosyntactic readings of each wordform, and a constraint grammar\footnote{vislcg3, blah blah} then removes
some extra readings based on syntactic context. The ultimate success of our 
stress placement system depends on the performance of the Constraint Grammar. 
Ideally, the Constraint Grammar would successfully remove all but the correct 
reading for each token, but in practice some
tokens still have more than one reading remaining. Therefore,
we also evaluate various approaches to deal with the remaining ambiguity,
as described in the following paragraph and Table~\ref{tab:conditions}.

The {\small {\tt bare}} approach is to not mark stress on words with more than 
one reading. Since both sets of readings in Table~\ref{tab:conditions} have more
than one reading, {\small {\tt bare}} does not output a stressed form. 

The {\small {\tt safe}} approach is to mark stress only on
tokens whose morphosyntactic ambiguity is stress-irrelevant. In 
Table~\ref{tab:conditions}, the first column has readings that result in two
different stress positions, so {\small {\tt safe}} does not output a stressed
form. However, in the second column, both readings have the same stress position, 
so {\small {\tt safe}} outputs that stress position.

The {\small {\tt guess}} approach is to randomly select one of the available 
readings, and abstain if no readings are available. In the first column of
Table~\ref{tab:conditions}, a random selection means that \rus{к\'{о}сти} 
\emph{k\'{o}sti} is twice as likely as \rus{кост\'{и}} \emph{kost\'{i}}. The
second column of Table~\ref{tab:conditions} contains stress-irrelevant
ambiguity, so a random selection of a reading has the same result as the
{\small {\tt safe}} approach.
 
The {\small {\tt guessFreq}} 
approach is to select the reading that is most frequent, with frequency data
taken from a separate hand-disambiguated corpus. If none of the readings is 
found in the corpus, then {\small {\tt guessFreq}} selects the reading
with the tag sequence (lemma removed) that is most frequent in our corpus. If the 
tag sequence is not found
in our frequency list, then {\small {\tt guessFreq}} backs off to the 
{\small {\tt guess}} algorithm. In the first column of
Table~\ref{tab:conditions}, {\small {\tt guessFreq}} selects \rus{к\'{о}сти} 
\emph{k\'{o}sti} because the tag sequence N-F-SG-GEN is more frequent than the
other alternatives. Note 
that for tokens with stress-irrelevant ambiguity (e.g. the second column of
Table~\ref{tab:conditions}), {\small {\tt guess}} and 
{\small {\tt guessFreq}} produce the same result as the {\small {\tt safe}} 
method.

So far, the approaches discussed are
dependent on the availability of readings from the FST. Two final approaches 
address unknown tokens.
The {\small {\tt guessSyll}} and {\small {\tt guessFreqSyll}} approaches are 
identical to {\small {\tt guess}} and {\small {\tt guessFreq}}, respectively,
except that the -{\small {\tt Syll}} approaches attempt to place stress on
unknown tokens (i.e. tokens not found in the FST) by guessing which 
\emph{syllable} is stressed. Empirical research with
Russian native speakers reading nonce words in running text concludes that
the default stress position in Russian is the last syllable of the stem, or
in other words, the last vowel that is not part of an inflectional ending 
\cite{Crosswhite.ea-03}. Therefore, the -{\small {\tt Syll}} approaches 
place the stress near the end of the stem. Determining 
which vowel is stem-final
is itself a difficult task. Our algorithm merely places the stress on the last 
vowel that is followed by a consonant. Although more sophisticated algorithms
are necessary to accurately locate the stem boundary, this is not the central
focus of this study.

For our baseline, we take the output of our morphological analyzer (without the 
Constraint Grammar) in combination with the {\small {\tt bare}}, 
{\small {\tt safe}}, {\small {\tt guess}} and {\small {\tt guessFreq}} 
approaches. We also compare our outcomes with the 
RussianGram\footnote{http://russiangram.com/} plugin 
for the Google Chrome web browser. RussianGram is not open-source, so we
can only guess what technologies support the service. In any case, it provides a
meaningful reference point for the success of each of the methods described 
above.

\section{Results}

\begin{table*}[t]
  \centering
  \begin{tabular}{r | r r r | r r }
    approach & accuracy\% & error\% & abstention\% & totTry\% & totFail\% \\
    \hline
    \hline
    {\small {\tt noCGbare}} & 30.43 & 0.17 & 69.39 & 30.61 & 69.57 \\
    {\small {\tt noCGsafe}} & 90.07 & 0.49 & 9.44 & 90.56 & 9.93 \\
    {\small {\tt noCGguess}} & 94.34 & 3.36 & 2.30 & 97.70 & 5.66 \\
    {\small {\tt noCGguessFreq}} & 95.53 & 2.59 & 1.88 & 98.12 & 4.47 \\
    {\small {\tt noCGguessSyll}} & 94.99 & 4.05 & 0.96 & 99.04 & 5.01 \\
    {\small {\tt noCGguessFreqSyll}} & 95.83 & 3.46 & 0.72 & 99.28 & 4.17 \\
    \hline
    {\small {\tt CGbare}} & 45.78 & 0.44 & 53.78 & 46.22 & 54.22 \\
    {\small {\tt CGsafe}} & 93.21 & 0.74 & 6.05 & 93.95 & 6.79 \\
    {\small {\tt CGguess}} & 95.50 & 2.59 & 1.90 & 98.10 & 4.50 \\
    {\small {\tt CGguessFreq}} & 95.73 & 2.40 & 1.88 & 98.12 & 4.27 \\
    {\small {\tt CGguessSyll}} & 95.92 & 3.33 & 0.74 & 99.26 & 4.08 \\
    {\small {\tt CGguessFreqSyll}} & 96.15 & 3.14 & 0.72 & 99.28 & 3.85 \\
    \hline
    {\small {\tt RussianGram}} & 90.09 & 0.79 & 9.12 & 90.88 & 9.91
  \end{tabular}
  \caption{Results of stress placement task evaluation. (N = 4048)}
  \label{tab:results}
\end{table*}

Single-syllable words and words that are unstressed in the gold standard were
not evaluated for reasons discussed above, leaving 4048 tokens for evaluation.
Since our approach is lexicon-based, some of our results should be interpreted
with respect to how many of the stressed wordforms in the gold standard corpus
can be found in the output of the finite-state transducer. We refer to this 
measure as \emph{recall}. Out of 4048 tokens, 3949 were 
found in the FST, which is equal to 97.55\%.
This number represents the ceiling for methods relying on the FST.
Higher scores are only achievable by expanding the FST or by using syllable-
guessing algorithms. After running the Constraint Grammar, recall was 97.35\%, 
a reduction of 0.20\%. Apparently the Constraint Grammar erroneously removes some 
correct readings, or else there are errors in the gold corpus.

Results were compiled for each of the 13 approaches discussed above: 
\emph{without} the Constraint Grammar ({\small {\tt noCG}}) x 6 approaches, 
\emph{with} the Constraint Grammar ({\small {\tt CG}}) x 6 approaches, 
and RussianGram ({\small {\tt RussianGram}}). Results are given in 
Table~\ref{tab:results}. Each token was categorised as either an accurate output, 
or one of two categories of failures: errors and abstentions. 
If the stress tool outputs a stressed wordform, and it is incorrect, then it is
counted as an `error'. If the stress tool outputs an unstressed wordform, then
it is counted as an `abstention'.

The right-most columns in 
Table~\ref{tab:results} combine values of the basic categories. The term `totTry' 
refers to the sum of the accuracy and error rate. This figure represents the 
proportion of tokens on which our system output a stressed wordform. In the case 
of {\small {\tt noCGbare}}, the accuracy\% (30.43) and error\% (0.17) sum to the 
totTry value of 30.61. The term `totFail' refers to the sum of error rate 
and abstention rate, which is the 
proportion of tokens for which the system failed to output the correct stressed 
form. In the case of {\small {\tt noCGbare}}, the error\% (0.17) 
and abstention\% (69.39) sum to the (rounded) totFail value of of 69.57. 

\rr{Are differences significant?}

The {\small {\tt noCGbare}} approach achieves a baseline accuracy of
30.43\%, so roughly two thirds of the tokens in our corpus are
morphosyntactically ambiguous. The error rate of 0.17\% primarily represents 
forms whose stress position varies from speaker to speaker (e.g.
\rus{зав\'{и}л\'{и}сь} \emph{zav\'{i}l\'{i}s'}), errors in the gold corpus (e.g. 
\rus{вер\'{и}м} \emph{ver\'{i}m}) . The {\small {\tt noCGsafe}} approach
achieves a 60\% improvement in accuracy (90.07\%), which means that 89.39\% of 
morphosyntactic ambiguity on our corpus is stress-irrelevant. Interestingly, the 
RussianGram web service achieves results that are very close 
to the {\small {\tt noCGsafe}} approach, so it would appear that it 
utilizes a similar approach.

Since the ceiling recall for the FST is 97.55\%, and since the 
{\small {\tt noCGsafe}} approach achieves 90.07\%, the maximum improvement 
that a Constraint Grammar could theoretically achieve is 7.48\%. A comparison
of {\small {\tt noCGsafe}} and {\small {\tt CGsafe}} reveals an improvement
of 3.14\%, which is about 42\% of the way to the ceiling recall.\rr{...while raising the error rate 0.25\%}

The {\small {\tt CGguess}} and {\small {\tt CGguessFreq}} approaches are also 
limited by the 97.55\% ceiling from the FST, and their accuracies achieve
improvements of 2.29\% and 2.52\%, respectively, over {\small {\tt CGsafe}}.
However, these gains come at the cost of error rates as much as 3.5 times 
higher than {\small {\tt CGsafe}}: +1.85\% and +1.66\%, respectively. It is
not surprising that {\small {\tt CGguessFreq}} has higher accuracy and a lower
error rate than {\small {\tt CGguess}}, since frequency-based guesses are by
definition more likely to occur. The frequency data were taken from a very 
small corpus, and it is likely that frequency from a larger corpus would yield 
better results.

The two {\small {\tt Syll}} approaches were designed to make a blind guess on 
every wordform that is not found in the FST, which would ideally result in an
abstention rate of 0\%. However, the abstention rates of approximately 0.7\%
are the result of the fact that some words in the FST, especially proper nouns, 
had not been assigned stress. Because the FST outputs a form -- albeit unstressed 
-- the {\small {\tt Syll}} algorithm is not called. This means that 
{\small {\tt Syll}} is only guessing on about 2\% of the tokens. The improvement
on overall accuracy from {\small {\tt CGguessFreq}} to 
{\small {\tt CGguessFreqSyll}} is 0.42\%, which means that the 
{\small {\tt Syll}} guess was accurate 21\% of the time.

\rr{Examine errors and categorize/refute them.}

\section{Discussion} \label{discussion}

One of the main points of this paper is to highlight the importance of 
syntactic context in the Russian wordstress placement task. 
\newcite{hall_sproat_russian_2013} achieve 98.7\% accuracy on stress placement of 
individual wordforms. To 
understand this result, it should be remembered that Russian has a considerable
number of words that are ambiguous with regard to stress. Based on the surface 
forms in our FST -- which is based on the same dictionary used for 
\newcite{hall_sproat_russian_2013} -- 
we calculate that only 29~518 (1.05\%) 
of the 2~804~492 wordforms contained in our FST are stress-ambiguous. \rr{These 
stress-ambiguous wordforms have an average of 2.01 possible stress positions.} 
Compare this with the finding of our experiment, that in running 
text, at least 7.5\% of the tokens are stress-ambiguous. Stress ambiguity is
therefore 7 times more prevalent in our corpus of running text than it is in 
wordform dictionaries. This is at least partly because many of 
the lexemes with shifting stress patterns are high-frequency. Any system that
cannot leverage syntactic context to disambiguate these wordforms cannot 
achieve accuracy above 92.5\% without making an higher number of errors.
Since the task of wordstress placement is virtually always performed on
running text, it seems prudent to make use of surrounding information.
The experiment described in this paper demonstrates that a Constraint Grammar
can effectively improve the accuracy of a stress placement system without 
significantly raising the error rate. Our Russian Constraint Grammar is
under continual development, so we expect higher accuracy in the future.

We are unaware of any other studies of Russian wordstress placement in running
text. The results of our experiment are promising, but many questions remain
unanswered.
The experiment was limited by properties of the gold standard corpus, including
its size, genre distribution, and quality. Ideally, a gold standard corpus for 
Russian wordstress should mark 
whether monosyllabic words are stressed, and of course its annotations should
be accurate. Our gold corpus represents a broad variety of text genres, which 
makes our results more generalizable, but a larger corpus would allow for
evaluating each genre individually. For example, the vast majority of words with
shifting stress are of Slavic origins, so we expect a genre such as technical
writing to have a lower proportion of words with stress ambiguity, since it
uses a more modern lexicon that has not inherited the idiosyncrasies that lead
to wordstress ambiguity.

In addition to genre, it is also likely that text
complexity affects the difficulty of the stress placement task. The distribution
of different kinds of syntactic constructions vary with text complexity 
\cite{Vajjala.Meurers-12}, and
we expect that the effectiveness of the Constraint Grammar will be affected by
those differences.

The resources needed for machine-learning --
such as a large corpus of Russian running text with marked stress -- are simply
not available at this time. Even so, lexicon- and rule-based approaches have
some advantages over machine-learning approaches. For example, we are able to
abstain from marking stress on tokens whose morphosyntactic ambiguity cannot be
adequately resolved. In language-learning applications, this reduces the 
likelihood of learners being exposed to incorrect wordforms, and accepting them
as authoritative. Such circumstances can lead to considerable frustration and
lack of trust in the learning tool. However, machine-learning does seem well-
suited to
placing stress on unknown words, since morphosyntactic analysis is problematic.
The syllable-guessing algorithm {\small {\tt Syll}} used in this experiment was 
overly simplistic, and so it was not surprising that it was only moderately 
successful. More rigorous rule-based approaches have been suggested in other 
studies \cite{Church-85,Williams-87,Xomicevic_2008}. For example, 
\newcite{Xomicevic_2008} attempt to parse the unknown token by matching known 
prefixes and suffixes.

Other studies have applied machine-learning to guessing stress of unknown words
\cite{Pearson.ea-00,Webster-04,Dou.ea-09,hall_sproat_russian_2013}. For example,
for unknown words, \newcite{hall_sproat_russian_2013} achieve an accuracy of 
83.9\%. Implementing a similar model in a rule-based system like ours would allow 
for some improvements to this result. Their model was trained on a full list of 
Russian words, which is not representative of the words that would be unknown to 
a system like ours. Most of the complicated wordstress patterns are a closed 
class\footnote{The growing number of masculine nouns with shifting stress 
(\emph{d\'{o}ktor$\sim$doktor\'{a}}) are an exception to this generalization.},
so we could exclude closed classes of words from the training data, leaving only
word classes that are likely to be similar to unknown tokens, such as those with 
productive derivational affixes.

Future work/potential applications:
Two sections: known words and unknown words
\begin{itemize}
\item As with any lexicon- and rule-based system, those components can be expanded and improved, especially considering the short span over which they were developed.
\item Another relevant issue is whether the ambiguity can be resolved at all. For example, when is stress marked in authentic by-native-for-native texts? A computer should not be expected to predict the stress on such words (although a computer could possibly identify words that should be marked for 
natives.)

\end{itemize}
%\section*{Acknowledgments}
%
%Do not number the acknowledgment section. Do not include this section
%when submitting your paper for review.

\bibliographystyle{acl}
\bibliography{rusaccent}

\end{document}
