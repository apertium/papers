%
% File eamt15.tex, an adaptation of eamt14.tex (which was a copy of
% eamt12.tex)
%
% Contact: eamt2015@dlsi.ua.es

%%% To ease future customizations, various replaceables have been paramaterized
%%% as listed in the newcommands section

\documentclass[11pt]{article}
\usepackage{eamt15}
\usepackage{times}
\usepackage{latexsym}
\setlength\titlebox{6.5cm}    % Expanding the titlebox
%%% YOUR PACKAGES BELOW THIS LINE %%%

\usepackage{multirow}

\newcommand{\confname}{EAMT 2015}
\newcommand{\website}{http://www.eamt2015.org/}
\newcommand{\contactname}{the conference chairs (Felipe
  S\'anchez-Mart\'inez, Gema Ram\'irez-S\'anchez and Fred Hollowood)}
\newcommand{\contactemail}{eamt2015@dlsi.ua.es}
\newcommand{\conffilename}{eamt15}
\newcommand{\downloadsite}{http://www.eamt2015.org/}
\newcommand{\paperlength}{$8$ (eight)}
\newcommand{\shortpaperlength}{$4$ (four)}

\title{A gap-filling approach to evaluation of machine translation for assimilation}

\author{First Author\\
  Affiliation / Address line 1\\
  Affiliation / Address line 2\\
  Affiliation / Address line 3\\
  {\tt email@domain}  \And
  Second Author\\
  Affiliation / Address line 1\\
  Affiliation / Address line 2\\
  Affiliation / Address line 3\\
  {\tt email@domain}}

\date{}

\begin{document}
\maketitle
\begin{abstract}
   ... 
\end{abstract}

\section{Introduction}

As suggested by [Church, Hovy 1993], modern machine translation (MT) systems may
be divided into two broad categories according to their purpose: post-editing systems, e.g.
Google Translate 1 , and assimilation systems, e.g. Apertium 2 . Both kinds may be
systematically evaluated, either to control for quality in the development process or to
compare the systems. The evaluation methods, however, differ depending on the type of
the system. The methods are described in detail in [Church, Hovy 1993].
In this work I perform the evaluation of an assimilation MT system called Apertium.
Apertium is an open-source community-maintained project that provides translation for
39 language pairs. The goal of this study is to assess whether machine translation
performed by this system assists users in understanding the gist of texts in an unknown
language. The evaluation is performed in three separate experiments for the following
translation directions: Basque to Spanish, English to Kazakh, and Tatar to Russian.
I build on the methodology established in [Trosterund, Unhammer 2012] and
\cite{oregan2013}, which are the works describing previous evaluations of
Apertium language pairs. [O’Regan, Forcada 2013] proposes the gap-filling task as a
means of assessing participant’s understanding of text. The same methodology is used in
this work and described in detail in the section below.

\section{Experiment design}

\subsection{Methodology}

This experiment is based on an idea that understanding of text may be assessed through
gap-filling questions. If the participant is able to restore principal information that had
been left out from the text, they are assumed to have understood it.
During the course of the experiment, the participants are offered sentences in their native
language (the target language of the pair), with some key words left out. Each sentence
has a counterpart in the source language of the pair. I use four different task settings:
\begin{itemize}
\item  Target sentence only. The participants are asked to fill the gaps not being given
any context. This task serves as a baseline score and as an indicator of phrases
that can be completed using common knowledge or language intuition (e.g.
idioms and strong collocations);
\item  Target sentence and source sentence. In addition to the question, the participants
are offered the same sentence in the source language of the pair. By setup, the
participants have no command of this language, however, it may help them to fill
in proper nouns or loan words;
\item  Target sentence and machine-translated hint. In addition to the task, the
participants see the source sentence translated via Apertium. This type of task is
used for measuring the contribution of machine translation to understanding the
gist of text;
\item  Target sentence and both the source and machine-translated hint. This task is
added mostly for completing the possible combination of task types, and to check
for any unexpected insights, should they arise.
\end{itemize}

\subsection{Material}

The data are collected separately for each experiment. A set of sentences is drawn from
parallel corpora. The source sentences are translated into target language via Apertium.
The target sentences are processed with a special script that removes significant words.
For the current experiments, the word is removed if it falls into a list of allowed parts of
speech:
\begin{itemize}
\item nouns;
\item proper nouns;
\item adjectives;
\item adverbs;
\item lexical verbs.
\end{itemize}
Parts of speech are determined automatically with Apertium morphological analyser. To
rule out homonymy, the set of all possible POS attributions of a given word is checked
against the POS list. If all of the possible word analyses match the parts of speech listed
above, it is considered a candidate for removal.
Each sentence in the set is evaluated in three “gap densities” (ratio of gaps to words in the
sentence): 10\%, 20\% and 30\%. After the script finds all possible keyword candidates in
the sentence, it removes the specified percentage of words, evenly distributing the gaps
across the sentence.
To exclude headlines and also to allow at least one gap in 10\%-gap sentences, only the
sentences over 10 words long are used in the tasks.
The sentences for each experiment are drawn from various parallel text sources:
\begin{enumerate}
\item  For Basque-Spanish, from the corpus of legal texts Memorias de traducción del
Servicio Oficial de Traductores del IVAP 3
\item  For English-Kazakh, from the official website of the President of the Republic of
Kazakhstan 4
\item  For Tatar-Russian, from the following sources on three different topics:
  \begin{enumerate}
    \item  Casual conversations, from a textbook 5 of spoken Tatar;
    \item  Legal texts, from the Constitution and laws 6 of Tatarstan;
    \item  News, from the President of Tatarstan website 7 .
  \end{enumerate}
\end{enumerate}

Each set features 36 pairs of sentences. For the first two experiments the pairs are drawn
randomly from the corpora, for Tatar-Russian, compiled by hand by the developer of the
language pair. After processing, the material is transformed into XML used for uploading
into the evaluation system (see section 2.4).

\subsection{Participants}

For each experiment we called for native speakers of L2 of the language pair (i.e.
Spanish, Kazakh, and Russian) who had no command of L1 of the pair (Basque, English
and Tatar, respectively). We had 11 participants in Basque-Spanish evaluation, 8 in
English-Kazakh, and 28 in Tatar-Russian (although not everyone has completed their task
in full, see discussion).

\subsection{Procedure}

The evaluations take place online, in a system 8 designed specifically for various MT
evaluation tasks. We adapted the code to accommodate for the gap-filling tasks. The
tasks are uploaded into the system and manually distributed between the participants by
the following rules:
\begin{enumerate}
\item  Each participant evaluates every sentence (understood as a succession of words),
a total of 36;
\item  Each participant evaluates 9 sentences in each of the four modes (section 2.1);
\item  All sentences of the set are evaluated with 10, 20 and 30\% of words removed;
\item  Each sentence-mode-percentage combination is evaluated by at least three
\end{enumerate}
The participants are given the instructions in their native language above each task. The
tasks are split into smaller sets for the participants’ convenience. The instructions are to

read all the available hints and fill each gap with one suitable word, guessing if unsure.
Participants’ answers are recorded and marked correct or incorrect automatically. In
addition, the time taken to fill the gaps in one sentence is recorded.
Since gap-filling task requires open answers, it is possible that the participants may
provide words that fit the gaps well, but do not match the original answer. To account for
these cases, we process all the answers to detect possible synonyms. An answer is
considered a candidate synonym if given by two or more evaluators and does not match
the answer key. We record each candidate synonym along with the answer key and the
context sentence. Based on this data, a native speaker of the target language decides
whether the candidate synonym is indeed synonymic to the answer key in the given
context. We then check participants’ results against the compiled synonym list and
increase scores where appropriate.

\section{Expected results}

We expect that, for tasks with MT assistance, the users will fill more gaps correctly,
which would correspond to better understanding of text. Therefore, the percentage of
correct answers should be higher for MT and two-hint kinds of tasks than for tasks with
source sentence as a hint and no hint at all. If the percentage were lower, we would
expect an error in experiment design.

\section{Results}

The experiment results are presented in the tables below. The first table of each set shows
the proportion of correct answers depending on evaluation mode and gap density. The
percentage is spread across all evaluators, that is, we divide the number of correct
answers given by all evaluators by the number of questions answered by all evaluators.
The second table of the set shows average time it took to fill the gaps in one sentence. To
reduce the noise from participants who were distracted during evaluation, when
calculating times we remove all the results over 6 minutes (the mode is approximately
two minutes).

\begin{table}
  \begin{tabular}{|l|r|r|r|r|}
    \hline
    \multirow{2}{*}{\textbf{Density}} & \multicolumn{4}{|c|}{Evaluation mode} \\\cline{2-5}
                                            & \textbf{MT \& Src} & \textbf{MT} & \textbf{Src} & \textbf{No hint} \\
    10\%                                    &   58.02            & 55.70       & 35.06        & 45.45        \\
    20\%                                    &   65.96            & 68.42       & 33.85        & 33.56        \\
    30\%                                    &   49.84            & 37.28       & 24.23        & 18.93        \\
    \hline
  \end{tabular}
  \caption{Basque--Spanish: Gaps successfully filled (\%), using a synonym list}
  \label{table:res-eus-spa}
\end{table}

\begin{table}
  \begin{tabular}{|l|r|r|r|r|}
    \hline
    \multirow{2}{*}{\textbf{Density}} & \multicolumn{4}{|c|}{Evaluation mode} \\\cline{2-5}
                                            & \textbf{MT \& Src} & \textbf{MT} & \textbf{Src} & \textbf{No hint} \\
    10\%                                    & 34.78              & 39.02       & 40.00        & 31.43        \\
    20\%                                    & 43.84              & 33.80       & 47.83        & 60.34        \\
    30\%                                    & 41.38              & 36.56       & 44.94        & 31.09        \\
    \hline
  \end{tabular}
  \caption{English--Kazakh: Gaps successfully filled (\%), using a synonym list}
  \label{table:res-eng-kaz}
\end{table}

\begin{table}
  \begin{tabular}{|l|r|r|r|r|}
    \hline
    \multirow{2}{*}{\textbf{Density}} & \multicolumn{4}{|c|}{Evaluation mode} \\\cline{2-5}
                                            & \textbf{MT \& Src} & \textbf{MT} & \textbf{Src} & \textbf{No hint} \\
    10\%                                    & 53.89              & 66.06       & 54.43        & 46.34        \\
    20\%                                    & 65.33              & 60.73       & 48.39        & 39.43        \\
    30\%                                    & 59.90              & 52.59       & 39.66        & 38.39        \\
    \hline
  \end{tabular}
  \caption{Tatar--Russian: Gaps successfully filled (\%), using a synonym list}
  \label{table:res-tat-rus}
\end{table}



\section{Discussion}

The following trends may be derived from the data above:
\begin{enumerate}
\item  In general, tasks with MT assistance score higher than task without MT
assistance;
\item  The more words are removed from context, the more difficult it is to fill the gaps
(see below for exceptions);
\item  Task without hints take less time to complete than others.
\end{enumerate}
The first two trends align with our expectations. Higher scores on MT tasks indicate
better understanding of text. The large number of gaps in the sentence makes it more
difficult to predict the answer based on the context, and also leaves more room for
translation mistakes. We would expect that tasks without hints take the largest amount of
time rather than that the smallest, since the participant must come up with their own
answer unassisted. However, in the no-hint task the participant reads only one sentence,
as opposed to two or three in other tasks, which may be the source of difference.
In Basque-Spanish and Tatar-Russian these trends are defined reasonably well. In
English-Kazakh, however, the scores for source-hint task (a parallel sentence in English
for assistance) are systematically higher than for other types of tasks. In addition, a 20\%no-hint task stands out with 60\% accuracy, which is substantially higher than any other
scores for this experiment.
Having analyzed the tasks and the responses for English-Kazakh, I have found the
following possible reasons of the obtained results:
\begin{enumerate}
\item  The experiment was planned for 11 evaluators, although only 8 attempted the
tasks and 6 completed them in full. Because of this, the sentences were evaluated
once or twice, and some were not attempted at all. To reliably compare results
across different categories, all tasks must be completed;
\item  Due to an error during experiment design, some sentences are not repeated across
categories, which also suggests that we may not compare these percentages;
\item  The outlier score of 60\% is made up by three sentences. Two of these scored
unusually high. It appeared that one of them is a news headline, and the other
contains clichés such as “express condolences”. There may also be interference
from other tasks of the set, because some of the sentences are drawn from the
same news article. We would expect the same sentences to contribute to higher
percentage in other task modes, but they were not completed there;
\item  It may be the case that the participants have at least some command of English,
which would have helped them to derive extra information in the source-hint
\end{enumerate}
It remains questionable whether we can compare results for different gap densities. In the
Basque-Spanish and Tatar-Russian experiments the 10, 20 and 30\% sets were comprised
of the same sentences. However, in each case different words were removed. It appears
that some content words are easier to fill than the others. This may explain why in
Basque-Spanish the 20\% MT tasks are completed with better accuracy than 10\% tasks.
It may be worth noting that the majority of the participants reported feeling frustrated in
the course of their work, especially while working on the no-hint tasks. 9 out of 51
participants quit the experiment before completing it.

\section{Future work}
The experiment may be repeated for any language pair (provided a parallel corpus) and
any machine translation system. To enhance the results, the procedure may be amended
in the following ways:
\begin{enumerate}
\item  Reduce the number of questions per participant. In the above experiments each
participant filled from 110 to 187 gaps, divided into small groups. Reducing the
amount of work may increase task completion rate;
\item  Increase the number of participants per experiment. The more evaluators work on
every question, the more reliable the data;
\item  Design the tasks with different gap densities in such a way that e.g. the 20\% task
contains all the gaps of the 10\% task, plus any additional ones;
\item  Draw sentences from large corpora. This will reduce the influence of previous
tasks, which may provide the context for other tasks in the set.
\end{enumerate}


\bibliographystyle{apalike}

\bibliography{2015-eamt-assim}

\end{document}
