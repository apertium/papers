%
% File nodalida2015.tex
%
% Contact beata.megyesi@lingfil.uu.se
%
% Based on the instruction file for EACL 2014
% which in turn was based on the instruction files for previous 
% ACL and EACL conferences.


\documentclass[11pt]{article}
\usepackage[T2A,T1]{fontenc}
\usepackage[utf8]{inputenc}
\usepackage[russian,english]{babel}

\usepackage{times}
\usepackage{latexsym}
\usepackage{fixltx2e} %allows subscripts
%\usepackage{mathptmx}
%\usepackage{txfonts}
\usepackage{url}
\usepackage[small,bf]{caption}
\special{papersize=210mm,297mm} % to avoid having to use "-t a4" with dvips 
%\setlength\titlebox{6.5cm}  % You can expand the title box if you really have to

\usepackage{nodalida2015}

\usepackage{linguex}
\usepackage{needspace}

\newcommand{\rus}[1]{\foreignlanguage{russian}{#1}}

\newcommand{\ft}[1]{\marginpar{\scriptsize F: #1}} % Fran's comments
\newcommand{\rr}[1]{\marginpar{\scriptsize R: #1}} % Rob's comments

\title{A preliminary constraint grammar for Russian}

\author{Francis M. Tyers \\
  HSL-fakultehta, \\
  UiT Norgga árktalaš universitehta, \\
  N-9018 Romsa \\
  {\tt francis.tyers@uit.no} \\\And
  Robert Reynolds \\
  HSL-fakultehta, \\
  UiT Norgga árktalaš universitehta, \\
  N-9018 Romsa \\
  {\tt robert.reynolds@uit.no} \\}

% \author{Author1 \\
%   Affiliation / Address line 1 \\
%   Affiliation / Address line 2 \\
%   Affiliation / Address line 3 \\
%   {\tt email@domain.com} \\\And
%   Author2 \\
%   Affiliation / Address line 1 \\
%   Affiliation / Address line 2 \\
%   Affiliation / Address line 3 \\
%   {\tt email@domain.com} \\}
% 
\date{2015}

\begin{document}
\maketitle
\begin{abstract}
 This paper presents preliminary work on a constraint
 grammar based disambiguator for Russian. Russian is
 a Slavic language with a high degree of both in-category
 and out-category homonymy in the inflectional system.
 The pipeline consists of a finite-state morphological
 analyser and constraint grammar. The constraint 
 grammar is tuned to be high recall at the expense of 
 low precision.
\end{abstract}

\section{Introduction}

Our purposes: Processing stressed wordforms and open-source machine translation

\cite{Karlsson-90}

% objectives:
%% make a free/open-source morphology for russian based on Z's dictionary that includes stress

\section{Review of literature}
% ROB + FRAN
% different CGs 
% faroese
% estonian

State-of-the-art morphological analysis in Russian is primarily based on finite-
state technology \cite{Nozhov-03,Segalovich-03}, although some success has 
also been found with machine-learning approaches \cite{sharoff08lrec-mocky}. 
However, no existing Russian morphology tools were suitable for our needs, so
we have 
The resources described in this paper Therefore, we developed free and 
open-source finite-state 
tools capable of analyzing and generating stressed wordforms, based on the
well-known \emph{Grammatical Dictionary of Russian} \cite{Zaliznjak-77}.
Our Finite-State Transducer\footnote{Using two-level 
morphology \cite{Koskenniemi-84}, implemented in both xfst
\cite{Beesley.Karttunen-03} and hfst \cite{hfst-11}} (FST) generates all possible 
morphosyntactic readings of each wordform, and our Constraint 
Grammar\footnote{Implemented using vislcg3 constraint grammar parser
(http://beta.visl.sdu.dk/cg3.html).}
\cite{Karlsson-90,Karlsson.Voutilainen.ea-95} then removes
some readings based on syntactic context. 

\cite{trosterud2009}

\section{Ambiguity in Russian}

\section{Pipeline}

\subsection{Morphological analyser}
% ROB

\cite{yablonsky1999russian}

\subsection{Disambiguation rules}

\begin{table}
  \centering
  \begin{tabular}{lrrr}
    \hline
         & {\sc select} & {\sc remove} & {\sc map} \\
    Safe & & & \\
    Safe heuristic  & & & \\
    Heuristic &  & & \\
    Syntax labelling &  &   & \\ 
    \hline
  \end{tabular}
  \caption{Rules in the grammar are separated into four sections. }
\end{table}

\section{Development process}

\subsection{Combining with a statistical tagger}

\section{Evaluation}

\begin{table*}
  \centering
  \begin{tabular}{|l|r|r|r|r|}
    \hline
    \textbf{Domain} & \textbf{Tokens} & \textbf{Precision} & \textbf{Recall} & \textbf{Ambig. solved} \\
    \hline
    Wikipedia       & 7,857      & 0.506        & 0.996    & 44.92\%  \\
    Literature      & 1,652      & 0.473        & 0.984    & 42.95\%  \\
    News            & 642        & 0.471        & 0.990    & 41.60\%  \\
    \hline
    \textbf{Average}& 10,150     & 0.498        &  0.994   &  44.39\% \\
    \hline
  \end{tabular}
  \caption{Results for the test corpora.}
\end{table*}


\begin{table}
  \centering
  \begin{tabular}{|l|r|r|}
    
    \hline
    Error in original        &   & 2 \\
    Missing lexeme           &   & 8  \\          % безбалластный, FALSENEG: Кий<np><ant><m><sg><ins> ['кий<n><m><nn><sg><ins>']
    Partial lexeme           &   & 2  \\          % франко-норманнское, рельс
    Missing analysis         &   &    \\        % 
    Equally valid analysis   &   &    \\   % большой<adj><comp><pred> ['больше<adv>'] (?)
    Correct reading removed  &   & 34   \\ % 
    \hline
                             &   & 46 \\
    \hline
  \end{tabular}
  \caption{Error analysis of false negatives}
\end{table}

% unknown words from test corpus, categorisation

% rule error analysis

\section{Future work}

% func analysis
% dep analysis

\section{Conclusions}


%\section*{Acknowledgments}
%
%Do not number the acknowledgment section. Do not include this section
%when submitting your paper for review.

\bibliographystyle{acl}
\bibliography{ruscg}

\end{document}
