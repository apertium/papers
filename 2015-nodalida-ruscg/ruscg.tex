%
% File nodalida2015.tex
%
% Contact beata.megyesi@lingfil.uu.se
%
% Based on the instruction file for EACL 2014
% which in turn was based on the instruction files for previous 
% ACL and EACL conferences.


\documentclass[11pt]{article}
\usepackage[T2A,T1]{fontenc}
\usepackage[utf8]{inputenc}
\usepackage[russian,english]{babel}

\usepackage{times}
\usepackage{latexsym}
\usepackage{fixltx2e} %allows subscripts
%\usepackage{mathptmx}
%\usepackage{txfonts}
\usepackage{url}
\special{papersize=210mm,297mm} % to avoid having to use "-t a4" with dvips 
%\setlength\titlebox{6.5cm}  % You can expand the title box if you really have to

\usepackage{nodalida2015}

\usepackage{linguex}
\usepackage{needspace}

\newcommand{\rus}[1]{\foreignlanguage{russian}{#1}}

\newcommand{\ft}[1]{\marginpar{\scriptsize F: #1}} % Fran's comments
\newcommand{\rr}[1]{\marginpar{\scriptsize R: #1}} % Rob's comments

\title{A preliminary constraint grammar for Russian}

\author{Author1 \\
  Affiliation / Address line 1 \\
  Affiliation / Address line 2 \\
  Affiliation / Address line 3 \\
  {\tt email@domain.com} \\\And
  Author2 \\
  Affiliation / Address line 1 \\
  Affiliation / Address line 2 \\
  Affiliation / Address line 3 \\
  {\tt email@domain.com} \\}

\date{2015}

\begin{document}
\maketitle
\begin{abstract}
  Hargle bargle
\end{abstract}

\section{Introduction}

\cite{Karlsson-90}

\section{Review of literature}
% ROB + FRAN
% different CGs 
% faroese
% estonian
% 

\cite{trosterud2009}

\section{Ambiguity in Russian}

\section{Pipeline}

\subsection{Morphological analyser}
% ROB

\subsection{Disambiguation rules}

\begin{table}
  \begin{tabular}{lrrr}
    \hline
         & {\sc select} & {\sc remove} & {\sc map} \\
    \hline
  \end{tabular}
\end{table}

\section{Development process}

\subsection{Combining with a statistical tagger}

\section{Evaluation}

\begin{table}
  \begin{tabular}{|l|r|r|}
    \hline
    Missing lexeme & &  \\          % безбалластный, FALSENEG: Кий<np><ant><m><sg><ins> ['кий<n><m><nn><sg><ins>']
    Partial lexeme & &  \\          % франко-норманнское
    Missing analysis & &  \\        % 
    Equally valid analysis & & \\   % большой<adj><comp><pred> ['больше<adv>'] (?)
    Correct reading removed & &  \\ % 
    \hline
  \end{tabular}

\end{table}

% unknown words from test corpus, categorisation

% rule error analysis

\section{Future work}

\section{Conclusions}


%\section*{Acknowledgments}
%
%Do not number the acknowledgment section. Do not include this section
%when submitting your paper for review.

\bibliographystyle{acl}
\bibliography{ruscg}

\end{document}
