\documentclass[a5paper]{article}

\usepackage{amssymb}
\usepackage{polyglossia}
\usepackage[shortcuts]{extdash}

\usepackage{fontspec}
\usepackage{xunicode}
\usepackage{xltxtra}

\usepackage{url}
\usepackage{hyperref}
\usepackage[obeyDraft]{todonotes}

\usepackage{graphicx}
\usepackage{geometry}

\setmainfont[Mapping=tex-text]{Liberation Serif}

\title{Finite-State Spell-Checking for Turkic Minority Languages}
\author{Francis M. Tyers \and Jonathan North Washington \and Ilnar Salimzyanov \and Tommi A. Pirinen}

\date{\today}

\begin{document}

\maketitle

\begin{abstract}
    In this article there's spell-chekcers and Turkic languages
\end{abstract}

\section{Introduction}
\label{sec:introduction}

Spell-checking is one of the most commonly used and most basic pieces of
software for language support. It is used by all platforms from regular
computers to tablets and mobile phones, and majority of software from office
suites to web browsers. Furthermore, the role of spell-checking softwares has
been increasing in modern platforms, as limited input systems require more
assistance from spell-checker to predict users input than traditional keyboard
inputs. The contemporary systems for spell-checking are mostly based on
word-form lists and statistics, and some basic form of suffix stripping and
word-form compounding. For language modeling in general there are well known
and more established models for use in morphologically complex languages. In
this article we use well-established finite-state spell-checking
technology~\cite{pirinen2012compiling} in conjunction with Turkic minority
languages and compare the results against traditional word-form list
approaches.

\section{Methods}

Finite-state spell-checking~\cite{agata2002typographical} is well-established
method for handling spell-checking of morphologically complexer languages.
Specifically weighted finite-state spell-checking has been studied and extended
in last years~\cite{pirinen2010finitestate}. In this article we use those
methods to build finite-state spell-checkers for Türkich languages. The
extension of finite-state spell-checkers by hunspell error models is done
as described iñ~\cite{pirinen2010building}.

\section{Data}

For the language models we have selected freely available open source dictionaries
from Apertium project\footnote{\url{}}, and translated them to finite-state
spell-checking systems as described in~\cite{pirinen2012compiling}. The
errors have been discovered from Wikipedia using methods described 
in~\cite{max2010mining}. The hunspell data has been acquired from the internet.


\section{Evaluation}

To evaluate our finite-state spell-checkers we have applied them and hunspell
to our hand-picked error corpus and measured the positions of correct
corrections as shown in table~\ref{table:precision}

We have also briefly measured speed and memory usage of the spell-checkers when
applied to large sections of running text containing both correctly written
strings and mistakes. The results are in table~\ref{table:efficiency}

\section{Discussion}

We have shown that spell-checking works nicely.

\subsection{Error Analysis}

It is interesting that... because edit distance of 2 does not work very well
for Turkic languages.

\subsection{Future Work}

More betterer error models should be built.

\section{Conclusion}

In this article we studied the use of finite-state spell-checking as an
alternative to other contemporary spell-checking methods for turkic langauges.
we have shown that it works better and is faster. 

\bibliographystyle{unsrt}
\bibliography{turkspell}

\end{document}
% vim: set spell:
