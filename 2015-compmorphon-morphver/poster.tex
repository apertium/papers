\documentclass[fontscale=0.33,landscape,paperwidth=48in,paperheight=36in]{baposter}  % fontscale=0.285, dvipdfm

\usepackage{setspace}
\usepackage{multicol}
\usepackage{multirow}

\usepackage{tikz}
%\usepackage{pgfbaselayers}
%\pgfdeclarelayer{background}
%\pgfdeclarelayer{foreground}
%\pgfsetlayers{background,main,foreground}

%%% Color Definitions %%%%%%%%%%%%%%%%%%%%%%%%%%%%%%%%%%%%%%%%%%%%%%%%%%%%%%%%%

\selectcolormodel{cmyk}

\definecolor{bordercol}{RGB}{0,0,0}
%\definecolor{headercolone}{RGB}{6,130,22}
\definecolor{headercol}{cmyk}{1,0.64,0,0.6}
%\definecolor{headercolone}{cmyk}{1,0.64,0,0.6}
%\definecolor{headercoltwo}{cmyk}{0.44,0,0.15,0.17}
%\definecolor{headercolone}{cmyk}{0.3,0,0.15,0.17}
%\definecolor{headercoltwo}{cmyk}{0.3,0,0.15,0.17}
\definecolor{iumint}{cmyk}{0.40,0,0.23,0}
\definecolor{iumidnight}{cmyk}{0.71,0.30,0.13,0.41}
%\definecolor{boxcolor}{RGB}{176,208,223}
%\definecolor{boxcolor}{cmyk}{0.3,0.08,0.08,0.1}
\definecolor{boxcolor}{cmyk}{0.2,0.03,0.03,0.1}
\definecolor{headerfontcol}{RGB}{255,255,255}
\newcommand{\hilitetwo}[1]{{\addfontfeature{Color=99333399}#1}}
\newcommand{\hiliteone}[1]{{\addfontfeature{Color=06821699}#1}}
\newcommand{\hilitegrey}[1]{{\addfontfeature{Color=77777799}#1}}
\newcommand{\hilitelightgrey}[1]{{\addfontfeature{Color=CCCCCC99}#1}}

%%% Utility functions %%%%%%%%%%%%%%%%%%%%%%%%%%%%%%%%%%%%%%%%%%%%%%%%%%%%%%%%%%

%%% Save space in lists. Use this after the opening of the list %%%%%%%%%%%%%%%%
\newcommand{\compresslist}{
        \setlength{\itemsep}{1pt}
        \setlength{\parskip}{0pt}
        \setlength{\parsep}{0pt}
}

\usepackage{polyglossia}
\setdefaultlanguage[variant=australian]{english}

\usepackage{expex}

\usepackage{fontspec}
\defaultfontfeatures{PunctuationSpace=2,Scale=MatchLowercase,Mapping=tex-text}
\newfontfeature{IPA}{+mgrk}
%\setromanfont[IPA]{FreeSerif}
%\setromanfont[IPA,Scale=0.8]{CMU Serif}
%\setromanfont[IPA]{Liberation Serif}
\setromanfont[IPA]{Linux Libertine O}
%\setromanfont[IPA]{Nimbus Roman No9 L}
\setmonofont[IPA]{DejaVu Sans Mono}
\usepackage[small,bf]{caption}
\newfontfamily\qipa[IPA,Scale=MatchLowercase]{FreeSerif}
\newfontfamily\qgmk[IPA,Scale=0.8]{CMU Serif}
\newfontfamily\litamono[IPA,Scale=0.5]{DejaVu Sans Mono}
\newfontfamily\ortamono[IPA,Scale=0.625]{DejaVu Sans Mono}

\newcommand{\tilda}{{\qipa ∼}}

\newcommand{\tags}[1]{\hiliteone{#1}}
\newcommand{\tag}[1]{\hilitegrey{<}\hiliteone{#1}\hilitegrey{>}}
\newcommand{\archiphon}[1]{\hilitegrey{\{}\hiliteone{#1}\hilitegrey{\}}}

%\newfontfamily\htwo[IPA,Scale=1.2]{FreeSans}}
%\newfontfamily\htwofont[IPA,Scale=1]{CMU Sans Serif}
%\newfontfamily\htwofont[IPA,Scale=1,Color=333344FF]{CMU Sans Serif}
\newfontfamily\htwofont[IPA,Scale=1,Color=111111FF]{CMU Sans Serif}
%\newfontfamily\titlefont[IPA,Scale=0.52]{CMU Serif}
%\newfontfamily\titlefont[IPA,Scale=0.7]{CMU Serif}
%\newfontfamily\titlefont[IPA,Scale=0.7]{CMU Sans Serif Demi Condensed}
\newfontfamily\titlefont[
	%SmallCapsFont={Andika},
	%SmallCapsFont={Alegreya Sans SC},
	SmallCapsFont={Andada SC},
	%SmallCapsFont={Carrois Gothic SC},
	SmallCapsFeatures={Letters=SmallCaps},
	Scale=0.8,
]{Andada SC}
%\newfontfamily\titlefont[IPA,Scale=0.7]{Nimbus Sans L}
%\newfontfamily\titlefont[IPA,Scale=0.55]{DejaVu Serif}

%\newcommand{\htwo}[1]{::: {\htwofont #1} :::}%\hrule} %\hline\\}
%\newcommand{\htwo}[1]{{\htwofont \textbf{:::#1:::}}}%\hrule} %\hline\\}
\newcommand{\htwo}[1]{{\htwofont \textbf{\dotfill{}#1\dotfill{}}}}


\usepackage{graphicx}  % [dvipdfm]

%\definecolor{MyGray}{rgb}{0.96,0.97,0.98}
%\newenvironment{codebox}{%
%   \begin{lrbox}{\@tempboxa}\begin{minipage}{\columnwidth}}{\end{minipage}\end{lrbox}%
%   \colorbox{MyGray}{\usebox{\@tempboxa}}
%}
%\newcommand{\codeex}[1]{\begin{codebox}#1\end{codebox}}

%\definecolor{grey}{rgb}{0.96,0.97,0.98}
\definecolor{grey}{rgb}{0.91,0.91,0.91}
\newcommand{\codeex}[1]{
   \fbox{\colorbox{grey}{
         \begin{minipage}[t]{0.91\textwidth}{\litamono
				\begin{spacing}{0.4}
	            #1
				\vspace{-1em}\end{spacing}
         }\end{minipage}
      }
   }
}
\newcommand{\outputex}[1]{
   \fbox{\colorbox{iumidnight}{
         \begin{minipage}[t]{0.91\textwidth}{\ortamono
				\begin{spacing}{0.2}\vspace{-0.8em}
	            \hilitelightgrey{#1}
				\vspace{-2.5em}\end{spacing}
         }\end{minipage}
      }
   }
}



\newcommand{\blank}[1]{\underline{\hspace{#1}}}

\usepackage{natbib}

\usepackage[colorlinks=true,citecolor=black,linkcolor=black,urlcolor=black]{hyperref}

\usepackage{subfigure}
\usepackage{booktabs}

%%\bibpunct{(}{)}{;}{A}{,}{,}
%\bibdata{paper}

\newcommand{\citemultileft}[1]{(\citeauthor{#1}, \citeyear{#1}}
\newcommand{\citemultimid}[1]{\citeauthor{#1}, \citeyear{#1}}
\newcommand{\citemultiright}[1]{\citeauthor{#1}, \citeyear{#1})}
\newcommand{\citetwoyears}[2]{\citeauthor{#1} (\citeyear{#1} and \citeyear{#2})}

% for glosses
\newcommand{\eng}[1]{`{\em #1}'}
%dammit, sc doesn't seem to be working
\newcommand{\gmk}[1]{{\qgmk \textsc{#1}}}


\usepackage{enumitem}
\setlist{nolistsep,leftmargin=*}
\newenvironment{itemise}[1]{
        \begin{itemize}\setlength{\leftmargin}{-4em}\setlength{\itemsep}{-0.2em}
        \vspace{-0.5em}
        #1
}{
        \end{itemize}
        \vspace{-2pt}
}

%\newcommand{\h2}[1]{{\big



\begin{document}
	% To get it to be the same size consistently on all machines..
	\special{papersize=48in,36in}
	\setlength{\pdfpageheight}{\paperheight}
	\setlength{\pdfpagewidth}{\paperwidth}

	%%% Setting Background Image %%%
	%\background{}

	\begin{poster}{
		grid=false,
		%eyecatcher=false,
		borderColor=bordercol,
		headerColorOne=iumidnight,
		headerColorTwo=iumidnight,
		%headerColorOne=blue,
		headerFontColor=white,
		% Only simple background color used, no shading, so boxColorTwo isn't necessary
		boxColorOne=boxcolor,
		headershape=roundedright,
		headerborder=open,
		headerheight=0.1\textheight,
		%headershape=roundedright,
		%headershade=plain,
		%headerfont=\Large\textsf, %Sans Serif
		%headerfont=\Large\sf\bf,
		textborder=rectangle,
		%background=plain,
		%background=user,
		background=none,
		headerborder=open,
		boxshade=plain,
		textborder=roundedleft,
		columns=4,
	}{ %Eye Catcher, empty if option eyecatcher=false - unused
		\includegraphics[height=6.5em]{uitlogo}
		\hspace{2.5cm}\begin{minipage}[t]{7em}
			\vspace{-2cm}
			\noindent\includegraphics[height=4.2em]{DCU_logo_2col}\\
			\vspace{-0.5ex}\hspace{0.105cm}\includegraphics[height=2.15em]{DCU_2010_Name_Mark_289}
		\end{minipage}

	}{ %Title (centered top)
		{%\vspace{0pt}\hspace{1cm} \
			{\titlefont \scshape{Finite-State Morphologies \& Text Corpora\\\vspace{0.2em}\scalebox{0.8}{as Resources for Improving Morphological Descriptions}}}}
	}{ %Authors (centered top, below title)
		\vspace{-0.6em}
		\begin{center}
			\begin{minipage}[t]{9em}
				\begin{spacing}{0.4}
				{Francis M.\ Tyers}\\
				{\footnotesize UiT Norgga Árktalaš Universitehta \\\texttt{francis.tyers@uit.no}}
				\end{spacing}
			\end{minipage}
			\begin{minipage}[t]{9.5em}
				\begin{spacing}{0.4}
					{Tommi Pirinen\vphantom{y}}\\
					{\footnotesize Ollscoil Chathair Bhaile Átha Cliath\vphantom{y}\\\texttt{tommi.pirinen@computing.dcu.ie}}
				\end{spacing}
			\end{minipage}
			\begin{minipage}[t]{12em}
				\begin{spacing}{0.4}
					{Jonathan North Washington}\\
					{\footnotesize Indiana University\\\texttt{jonwashi@indiana.edu}}
				\end{spacing}
			\end{minipage}
		\end{center}
	}{ %More eye catchers, on the right side
		\includegraphics[height=6.25em]{iu_tab-crop}
		\hspace{2.5cm}\includegraphics[angle=90,height=6.8em]{apertium5b}%\hspace{-2em}
	}

	\headerbox{Morphological Descriptions}{name=morphdesc,column=0,row=0}{
		\htwo{What are they?}
		\begin{itemize}
			\item core part of grammatical descriptions
			\item found in reference grammars and textbooks
		\end{itemize}

		Descriptions of a language's
		\begin{itemize}
			\item \textbf{morphotactics} — the morphemes of a language and how they can be combined
			\item \textbf{morphophonology} — the alternations between the various phonological and orthographical forms of each morpheme in a language
		\end{itemize}
		
		\htwo{Reasons for being \hilitetwo{incomplete}}
		\begin{itemize}
			\item restrictions of the medium (e.g., number of pages)
			\item limitations of introspection, working with native speakers
			\item \textbf{lack of automatic testing}
		\end{itemize}
	}
	
	\headerbox{Morphological transducers}{name=morphtrans,below=morphdesc}{
		\htwo{Morphological transducers}
		\begin{itemize}
			\item Efficient (in speed \& size) models of a language's morphology
			\item Take a surface form, and produce valid lexical form(s)
			\item Take a lexical form, and produce valid surface form(s)\vspace{-0.1ex}\\
		%\end{itemize}
		\scalebox{0.98}{
			алдым \hilitetwo{↔} \texttt{ал\tag{v}\tag{tv}\tag{ifi}\tag{p1}\tag{sg}}, \texttt{алд\tag{n}\tag{px1sg}\tag{nom}}
		}
		\end{itemize}

		\htwo{Transducers for Turkic languages}
		\begin{itemize}
			\item Turkish (\hiliteone{Çöltekin, 2010 \& 2014}; \hilitetwo{Oflazer, 1994})
			\item Crimean Tatar (\hilitetwo{Altıntaş, 2001})
			\item Turkmen (\hilitetwo{Tantuğ et al., 2006})
			\item Kazakh (\hilitetwo{Бекманова \& Махимов, 2013})
			%\vspace{1pt}\hfill{} \hiliteone{GPL (=free and open)}!
			\item our Kyrgyz, Kazakh, Tatar, Kumyk:\ all \hiliteone{GPL (=free and open)}!
		\end{itemize}

		\htwo{Framework:\ HFST}
		\begin{itemize}
			\item Reimplements Xerox FST formalisms ({\tt lexc} \& {\tt twol})
			\item Also provides a wrapper around popular free/open-source FST toolkits: SFST, OpenFST, and Foma
		\end{itemize}
		\htwo{Approach}
		\begin{itemize}
			\item morphotactics implemented in \texttt{lexc}
			\item morphophonology implemented in \texttt{twol}
			\item compiled separately; compose-intersected to single transducer\vspace{-0.1ex}\\
			алдым\hfill{}\hilitetwo{↔}\hfill{}\texttt{ал\hilitegrey{>}\archiphon{D}\archiphon{I}\hilitegrey{>}м}\hfill{}\hilitetwo{↔}\hfill{}\texttt{ал\tag{v}\tag{tv}\tag{ifi}\tag{p1}\tag{sg}}\vspace{-0.1ex}\\
			алдым\hfill{}\hilitetwo{↔}\hfill{}\texttt{алд\hilitegrey{>}\archiphon{I}м}\hfill{}\hilitetwo{↔}\hfill{}\texttt{алд\tag{n}\tag{px1sg}\tag{nom}}
		\end{itemize}

	}

	\headerbox{Categorisation}{name=adjectives,column=3}{
	For example, many grammars of Turkic languages (e.g., Somfai Kara, 2002) overgeneralise adjective morphology, stating that all adjectives may act like nouns (and take nominal morphology), or that all adjectives may be used as adverbs with no morphological changes.  We have found that there is a range of “adjective classes” in most Turkic languages, including some which may not be substantivised and some which may not be used adverbially.  This phenomenon is of particular interest because native-speaker intuitions on the morphological limitations of a given adjective can often be more restrictive than the range of possible uses found in a large text corpus.
		\htwo{Adjectives}
			\begin{itemize}
				\item morphologically distinct adjective classes
				\item most sources claim:\ adjectives can be used substantively and adverbially
				\item Other Turkic transducers:\ \texttt{0}-derivation (\hilitetwo{overgenerates})
				\item but not all adjectives have all of the following:\\
				comparative forms, substantive readings, adverbial readings
				\item Our approach:\ categorisation 
				\item if properly categorised, only correct forms are analysed and generated
			\end{itemize}

			\scalebox{0.7}{
				\begin{tabular}{lllll}
					\toprule
					\textbf{Type} & \textbf{Gloss} & {\small \texttt{\tag{adj}(\tag{comp})}} & {\small \texttt{\tag{adj}(\tag{comp})\tag{subst}}} & {\small \texttt{\tag{adj}(\tag{comp})\tag{advl}}} \\
					\midrule
					\textbf{A1} & \eng{good} & {\qipa яхшы (яхшырак)} & {\qipa яхшы (яхшырак)} & {\qipa яхшы (яхшырак)} \\
					\textbf{A2} & \eng{old} & {\qipa иске (искерәк)} & {\qipa иске (искерәк)} & {\qipa — (—)} \\
					\textbf{A3} & \eng{dead} & {\qipa үле (—)} & {\qipa үле (—)} & {\qipa — (—)} \\
					\textbf{A4} & \eng{basic} & {\qipa төп (—)} & {\qipa — (—)} & {\qipa — (—)} \\
					\bottomrule
				\end{tabular}
			}

		\htwo{Adverbs}
			\begin{itemize}
				\item Certain adverbs have special attributive and ablative forms
				\item Mostly time adverbs
				%\item Mostly time adverbs, e.g.\\
				%{\qipa бүгүн} \eng{today}, {\qipa быйыл} \eng{this year}, {\qipa кечээ} \eng{yesterday}, {\qipa жана} \eng{just now}
				%\item Attributive forms:\\
				%{\qipa бүгүнкү} \eng{today's}, {\qipa быйылкы} \eng{this year's}, {\qipa кечээги} \eng{yesterday's}, {\qipa жанагы} \eng{just now}
				%\item Ablative forms:\\
				%{\qipa бүгүнтөн} \eng{from today}, {\qipa быйылтан} \eng{from this year}, {\qipa кечээтен} \eng{from yesterday}, {\qipa жанатан} \eng{from just now}
				\item Some also have noun readings:\ regular ablative, other cases:
				%{\qipa бүгүндөн} \eng{from today}, {\qipa быйылдан} \eng{from this year}
			\end{itemize}
			\scalebox{0.83}{
			\begin{tabular}{lllll}
				\toprule
				word & {\qipa бүгүн} & {\qipa быйыл} & {\qipa кечээ} &  {\qipa жана}\\
				gloss & \eng{today} & \eng{this year} & \eng{yesterday} & \eng{just now} \\
				\midrule
				\tag{attr} form & {\qipa бүгүнкү} & {\qipa быйылкы} & {\qipa кечээги} &  {\qipa жанагы}\\
				\tag{adv}\tag{abl} form & {\qipa бүгүнтөн} & {\qipa быйылтан} & {\qipa кечээтен} &  {\qipa жанатан}\\
				\tag{n}\tag{abl} form & {\qipa бүгүндөн} & {\qipa быйылдан} & — & — \\
				\bottomrule
			\end{tabular}
			}\vspace{0.2em}
				\codeex{
					\begin{spacing}{0.8}
					\texttt{LEXICON ADV-WITH-KI-ABL\\
\\
ADV ;\\
ADV-KI ;\\
ADV-ABL ;
					}
					\end{spacing}
					\vspace{0em}
				}
	}

	\headerbox{Ongoing and future work}{name=nedostatki,below=adjectives,column=3}{
		\begin{itemize}
			\item Disambiguation, more stems, clean up transducers
			\item Machine translation between these languages
			\item Bring other Kypchak transducers to comparable performance:\\
			Qaraqalpaq, Bashqort, Nogay, Crimean Tatar
			\item Other Turkic lgs:\ Uzbek, Uyghur, Chuvash, Yakut, Tuvan, etc.
		\end{itemize}
	}

	\headerbox{Further information}{name=getting,column=3,below=nedostatki}{
		\begin{itemize}
			\item Part of \textbf{Apertium Turkic} project:\\
			\url{http://wiki.apertium.org/wiki/Apertium\_Turkic}
			\item Transducers available \textbf{live} at \href{http://turkic.apertium.org/}{\texttt{turkic.apertium.org}}
			\item \textbf{Source code} available from Apertium's svn repo
			%info at {\small \url{http://wiki.apertium.org/wiki/apertium-kir}}
				\item Turkic RBMT \textbf{mailing list} (>25 subscribers):\\
			\texttt{apertium-turkic@lists.sourceforge.net}\\
			\vspace{-0.5ex}Feel free to post in any language!\\\vspace{-2.5ex}
			\item See our papers in LREC proceedings\\
			(2012:\ Kyrgyz, 2014:\ Kazakh, Tatar, Kumyk)
			\item And feel free to contact the authors any time!
		\end{itemize}
	}

%	\headerbox{References}{name=references,column=2,below=nedostatki}{
%		%\begin{thebibliography}{1}\itemsep=-0.01em
%		\setlength{\baselineskip}{0.4em}
%	}



		\headerbox{Example:\ Tuvan velar elision}{name=tuvanvelars,column=1,span=2}{
			\begin{multicols}{2}
				\htwo{Previous descriptions}

					\begin{itemize}
						\item Anderson and Harrison (1999, pp.\ 9, 22-23)
					\end{itemize}
					\includegraphics[width=0.49\textwidth]{img/andersonharrison9}\\
					\includegraphics[width=0.49\textwidth]{img/andersonharrison22-23}
					\begin{itemize}
						\item Исхаков, Ф.Г.; Пальмбах (1961, pp.\ 117-118)
					\end{itemize}
					\includegraphics[width=0.49\textwidth]{img/isxakovpalmbax117}

				\htwo{More complete characterisation}
				\begin{itemize}
					\item /k/ ‹к› elides at the end of >1-σ stems intervocalically\\
						инэк+I → инээ, өк+I → өгү
					\item /g/ ‹г› elides at the end of any stem intervocalically\\
						өг+I → өө 
					\item /ŋ/ ‹ң› elides at the end of \textbf{some} stems intervocalically.\\
						\textbf{соң+I → соо}, чаң+I → чаңы, түң+I → түңү
				\end{itemize}

				\htwo{Correct forms in the grammars}

				But not part of the morphophonological description.\\
				(A. \& H., p.\ 35)\hfill{}\includegraphics[width=0.2\textwidth]{img/andersonharrison35}\\
				(И.\ \& П., p.\ 447)\hfill{}\includegraphics[width=0.35\textwidth]{img/isxakovpalmbax447}


			\columnbreak{}
				\htwo{Uanalysed forms before implementation}
					\begin{itemize}
						\item \hilitetwo{инээ}, \hiliteone{өгү}, \hilitetwo{өө}, \hilitetwo{соо}, \hiliteone{чаңы}
					\end{itemize}\vspace{-0.3ex}
					\begin{minipage}[t]{0.5\textwidth}
					\outputex{
						\begin{multicols}{4}
							1590 соонда\vphantom{ү}\\
							610 те\vphantom{ү}\\
							153 өөнүң\vphantom{ү}\\
							48 соондан\vphantom{ү}\\
							46 өөм\vphantom{ү}\\
							20 соон\vphantom{ү}\\
							18 өө\vphantom{ү}\\
							17 өөнге\vphantom{ү}\\
							9 соондагы\vphantom{ү}\\
							8 өөн\vphantom{ү}\columnbreak\\
							6 инээниң\vphantom{ү}\\
							6 соонче\vphantom{ү}\\
							6 соонга\vphantom{ү}\\
							3 инээм\vphantom{ү}\\
							3 өөнден\vphantom{ү}\\
							3 өөнде\vphantom{ү}\\
							3 өөмнүң\vphantom{ү}\\
							2 инээн\vphantom{ү}\\
							2 инээ\vphantom{ү}\\
							2 өөнче\vphantom{ү}\columnbreak\\
							2 өөңерге\vphantom{ү}\\
							2 өөңер\vphantom{ү}\\
							2 өөмнү\vphantom{ү}\\
							2 өөвүстен\vphantom{ү}\\
							2 өөвүске\vphantom{ү}\\
							2 өөвүс\vphantom{ү}\\
							1 инээңни\vphantom{ү}\\
							1 инээмниң\vphantom{ү}\\
							1 Инээм\vphantom{ү}\\
							1 инээвистиң\vphantom{ү}\\
							1 өөңнүң\vphantom{ү}\\
							1 өөң\vphantom{ү}\\
							1 өөвүсче\vphantom{ү}\\
							1 өөвүстүң\vphantom{ү}\\
							1 өөвүстү\vphantom{ү}\\
							1 өөвүсте
						\end{multicols}}
					\end{minipage}

				\htwo{Initial implementation}
					\begin{minipage}[t]{0.5\textwidth}
					\codeex{"Intervocalic voiced velar deletion"\\
г:0 <=> :Vow/:0* \_ [ \%>: :Vow ]/:0* ;\\
\-~\\
"Intervocalic voiceless velar deletion"\\
к:0 <=> :Vow/:0* \_ [ \%>: :Vow ]/:0* ;\\
\-~~~~except\\
\-~~~~~~~~[ .\#.\ | \%- ] [ ( :Cns* ) ( :Vow* ) :Vow ]/:0 \_ [ \%>:\ :Vow ]/:0* ; %[ :0 - \%>:\ ]* ;
					}
					\end{minipage}

				\htwo{Unanalysed forms after initial implementation}
					%(has forms of соң in it)
					\begin{itemize}
						\item \hiliteone{инээ}, \hiliteone{өгү}, \hiliteone{өө}, \hilitetwo{соо}, \hiliteone{чаңы}
					\end{itemize}\vspace{-0.3ex}
					\begin{minipage}[t]{0.5\textwidth}
					\outputex{
						\begin{multicols}{3}
							1590 соонда\vphantom{ң}\\
							610 те\vphantom{ң}\\
							48 соондан\vphantom{ң}\\
							20 соон\vphantom{ң}\\
							6 соонче\vphantom{ң}\\
							6 соонга
						\end{multicols}
					}
					\end{minipage}

				\htwo{Second attempt at implementation}

					\begin{minipage}[t]{0.5\textwidth}
					\codeex{\hilitegrey{"Intervocalic voiced velar deletion"}\\
Cx\hilitegrey{:0 <=> :Vow/:0* \_ [ \%>: :Vow ]/:0* ;}\\
\-~~~~~where Cx in ( г ң ) ;\\
\-~\\
\hilitegrey{"Intervocalic voiceless velar deletion"\\
к:0 <=> :Vow/:0* \_ [ \%>:\ :Vow ]/:0* ;\\
\-~~~~except\\
\-~~~~~~~~[ .\#.\ | \%- ] [ ( :Cns* ) ( :Vow* ) :Vow ]/:0 \_ [ \%>:\ :Vow ]/:0* ;} %[ :0 - \%>:\ ]* ;
					}
					\end{minipage}

				\htwo{Unanalysed forms after second implementation}
					\begin{itemize}
						\item \hiliteone{инээ}, \hiliteone{өгү}, \hiliteone{өө}, \hiliteone{соо}, \hilitetwo{чаңы}
					\end{itemize}\vspace{-0.3ex}
					\begin{minipage}[t]{0.5\textwidth}
					\outputex{
						\begin{multicols}{3}
							610 те\vphantom{ү}\\
							203 бажыңынга\vphantom{ү}\\
							164 бажыңының\vphantom{ү}\\
							102 бажыңы\vphantom{ү}\\
							86 бажыңын\vphantom{ү}\\
							56 түңү\vphantom{ү}\\
							47 чаңы\vphantom{ү}\\
							44 чаңын\vphantom{ү}\\
							13 түңүн\vphantom{ү}\\
							12 чаңының\vphantom{ү}\\
							5 түңүнден\vphantom{ү}\\
							4 чаңындан\vphantom{ү}\\
							4 чаңында\vphantom{ү}\\
							3 түңүнүң\vphantom{ү}\\
							2 түңүнге
						\end{multicols}}
					\end{minipage}

					%(has forms of бажың, түң, чаң in it)

				\htwo{Final implementation}
			
					\begin{minipage}[t]{0.5\textwidth}
					\codeex{\hilitegrey{"Intervocalic voiced velar deletion"\\
Cx:0 <=> :Vow/:0* \_ [ \%>:\ :Vow ]/:0* ;}\\
\-~~~~~except\\
\-~~~~~~~~~:Vow \_ [ \%\{☭\%\}: :Vow ]/:0* ; \\
\-~~~~~\hilitegrey{where Cx in ( г ң ) ;\\
\-~\\
"Intervocalic voiceless velar deletion"\\
к:0 <=> :Vow/:0* \_ [ \%>:\ :Vow ]/:0* ;\\
\-~~~~except\\
\-~~~~~~~~[ .\#.\ | \%- ] [ ( :Cns* ) ( :Vow* ) :Vow ]/:0 \_ [ \%>:\ :Vow ]/:0* ;%\\
%\-~~~~~~~~:Vow \_ [ \%\{☭\%\}:\ :Vow ]/:0* ;
					}}
					\end{minipage}

				\htwo{Unanalysed forms after final implementation}
					\begin{itemize}
						\item \hiliteone{инээ}, \hiliteone{өгү}, \hiliteone{өө}, \hiliteone{соо}, \hiliteone{чаңы}
					\end{itemize}\vspace{-0.3ex}
					\begin{minipage}[t]{0.5\textwidth}
					\outputex{
						\begin{multicols}{3}
							610 те\\
						\end{multicols}}
					\end{minipage}
			\end{multicols}

			%\htwo{Examples of correct forms}


		}

		\headerbox{Example: Tuvan vowel harmony after consonant clusters in loanwords}{name=otherexamples,column=1,span=2,below=tuvanvelars}{
			\begin{minipage}{0.5\textwidth}
				\codeex{"\{I\} harmony"\\
\%\{I\%\}:Vy <=> :Vx [ :Cns* :RealCns ]/[ :0 | \%- ]* \_ ;\\
\-~~~~~~~~~where Vx in ( ү и е э ө а о ы у я ё ю )\\
\-~~~~~~~~~~~~~~ Vy in ( ү и и и ү ы у ы у ы у у )\\
\-~~~~~~~~~matched ;}
			\end{minipage}

			\begin{minipage}{0.5\textwidth}
				\codeex{"\{I\} harmony"\\
\%\{I\%\}:Vy <=> :Vx [ :Cns* :RealCns ]/[ :0 | \%- ]* \_ ;\\
\-~~~~~~~~~except\\
\-~~~~~~~~~~~~~[ :BackVow :Cns* :Cns л: ь: :Cns* :RealCns ]/:0* \_ ;\\
\-~~~~~~~~~~~~~[ :BackVow :Cns* :Cns л: ь:0 ]/:[ :0 - ь: ]* \_ ;\\
\-~~~~~~~~~where Vx in ( ү и е э ө а о ы у я ё ю )\\
\-~~~~~~~~~~~~~~ Vy in ( ү и и и ү ы у ы у ы у у )\\
\-~~~~~~~~~matched ;\\
\\
"\{I\} always front when intervening Cль"\\
\%\{I\%\}:и <=> [ :BackVow :Cns* :Cns л: ь: :Cns* :RealCns ]/:0* \_ ;\\
\-~~~~~~~~~~~~[ :BackVow :Cns* :Cns л: ь:0 ]/:[ :0 - ь: ]* \_ ;}
			\end{minipage}


		}

		\headerbox{Examples:\ ideas for other examples}{name=otherexamples,column=0,span=1,below=tuvanvelars}{ %{name=otherexamples,column=1,span=2,below=tuvanvelars}{
			\begin{itemize}
				\item Kyrgyz: conditions where л desonorises to д
				\item Tuvan: рубль - рубльдиң *рубльдуң
				\item clitics can go everywhere/anywhere
				\item underformalised - “particles☹” can go anywhere
				\item all adjs act like nouns (Somfai-Kara, p. 28)
				\item all adjectives can be used as adverbs without any morphological changes (Somfai-Kara, p. 29)
				\item Stuff found doing treebank
			\end{itemize}

		}

		\headerbox{Morphophonology}{name=morphophon,column=0,below=otherexamples}{

			\htwo{Desonorisation (kaz \& kir)}
				\begin{itemize}
					\item {\texttt \{N\}} desonorises to д after a consonant\\
						алма-\hiliteone{\{N\}}\{I\} → алма\hiliteone{н}ы \eng{apple--\gmk{ACC}} \\
						сыр-\hiliteone{\{N\}}\{I\} → сыр\hiliteone{д}ы \eng{secret--\gmk{ACC}}
					\item {\texttt \{L\}} desonorises to д after cons.\ of sonority ≤ /l/ \\
						сыр-\hiliteone{\{L\}}\{A\}р → сыр\hiliteone{л}ар \eng{secret--\gmk{PL}} \\
						кыз-\hiliteone{\{L\}}\{A\}р → кыз\hiliteone{д}ар \eng{girl--\gmk{PL}} \\
				\end{itemize}

				\vspace{-0.5ex}
				\codeex{
					\texttt{"L Desonorisation"\\
						\%\{L\%\}:д <=> :VoicedLowSonCns \%>:\ \blank{1em} ;}\\
			
					\texttt{"N Desonorisation"\\
						\%\{N\%\}:д <=> :VoicedCns \%>:\ \blank{1em} ;}
				}
			\vspace{0.1em}

			\htwo{Epenthesis}
				\begin{itemize}
					\item Turn {\texttt \{y\}} into a harmonised high vowel when a vowel doesn't follow the following consonant:\\
					мур\archiphon{у}н → мурун \eng{nose}\\
					мур\archiphon{у}н+\archiphon{I}м → мурдум \eng{my nose}\\
				\end{itemize}
					\vspace{-0.5ex}
\codeex{
			\begin{spacing}{0.9}
{\small \texttt{\%\{y\%\}:Vy <=> [ :LastVowel :Cns* :Cns ]/[:0] \blank{1em}\\
	\vspace{-0.5ex}\hspace{1pt}\hfill{} [ :Cns [ .\#.\ | :Cns ] ]/[ :0 | \%>:]\ ;\\
  \vspace{-0.5ex}\hspace{1ex}where  Vy  in  (  и  ү  и  и  ү  ы  ы  у  у  ы  у  у )\\
  \vspace{-0.5ex}LastVowel  in  (  и  ү  е  э  ө  я  а  ё  о  ы  ю  у )\\
  \vspace{-0.5ex}\hspace{12ex}         matched ;}}\vspace{-1.5em}
  			\end{spacing}
}
\vspace{0.1em}
		}

	\end{poster}
\end{document}
