%
% File eamt15.tex, an adaptation of eamt14.tex (which was a copy of
% eamt12.tex)
%
% Contact: eamt2015@dlsi.ua.es

%%% To ease future customizations, various replaceables have been paramaterized
%%% as listed in the newcommands section

\documentclass[11pt]{article}
\usepackage{eamt15}
\usepackage{times}
\usepackage{latexsym}
\setlength\titlebox{6.5cm}    % Expanding the titlebox
%%% YOUR PACKAGES BELOW THIS LINE %%%

\usepackage{multirow}
\usepackage{natbib}
\usepackage{color}
\usepackage{url}

\newcommand{\confname}{EAMT 2015}
\newcommand{\website}{http://www.eamt2015.org/}
\newcommand{\contactname}{the conference chairs (Felipe
  S\'anchez-Mart\'inez, Gema Ram\'irez-S\'anchez and Fred Hollowood)}
\newcommand{\contactemail}{eamt2015@dlsi.ua.es}
\newcommand{\conffilename}{eamt15}
\newcommand{\downloadsite}{http://www.eamt2015.org/}
\newcommand{\paperlength}{$8$ (eight)}
\newcommand{\shortpaperlength}{$4$ (four)}

\newcommand{\comment}[1]{\marginpar{\scriptsize\sf \textcolor{blue}{#1}}}


\title{A comparison of approaches to Kazakh to Kyrgyz machine translation}

\author{
  Jonathan North Washington\\
  Departments of Linguistics\\
  and Central Eurasian Studies\\
  Indiana University\\
  Bloomington, Indiana 47405 USA\\
  \texttt{jonwashi@indiana.edu}  \And
  \textbf{Francis M. Tyers}\\
  HSL-fakultetet\\ 
  UiT Norgga \'{a}rktala\v{s} universitehta \\
  9017 Romsa, Norway \\
  {\tt francis.tyers@uit.no}
}

\date{}

\begin{document}
\maketitle

\begin{abstract}
...
\end{abstract}

\section{Introduction}

This paper presents a comparison of approaches to machine translation
from Kazakh to Kyrgyz.
The paper will be laid out as follows: Section~\ref{sec:prev}\ gives a short review of some previous work in the area of Turkic--Turkic language machine translation; Section~\ref{sec:lang} introduces Kazakh and Kyrgyz and compares their grammar;
Section~\ref{sec:sys}\ describes the systems we built and the tools used to construct them;
Section~\ref{sec:eval}\ gives an evaluation of the systems; and
finally Section~\ref{sec:conc}\ describes our aims for future work and some
concluding remarks.


\section{Previous work}
\label{sec:prev}

Within the Apertium project, work on several MT systems between Turkic languages has been started (Kazakh--Tatar, Tatar--Bashkir, Turkish--Kyrgyz, and Azeri--Turkish).  There are two published works describing these systems, the first is \cite{tyerswashingtonsalimzyanbattalov12} which describes a Tatar--Bashkir machine translation system; due to the closeness of these languages, it proved to provide high accuracy in its translations, but being a prototype system by design, had relatively low coverage. The second is \cite{tyerswashingtonsalimzyan13} which describes a high-coverage Kazakh--Tatar system.

Besides these systems, several previous works on making machine translation systems between Turkic languages 
exist, although to our knowledge none are publicly available except for the Turkish--Azerbaijani pair available through Google Translate.\footnote{\url{http://translate.google.com}}
Some MT systems have been reported that translate between Turkish and other Turkic languages, 
including Turkish--Crimean Tatar \citep{altintas01},
Turkish--Azerbaijani \citep{hamzaoglu93}, Turkish--Tatar \citep{suleymanov08}, and
Turkish--Turkmen \citep{tantug07}, though none of these have been released to a public audience. 

\section{Kazakh and Kyrgyz}

\section{Approaches}

\subsection{Rule-based machine translation}

\subsection{Corpus-based machine translation}

\subsection{Corpus}

\subsubsection{Phrase-based}

\subsubsection{Character-based }

\section{Evaluation}

\section{Future work}

\section{Conclusions}

\section*{Acknowledgements}

\emph{Anonymised for review}

\bibliographystyle{apalike}
\bibliography{2015-eamt-kazkir}

\end{document}
