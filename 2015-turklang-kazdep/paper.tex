\documentclass[a4paper,11pt, onecolumn]{article}

\usepackage{turklang}

\usepackage{polyglossia}
\setdefaultlanguage[variant=australian]{english}

\usepackage{fontspec}
\usepackage{xunicode}

\usepackage{multirow}
\usepackage[small,bf]{caption} % CHECK IF THIS IS OK
\usepackage[colorlinks=true,citecolor=black,linkcolor=black,urlcolor=blue]{hyperref}

\usepackage{natbib}
\bibdata{paper}

%\usepackage{times}
\setmainfont{Times New Roman}

\title{Towards a free/open-source universal-dependency treebank for Kazakh}
\name{Francis M. Tyers$^a$ and Jonathan Washington$^b$}
\address{$^a$ HSL-fakultehta, UiT Norgga árktalaš universitehta, N-9015 Tromsø, Norway\\
	$^b$ Departments of Linguistics and Central Eurasian Studies, Indiana University, Bloomington, IN 47405, USA\footnote{Corresponding author.  E-mail address: jonwashi@indiana.edu}}

\keywords{Kazakh; treebank; dependency grammar; universal dependency}

%250-300-word ``extended abstract in English''
\abstract{
This article describes the first steps towards a free/open-source dependency treebank
for Kazakh based on universal dependency (UD) annotation standards. The treebank contains
302 sentences and is based on texts from a range of open-source and public domain sources.
This ensures its free availability and extensibility. Texts in the treebank are first morphologically 
analysed and disambiguated and then annotated manually for dependency structure. 
}

\begin{document}

\maketitleabstract{}


\section{Introduction}

\citet{Lynn12}, \citet{Atalay03}, \citet{Oflazer03}, \citet{DeMarneffe14}

\section{Background}

\subsection{Kazakh}

% case and function
% "zero" derivation

\subsection{Treebanks} % Related work

% what are treebanks used for ?

\section{Methodology}

\subsection{Corpus}

\begin{table}
  \centering
\begin{small}
  \begin{tabular}{|l|l|r|r|r|}
    \hline
    \textbf{Document} & \textbf{Description} & \textbf{Sentences} & \textbf{Tokens} & \textbf{Avg. length}\\
    \hline
    UN Declaration on Human Rights & Legal text on human rights & 25 & 409 & 16.3 \\
    Phrasebook                     & Phrases from Wikitravel   & 37 & 205 & 5.5 \\
    Жиырма Бесінші Сөз             & Philosophical text        & 34 & 525 & 15.4 \\ 
    Өлген қазан                    & Folk tale                  & 8 & 140 & 17.5 \\
    Ер Төстік                      & Folk tale                  & 23 & 203 & 8.8 \\
    Азамат қайда?                  & Story for language learners & 48 & 434 & 9.0 \\
    Футболдан әлем чемпионаты 2014 & Wikipedia article (2014 World Cup) & 14 & 246 & 17.5 \\
    Иран & Wikipedia article (Iran)                                & 111 & 1562 & 14.0 \\
    Радиан & Wikipedia article (Radian)                            & 2 & 17 & 8.5 \\
    \hline
    \multicolumn{2}{c|}{~} & 302 & 3741 & 12.3 \\\cline{3-5}
  \end{tabular}
\end{small}
  \caption{Composition of the corpus. The corpus covers a range of genres and text types from 
    free and public-domain sources.}
\end{table}


\subsection{Preprocessing}

Preprocessing the corpus consists of running the text through the Kazakh morphological
analyser \citep{Washington14}, which also performs tokenisation of multiword units based 
the longest match left-to-right. Tokenisation for Kazakh is a non-trivial task, and so
we do not simply take space as a delimiter. The morphological analyser returns all 
the possible morphological analyses for each word based on a lexicon of around 20,000 lexemes.
After tokenisation and morphological analysis, the text is processed with a constraint-grammar 
based disambiguator for Kazakh consisting of 113 rules which remove inappropriate 
analyses in context. This reduces the average number of analyses per word from around 3.4
to around 1.7.

% tokenisation
% http://universaldependencies.github.io/docs/u/overview/tokenization.html

\subsubsection{Tokens and words}

% simple tokens
% multiword units
% copula
% clitics
% productive suffixes -lı (cite wafflelazor)

\section{Annotation guidelines}

% general procedure

% - disambiguate
% - annotate
% - check the tree structure
% - fix errors
% - check for label consistency
% - fix errors

\begin{table}
  \centering
  \begin{tabular}{|l|l|}
    \hline
    \textbf{Label} & \textbf{Description} \\

    \hline
  \end{tabular}
  \caption{Universal dependency label set}
\end{table}

\subsection{Copula}

\subsection{Coordination}

\subsection{Complex nominals}

% compounds + izafet

\subsection{Non-finite clauses}

\section{Evaluation}

\section{Future work}

%stabilise guidelines

%crosslingual parsing: Tatar, Kumyk, Kyrgyz, Tuvan

%language-specific dependency relations --- (acl:relcl, ...)

\section{Conclusions}

\section*{Acknowledgements}

\bibliographystyle{apa}
\bibliography{paper}

\end{document}
