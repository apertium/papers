\documentclass[a4paper,11pt, onecolumn,twoside]{article}

\usepackage{turklang}


%%%%% Language and font stuff %%%%%

\usepackage{alltt}
\usepackage{polyglossia}
\setdefaultlanguage[variant=australian]{english}

\usepackage{fontspec}
\usepackage{xunicode}

%\usepackage{times}
\setmainfont{Times New Roman}
\newfontfamily\ll[]{Linux Libertine O}
\newfontfamily\udfont[Scale=1.3,Letters=SmallCaps]{Linux Libertine O}
\newfontfamily\fsfont[Scale=MatchLowercase]{FreeSerif}



%%%%% General formatting stuff %%%%%

\usepackage[normalem]{ulem}
\usepackage{multirow}
\usepackage{multicol}
\usepackage[small,bf]{caption} % CHECK IF THIS IS OK
\usepackage[colorlinks=true,citecolor=black,linkcolor=black,urlcolor=blue]{hyperref}

\usepackage{subcaption}
\usepackage{booktabs}

%\usepackage{supertabular}
%\newcount\n
%\n=0
%\def\tablebody{}
%\makeatletter
%\loop\ifnum\n<100
        %\advance\n by1
        %\protected@edef\tablebody{\tablebody
                %\textbf{\number\n.}& shortText
                %\tabularnewline
        %}
%\repeat

%\makeatletter
%\let\mcnewpage=\newpage
%\newcommand{\TrickSupertabularIntoMulticols}{%
  %\renewcommand\newpage{%
    %\if@firstcolumn
      %\hrule width\linewidth height0pt
      %\columnbreak
    %\else
      %\mcnewpage
    %\fi
  %}%
%}
%\makeatother

%\usepackage{natbib}
%\bibdata{paper}
%\usepackage[backend=bibtex]{biblatex}
\usepackage[backend=biber,citestyle=apa,style=authoryear,maxbibnames=99,maxcitenames=2]{biblatex}
\bibliography{paper}
\renewcommand*{\nameyeardelim}{\addcomma\space}


%%%%% Stuff for this paper %%%%%
\usepackage{tikz-dependency}
\newcommand{\gmk}[1]{{\rm {\sc \texttt{#1}}}}
\newcommand{\kazakh}[1]{{\em #1}}
\newcommand{\gloss}[1]{`#1'}
\newcommand{\tgloss}[1]{{\em #1}}
\newcommand{\tag}[1]{\texttt{‹#1›}}
%\newcommand{\udtag}[1]{\texttt{@#1}}
\newcommand{\udtag}[1]{{\ll \textsc{#1}}}
\newcommand{\udlabel}[1]{{\udfont #1}}
\newcommand{\tilda}{{\fsfont ∼}}
%\newcommand{\corrauthor}[1]{\footnotemark\footnotetext{hargle}}


%%%%% Paper metadata %%%%%

\title{Towards a free/open-source universal-dependency treebank for Kazakh}
\name{Francis M. Tyers$^a$ and Jonathan Washington$^b$}
\shortauthor{Tyers \& Washington}
\address{$^a$ HSL-fakultehta, UiT Norgga árktalaš universitehta, N-9015 Tromsø, Norway\\
	$^b$ Departments of Linguistics and Central Eurasian Studies, Indiana University, Bloomington, IN 47405, USA\footnote{Corresponding author.  E-mail address: jonwashi@indiana.edu}}

\keywords{Kazakh; treebank; dependency grammar; universal dependency}

%250-300-word ``extended abstract in English''
\abstract{
This article describes the first steps towards a free/open-source dependency treebank for Kazakh based on
universal dependency (UD) annotation standards. The treebank contains 302 sentences and is based on
texts from a range of open-source and public domain sources. This ensures
its free availability and extensibility. Texts in the treebank are first morphologically analysed
and disambiguated and then annotated manually for dependency structure. In the article we present
some issues in dependency syntax for Kazakh and how these are analysed in the universal-dependency
framework. Preliminary results for statistical dependency parsing of Kazakh are reported, along with
some directions for future research.
}


%%%%% End preamble %%%%%

\begin{document}

%\thanks{hargle}
\maketitleabstract{}
\thispagestyle{firststyle}

\section{Introduction}

This article describes work towards the development of a dependency treebank for Kazakh, a Turkic language spoken in Central Asia and Europe.
Despite its status as a \emph{core}\footnote{FIXME} Turkic %FIXME
language, little computational-linguistic research has been published on syntactic
parsing for Kazakh.
A valuable resource in the study of syntactic parsing is
a treebank---a corpus of parsed text containing gold-standard syntactic annotation.

Freely available treebanks exist for many languages, such as large languages like Finnish \parencite{haverinen2013tdt,voutilainen11} 
and Polish \parencite{wolinski11} and smaller languages like Irish \parencite{Lynn12}. To our knowledge, however, a treebank exists for only one other Turkic language, Turkish \parencite{Oflazer03}, which is unfortunately not freely available.

In building our treebank we take advantage of existing work done on tokenisation,
morphological analysis and part-of-speech tagging for Kazakh. We also take a pragmatic
and iterative view of development of the treebank, in line with recent work 
on cross-linguistic parsing with universal dependencies \parencite{DeMarneffe14}.

The remainder of the paper is organised as follows. Section~\ref{sec:back} gives some
background linguistic information on Kazakh, and outlines some special challenges in
parsing Kazakh. In Section~\ref{sec:method} we describe the corpus that we annotated and the methodology
used in annotating it. Section~\ref{sec:annotation} gives a sketch of some decisions
we have made with respect to annotation guidelines, referring back to the discussion in
Section~\ref{sec:back}. For reasons of space, these guidelines are not complete, but
present a subset of guidelines which are of particular interest. A small experiment
in statistical dependency parsing using the corpus is presented in Section~\ref{sec:eval},
and in Sections~\ref{sec:future} and \ref{sec:conclusions} we give perspectives
for future work and some concluding remarks.

%\textcite{Atalay03}

\section{Background}\label{sec:back}

\subsection{Kazakh}\label{sec:kazakh}

Kazakh (қазақ тілі), a Turkic language of Central Asia and Europe, is spoken by around 13 million people in Kazakhstan, China, Mongolia, and adjacent areas \parencite{ethnologue}.  While works like \textcite{Balaqayev54} provide decent syntactic overviews of the language, there is little to no work on the syntax of Kazakh within modern theoretical syntactic frameworks.  The authors are familiar with such work on related languages, especially Turkish (e.g., \cite{Kornfilt97} and \cite{GokselKerslake05}); while not directly consulted for this work, these works have contributed to our understanding of Kazakh syntax.

As an agglutinative language with rich morphology and agreement phenomena, Kazakh presents some interesting challenges for computational syntax.  These challenges include the syntactic functions of the various ``case'' morphemes, problems of ``zero derivation'', non-finite clauses, and copulas and copula constructions.  An existing morphological transducer of Kazakh \parencite{Washington14} implements analyses of how these various phenomena occur on the morphological level.  These phenomena will be described in this section, and how they were dealt with in the annotation of the treebank will be described in section~\ref{sec:annotation}. 

In Kazakh, as in most languages with case, there is not a one-to-one relation between ``case'' morphemes and syntactic function (not to mention a wide range of semantic functions).  The main syntactic functions of the traditionally defined cases in Kazakh are summarised in table~\ref{tab:cases}.

\begin{table}[htbp]
	\centering
	\caption{Primary syntactic functions of traditionally defined cases in Kazakh.}\label{tab:cases}
	\begin{small}
		\begin{tabular}{l l p{8em} p{27.15em}}
			\toprule
				\textbf{Case} & \textbf{Morpheme} & \textbf{Functions} & \textbf{Examples} \\
			\midrule
				nominative & - & subject\newline
										attributive\newline
										indefinite object\newline
										indefinite genitival 
									& \kazakh{\textbf{Дәрігер} үйді көреді.} \gloss{\textbf{The doctor} sees the house.} \newline
										\kazakh{\textbf{қонақ} үй} \gloss{hotel (lit., \textbf{guest} house)} \newline 
										\kazakh{Дәрігер \textbf{үй} көреді.} \gloss{The doctor sees \textbf{a house}.}\newline
										\kazakh{\textbf{үй} жануарлары} \gloss{\textbf{house} animals} \\\midrule
				accusative & -/NI/ & definite object\newline
											embedded subject\newline
											intr.\ causative subj.
										& \kazakh{Дәрігер \textbf{үйді} көреді.} \gloss{The doctor sees \textbf{the house}.} \newline
											bar \\\midrule
				genitive & -/NIң/ & definite genitival\newline
											embedded subject
										& \kazakh{\textbf{үйдің} жануарлары} \gloss{the animals of \textbf{the house}} \newline
											\kazakh{Ол \textbf{Айгүлдің} ойнағанына қарап тұр.} \gloss{She's watching \textbf{Aygül} play.} \\\midrule
				locative & -/DA/ & adverbial
										& \\\midrule
				ablative & -/DAн/ & adverbial,\newline
											comparator \\\midrule
				dative & -/GA/ & indirect object\newline
										attributive\newline
										tr.\ causative subj.
									& \\\midrule
				instrumental & -/Mен/ &  \\
			\bottomrule
		\end{tabular}
	\end{small}
\end{table}

As seen in the table, the morphologically unmarked nominative case (e.g., \kazakh{үй} \gloss{the/a house} \gmk{nom}) has a wide variety of uses, including indefinite object and genitival.  Definite objects are marked with the accusative case (e.g., \kazakh{үйді} \gloss{the house} \gmk{acc}), and definite genitivals are marked with the genitive case (e.g., \kazakh{үйдің} \gloss{of the house} \gmk{gen}).  The Kazakh transducer marks all bare nominals both as \tag{nom} and \tag{attr}, but the various functions may be disambiguated syntactically.  Genitival nominals, whether definite or not, must have a corresponding nominal with possessive morphology that agrees in person and number (e.g., \kazakh{үйдің есіг\textbf{і}} \gloss{the/a door to the house}; \kazakh{үй тапсырма\textbf{сы}} \gloss{homework}), and subject nominals must have a corresponding predicate containing e.g., a verb or a copula that agrees in person and number with the nominative nominal (e.g., \kazakh{үй көшіріле\textbf{ді}} \gloss{the house gets moved}).  When a nominative nominal depends on another nominal that does not have possessive case, the first one must be considered attributive (e.g., \kazakh{үй киім} \gloss{house clothes}).

The attributive use of nominals is similar to the use of adjectives, and could even be thought of as a ``zero-derivation'' of nominals into adjectives.  Interestingly, many adjectives can also be used substantively, i.e., as nominals.  Table~\ref{tab:zeroderiv} shows the various functions that of nominals, adjectives, and adverbs may take.

\begin{table}[htbp]
	\centering
	\caption{The default (first line of each) and ``zero-derived'' uses of nominals, adjectives, and adverbs}\label{tab:zeroderiv}
	\begin{tabular}{lp{5em}p{20em}}
		\toprule
			\textbf{Category} & \textbf{Function} & \textbf{Example} \\
		\midrule
			Nominals & Substantive\newline
							Attributive\newline
							Adverbial
						& \kazakh{\textbf{Қонақ}тарымыз бай.} \gloss{Our \textbf{guest}s are rich.}\newline 
							\kazakh{\textbf{қонақ} үй} \gloss{hotel (lit., \textbf{guest} house)}\newline
							\kazakh{бір \textbf{рет}} \gloss{one \textbf{time}} \\\midrule
			Adjectives & Attributive\newline
							Substantive\newline
							Adverbial
						& \kazakh{\textbf{жақсы} үй} \gloss{\textbf{nice} house} \newline
						\kazakh{ең \textbf{жақсы}лары} \gloss{\textbf{the nice}st \textbf{ones}} \newline
						\kazakh{Оны \textbf{жақсы} танимын.} \gloss{I know him/her \textbf{well}.}
						\\\midrule
			Adverbs & Adverbial
						& \kazakh{Ол \textbf{кеше} кетті.} \gloss{S/he left \textbf{yesterday}.}
						\\
		\bottomrule
	\end{tabular}

\end{table}

Nominals (\tag{n}) default to substantive and adjectives (\tag{adj}) default to attributive.  For the other readings, the transducer provides readings such as \tag{adj}\tag{advl} and \tag{n}\tag{attr}.

%Another type of ``zero-derivation'' involves the use of adjectives and nouns.  In Kazakh, adjectives are all attributive, but many may also be used either nominally (i.e., as nouns) or adverbially (i.e., as adverbs), or both.  The morphological transducer internally provides a series of classes; depending on class membership, an adjective may receive the readings \tag{adj}\tag{advl} or \tag{adj}\tag{subst}, besides the default attributive reading of \tag{adj}.  If an adjective may be used substantivally, then it will receive a range of substantive morphology as well.  Similarly, many nouns may be used attributively, besides the default substantive reading.  Besides the nominative (and indefinite accusative/genitive) reading of \tag{nom}, the transducer provides an attributive reading for most nominals of \tag{attr}.

Kazakh, like most Turkic languages, makes frequent use of non-finite verb forms by deriving verbal adjectives, verbal nouns, and verbal adverbs from verbs.  Verbal adjective phrases modify nouns, as in \kazakh{\textbf{мені көрген} дәрігер} \gloss{the doctor \textbf{that saw me}}.  Verbal nouns can function as subjects or complements of verbs or copulas, as in \kazakh{Дәрігер сені \textbf{көргенін} білген жоқпын.} \gloss{I didn't know that the doctor had seen you.} and \kazakh{Дәрігер сені \textbf{көргені} жақсы болыпты.} \gloss{It's good that doctor saw you.}  Verbal adverbs allow a verb phrase to function as a clausal verbal adjunct, as in \kazakh{Мен \textbf{дәрігерді көріп} қуанып кеттім.} \gloss{\textbf{Seeing the doctor}, I got happy.}  Verbal nouns with certain case morphology and/or occurring with certain postpositions can also function as a verbal adjunct much in the same way that verbal adverb clauses do, as in \kazakh{Мен \textbf{дәрігерді көргенде} қуанып кеттім.} \gloss{I got happy \textbf{when I saw the doctor}.}, here with a verbal noun in the locative case.

Another category that may also be non-overt is the copula.  The primary strategy for copula constructions in Kazakh is the use of a defective verb \kazakh{е-}.  In the present tense, the verb itself does not surface, but agreement morphology surfaces (cliticised to the previous word) in all but the third person forms (e.g., \kazakh{Мен үйде\textbf{мін}.} \gloss{I'm at home.} versus \kazakh{Ол үйде.} \gloss{S/he's at home.}).  The defective copula verb also surfaces in the recent past tense (e.g., \kazakh{Мен үйде еді\textbf{м}.} \gloss{I was at home.}, \kazakh{Інім үйде еді.} \gloss{My younger brother was at home.}).  Of particular interest are non-finite forms of the copula.  Copula clauses can be attributive, as in \kazakh{\textbf{үйі жақсы} дәрігер} \gloss{the doctor \textbf{who has a nice house}}, or literally \gloss{\textbf{his/her house is nice} the doctor}.  While this construction never has overt copula marking, copula clauses which are complements of verbs always have an overt copula form, e.g. \kazakh{Дәрігердің үйі жақсы \textbf{екенін} білген жоқпын.} \gloss{I didn't know \textbf{that} the doctor's house \textbf{was} nice.}  Here, \kazakh{екен} is a suppletive gerundive form of the copula and is marked as accusative.


How each of these issues bears on a dependency analysis will be discussed in section~\ref{sec:annotation}.

% case and function
%% indefinite acc/gen = "nom"
%% attr/subst/advl function of cases
% "zero" derivation
%% attr/subst/etc.
%% copulas?

\subsection{Treebanks} % Related work

A treebank is a parsed corpus of sentences annotated syntactically following a particular
syntactic theory. Two broad groups can be distinguished, phrase-structure treebanks
which annotate constituency strucure, and dependency treebanks which annotate dependency
structure. Some treebanks combine both.

Treebanks can be used directly for linguistic and computational linguistic research, by
performing search queries. For example, to extract a valency lexicon for verbs, or to
study the frequency of various syntactic phenomena such as word order or nominal case usage
and syntactic function.

They can also be used to train statistical parsers which can
be used to annotate previously unseen texts. These parsers can be used in end-user applications
such as machine translation and computer-aided language learning.

According to \textcite{nivre08}, a parser trained on a treebank of only 1,500 sentences 
can provide reasonable parsing accuracy.


\section{Methodology}\label{sec:method}

\subsection{Corpus}

\begin{table}[htbp]
  \centering
  \caption{Composition of the corpus. The corpus covers a range of genres and text types from free and public-domain sources.}
	\begin{small}
		\begin{tabular}{llrrr}
			\toprule
				\textbf{Document} & \textbf{Description} & \textbf{Sentences} & \textbf{Tokens} & \textbf{Avg. length}\\
			\midrule
				UN Declaration on Human Rights & Legal text on human rights & 25 & 409 & 16.3 \\
				Phrasebook                     & Phrases from Wikitravel   & 37 & 205 & 5.5 \\
				Жиырма Бесінші Сөз             & Philosophical text        & 34 & 525 & 15.4 \\ 
				Қожанасырдың тойға баруы       & Folk tale from Wikisource  & 8 & 140 & 17.5 \\
				Ер Төстік                      & Folk tale from Wikisource  & 23 & 203 & 8.8 \\
				Азамат қайда?                  & Story for language learners & 48 & 434 & 9.0 \\
				Футболдан әлем чемпионаты 2014 & Wikipedia article (2014 World Cup) & 14 & 246 & 17.5 \\
				Иран & Wikipedia article (Iran)                                & 111 & 1562 & 14.0 \\
				Радиан & Wikipedia article (Radian)                            & 2 & 17 & 8.5 \\
			\midrule
			\multicolumn{2}{c}{~} & 302 & 3741 & 12.3 \\\cmidrule[\heavyrulewidth]{3-5}
		\end{tabular}
	\end{small}
\end{table}


\subsection{Preprocessing}

%\begin{figure}
%\begin{tiny}
%\begin{alltt}
%"<Елде>"
%        "ел" n loc
%"<ағылшын>"
%        "ағылшын" n nom
%"<үстемдігіне>"
%        "үстемдік" n px3sp dat
%"<қарсы>"
%        "қарсы" post
%"<жаңа>"
%        "жаңа" adj
%"<толқулар>"
%        "толқу" n pl nom
%"<басталды>"
%        "баста" v tv pass ifi p3 pl
%"<.>"
%        "." sent
%\end{alltt}
%\end{tiny}
%\caption{Example output after preprocessing and manual disambiguation. Each token is annotated for lemma,
%   part of speech and morphosyntactic information.}
%\end{figure}

Preprocessing the corpus consists of running the text through the Kazakh morphological
analyser \parencite{Washington14}, which also performs tokenisation of multiword units based 
the longest match left-to-right. Tokenisation for Kazakh is a non-trivial task, and so
we do not simply take space as a delimiter. The morphological analyser returns all 
the possible morphological analyses for each word based on a lexicon of around 20,000 lexemes.
After tokenisation and morphological analysis, the text is processed with a constraint-grammar 
based disambiguator for Kazakh consisting of 113 rules which remove inappropriate 
analyses in context. This reduces the average number of analyses per word from around 3.4
to around 1.7.

% tokenisation
% http://universaldependencies.github.io/docs/u/overview/tokenization.html

\subsubsection{Tokens and words}

Tokenisation of the corpus is performed by our morphological analyser. This analyser
performs tokenisation on the basis of a left-to-right longest match algorithm described
in \textcite{garrido02}. Simple tokens such as \kazakh{толқулар} \gloss{riots} are maintained
as a single token, and their lemma and morphological analysis is returned. Multiword
units such as \kazakh{ауа райы} \gloss{weather} and \kazakh{ата-анасының} \gloss{of their parents} are combined
into a single token. Abbreviations and numerals which bear case, such as \kazakh{АҚШ-пен} \gloss{with the USA}
and \kazakh{90\%-ына} \gloss{to 90\%} are analysed as a single token, as are light-verbs such as
\kazakh{пайда бол} \gloss{to appear} and tense forms written with space like \kazakh{оқыған жоқ},
the third-person negative past of \kazakh{оқы-} \gloss{to study/read}.

In some cases a single token is split into two tokens, as with the aorist copula suffixes,
\kazakh{үйдемін} \gloss{I am in the house} is tokenised as \kazakh{үй.\gmk{loc} + е.\gmk{cop.aor.sg1}}. Furthermore
two input tokens may result in three output tokens, \kazakh{бар ма?} \gloss{is there?} would be
tokenised as \kazakh{бар.\gmk{adj} + е.\gmk{cop.aor.sg1} + ма.\gmk{qst}}.


% productive suffixes -lı (cite wafflelazor)

\section{Annotation guidelines}\label{sec:annotation}

%FIXME:
% general procedure

% - disambiguate
% - annotate
% - check the tree structure
% - fix errors
% - check for label consistency
% - fix errors

\begin{table}[htbp]
        \centering
\begin{small}
        \begin{tabular}{ll|ll|ll}
                \toprule
                \textbf{Label} & \textbf{Description} & \textbf{Label} & \textbf{Description} & \textbf{Label} & \textbf{Description} \\
                \midrule
\texttt{acl} & clausal modifier of noun & \texttt{\sout{csubjpass}} & clausal passive subject & \texttt{\sout{neg}} & negation modifier\\
\texttt{advcl} & adverbial clause modifier & \texttt{dep} & unspecified dependency & \texttt{nmod} & nominal modifier\\
\texttt{advmod} & adverbial modifier & \texttt{det} & determiner & \texttt{\sout{nsubj}} & nominal subject\\
\texttt{amod} & adjectival modifier & \texttt{discourse} & discourse element & \texttt{\sout{nsubjpass}} & passive nominal subject\\
\texttt{appos} & appositional modifier & \texttt{\sout{dislocated}} & dislocated elements & \texttt{nummod} & numeric modifier\\
\texttt{aux} & auxiliary & \texttt{dobj} & direct object & \texttt{parataxis} & parataxis\\
\texttt{\sout{auxpass}} & passive auxiliary & \texttt{\sout{expl}} & expletive & \texttt{punct} & punctuation\\
\texttt{case} & case marking & \texttt{foreign} & foreign words & \texttt{remnant} & remnant in ellipsis\\
\texttt{cc} & coordinating conjunction & \texttt{\sout{goeswith}} & goes with & \texttt{\sout{reparandum}} & overridden disfluency\\
\texttt{ccomp} & clausal complement & \texttt{iobj} & indirect object & \texttt{root} & root\\
\texttt{compound} & compound & \texttt{list} & list & \texttt{vocative} & vocative\\
\texttt{conj} & conjunct & \texttt{mark} & marker & \texttt{xcomp} & open clausal complement\\
\texttt{cop} & copula & \texttt{\sout{mwe}} & multi-word expression & \\
\texttt{\sout{csubj}} & clausal subject & \texttt{name} & name & \\

                \bottomrule
        \end{tabular}
\end{small}
        \caption{Universal dependency label set}
\end{table}

%\begin{small}
%\begin{multicols*}{3}
	%\tablehead{\toprule \textbf{Label} & \textbf{Description}\\\midrule}
	%\tabletail{\bottomrule}
%\TrickSupertabularIntoMulticols

	%\begin{supertabular}{ll}
%\texttt{acl} & clausal modifier of noun\\
%\texttt{advcl} & adverbial clause modifier\\
%\texttt{advmod} & adverbial modifier\\
%\texttt{amod} & adjectival modifier \\
%\texttt{appos} & appositional modifier\\
%\texttt{aux} & auxiliary  \\
%\texttt{\sout{auxpass}} & passive auxiliary \\
%\texttt{case} & case marking \\
%\texttt{cc} & coordinating conjunction \\
%\texttt{ccomp} & clausal complement \\
%\texttt{compound} & compound \\
%\texttt{conj} & conjunct\\ 
%\texttt{cop} & copula \\
%\texttt{\sout{csubj}} & clausal subject \\
%\texttt{\sout{csubjpass}} & clausal passive subject\\
%\texttt{dep} & unspecified dependency\\
%\texttt{det} & determiner \\
%\texttt{discourse} & discourse element \\
%\texttt{\sout{dislocated}} & dislocated elements \\
%\texttt{dobj} & direct object \\
%\texttt{\sout{expl}} & expletive \\
%\texttt{foreign} & foreign words \\
%\texttt{\sout{goeswith}} & goes with \\
%\texttt{iobj} & indirect object \\
%\texttt{list} & list \\
%\texttt{mark} & marker \\
%\texttt{\sout{mwe}} & multi-word expression \\ 
%\texttt{name} & name \\
%\texttt{\sout{neg}} & negation modifier\\
%\texttt{nmod} & nominal modifier\\
%\texttt{\sout{nsubj}} & nominal subject\\
%\texttt{\sout{nsubjpass}} & passive nominal subject\\
%\texttt{nummod} & numeric modifier\\
%\texttt{parataxis} & parataxis\\
%\texttt{punct} & punctuation\\
%\texttt{remnant} & remnant in ellipsis\\
%\texttt{\sout{reparandum}} & overridden disfluency\\
%\texttt{root} & root\\
%\texttt{vocative} & vocative\\
%\texttt{xcomp} & open clausal complement\\
	%\end{supertabular}
%\end{multicols*}
%\end{small}

\subsection{Copula}

The copula (both \emph{е-} and \emph{бол-}) is a challenging problem for dependency 
analysis of Kazakh. The universal dependency guidelines state that the copula should
be the dependent of the lexical predicate. However, in many cases the copula in Kazakh
is found in embedded clauses ...

We have uniformly annotated the copula as a leaf node with the predicate, or adverbial
as the head of the structure. For certain structures this is convenient, such as the
bare copula in phrases like \ref{fig:cop1} or \ref{fig:cop2}, but for phrases where
the copula is part of an embedded clause this is not necessarily the most  
effective choice. In \ref{fig:cop3}, the copula holds all the morphological information,
agreement with the subject and accusative marking for the embedded clause.

\begin{figure}[htbp]
        \centering

        \begin{subfigure}[b]{0.25\textwidth}
                \centering
                \begin{dependency}[theme = simple, font = \small]
                   \begin{deptext}[column sep=0.2cm]
                                Айгүл \& студент \& · \\
                                \gmk{np} \& \gmk{n} \& \gmk{aor} \\
                                \gloss{Aygül} \& \gloss{student} \& \gloss{is} \\
                        \end{deptext}
                        \deproot[edge unit distance=1.3ex]{2}{\udlabel{root}}
                        \depedge{2}{1}{\udlabel{subj}}
                        \depedge{2}{3}{\udlabel{cop}}
                \end{dependency}
                \caption{Non-surfaced copula.}\label{fig:cop1}
        \end{subfigure}
        \quad
        \begin{subfigure}[b]{0.25\textwidth}
                \centering
                \begin{dependency}[theme = simple, font = \small]
                   \begin{deptext}[column sep=0.2cm]
                                Айгүл \& студент \& болған \\
                                \gmk{np} \& \gmk{n} \& \gmk{past} \\
                                \gloss{Aygül} \& \gloss{student} \& \gloss{was}\\
                        \end{deptext}
                        \deproot[edge unit distance=1.3ex]{2}{\udlabel{root}}
                        \depedge{2}{1}{\udlabel{subj}}
                        \depedge{2}{3}{\udlabel{cop}}
                \end{dependency}
                \caption{Copula in past tense.}\label{fig:cop2}
        \end{subfigure}
        \quad
        \begin{subfigure}[b]{0.40\textwidth}
                \centering
                \begin{dependency}[theme = simple, font = \small]
                   \begin{deptext}[column sep=0.2cm]
                                Айгүл \& оның \& қайда \& екенін \& білмейді. \\
                                \gmk{np} \& \gmk{gen} \& \gmk{adv} \& \gmk{cop} \& \gmk{neg.aor} \\
                                \gloss{Aygül} \& \gloss{he} \& \gloss{where} \& \gloss{being} \& \gloss{knows not} \\
                        \end{deptext}
                        \deproot[edge unit distance=1.3ex]{5}{\udlabel{root}}
                        \depedge{3}{2}{\udlabel{subj}}
                        \depedge{5}{3}{\udlabel{ccomp}}
                        \depedge{3}{4}{\udlabel{cop}}
                \end{dependency}
                \caption{Copula in an embedded clause.}\label{fig:cop3}
        \end{subfigure}
        \caption{Dependency trees of copula constructions.}
\end{figure}




\subsection{Coordination}

One difference in our annotation scheme compared to the standard universal dependency
analysis is with coordination. While the universal dependency scheme takes the first
conjunct as the head, we take the last. This decision was made based on the fact that
Kazakh is a head-final language and morphological marking is only obligatory on the
last conjunct in a series. Furthermore, experiments in representing coordination
in other head-final languages have found that the final-conjunct head analysis 
results in better parser accuracy \parencite{bengoetxea09}.

\begin{figure}[htbp]
    \centering

        \begin{dependency}[theme = simple, font = \small]
           \begin{deptext}[column sep=0.08cm]
%               1         2         3          4              5          6          7        8
                Олар \& Финляндия, \& Швеция \& және \& Эстонияидан \& кейін \& Ресейге \& барды. \\
                \gmk{prn} \& \gmk{nom} \& \gmk{nom} \& \gmk{cc} \& \gmk{abl} \& \gmk{post} \& \gmk{dat} \& \gmk{past} \\
                \tgloss{They} \& \tgloss{Finland}, \& \tgloss{Sweden} \& \tgloss{and} \& \tgloss{Estonia} \& \tgloss{after} \& \tgloss{Russia} \& \tgloss{visited.} \\
                \end{deptext}
                \depedge{5}{3}{\udlabel{conj}}
                \depedge{5}{4}{\udlabel{cc}}
                \depedge{5}{2}{\udlabel{conj}}
                \depedge{5}{6}{\udlabel{case}}
                \depedge{8}{5}{\udlabel{nmod}}
                \depedge{8}{7}{\udlabel{nmod}}
                \deproot[edge unit distance=1.3ex]{8}{\udlabel{root}}
        \end{dependency}
        \caption{Coordination: All conjucts are attached to the final conjunct, which is the head of the coordinated phrase.}\label{fig:coord}

\end{figure}



\subsection{Complex nominals}

% compounds + izafet

There are different ways in which two nominals may occur together to act as a single nominal.  Compounds are formed by an attributive nominal (indistinguishable from the bare / nominative form, but tagged with \tag{attr}) preceding another nominal, as shown in \ref{fig:compound}.  An indefinite genitive construction is formed by an indefinite genitive nominal (indistinguishable from the bare / nominative form, and tagged with \tag{nom}) preceding a nominal that has third-person possessive morphology, as shown in \ref{fig:indefgen}.  A definite genetive construction is formed by a genitive-marked nominal (tagged \tag{gen}) precding a nominal that has third-person possessive morphology, as shown in \ref{fig:defgen}.

%One example of this is indefinite genitives.  In such constructions, a ``modifying'' nominal is morphologically not marked (which means it will be tagged as \tag{nom}, as discussed in §\ref{sec:kazakh}), though should be considered an indefinite genitive.  The modified element is a nominal with third-person possessive morphology, an example of which is provide in figure~\ref{fig:indefgen}.

\begin{figure}[htbp]
	\centering

	\begin{subfigure}[b]{0.3\textwidth}
		\centering
		\begin{dependency}[theme = simple, font = \small]
		   \begin{deptext}[column sep=0.3cm]
				көрші \& елдер \\
				\gmk{n.attr} \& \gmk{n.pl.nom} \\
				\tgloss{neighbour} \& \tgloss{countries} \\
			\end{deptext}
			\depedge{2}{1}{\udlabel{compound}}
		\end{dependency}
		\caption{Compounding: an attributive nominal depending on a nominal.}\label{fig:compound}
	\end{subfigure}
	\quad
	\begin{subfigure}[b]{0.3\textwidth}
		\centering
		\begin{dependency}[theme = simple, font = \small]
		   \begin{deptext}[column sep=0.3cm]
				әлем \& чемпионаты \\
				\gmk{n.nom} \& \gmk{n.px3sg.nom} \\
				\tgloss{world} \& \tgloss{championship} \\
			\end{deptext}
			\depedge{2}{1}{\udlabel{nmod}}
		\end{dependency}
		\caption{Indefinite genitive: An indefinite genitive depending on a nominal.}\label{fig:indefgen}
	\end{subfigure}
	\quad
	\begin{subfigure}[b]{0.3\textwidth}
		\centering
		\begin{dependency}[theme = simple, font = \small]
		   \begin{deptext}[column sep=0.3cm]
				Иранның \& экономиясы \\
				\gmk{n.gen} \& \gmk{n.px3sg.nom} \\
				\tgloss{Iran's} \& \tgloss{economy} \\
			\end{deptext}
			\depedge{2}{1}{\udlabel{nmod}}
		\end{dependency}
		\caption{Definite genitive: A definite genitive depending on a nominal.}\label{fig:defgen}
	\end{subfigure}

	\caption{Dependency trees of complex nominal relations.}
\end{figure}

As seen in the graphs, the compound relationship of an attributive nominal depending on another nominal is labelled \udtag{compound}, and genitive relations are considered \udtag{nmod}, regardless of whether there is a definite or indefinite genitive construction.


\subsection{Non-finite clauses}

As discussed in section~\ref{sec:kazakh}, Kazakh makes extensive use of non-finite clauses, including verbal adjective clauses, verbal noun clauses, and verbal adverb clauses.

Verbal adjective clauses modify a head nominal, effectively allowing a whole verb phrase to act as an adjective.  The dependency relation between them is \udtag{acl}, per UD documentation.  An example is provided in figure~\ref{fig:gpr}

\begin{figure}[htbp]
	\centering
	\begin{dependency}[theme = simple, font = \small]
		\begin{deptext}[column sep=0.08cm]
			Бразилия \& өз \& жерінде \& чемпионатты \& екі \& рет \& өткізген \& бесінші \& ел \& болды. \\
			\gmk{np.nom} \& \gmk{det} \& \gmk{nom.px3sp.loc} \& \gmk{n.acc} \& \gmk{num} \& \gmk{n} \& \gmk{v.tv.gpr} \& \gmk{num.ord} \& \gmk{n.nom} \& \gmk{v.iv.ifi} \\
			\tgloss{Brazil} \& \tgloss{own} \& \tgloss{in its land} \& \tgloss{championship} \& \tgloss{two} \& \tgloss{times} \& \tgloss{having held} \& \tgloss{fifth} \& \tgloss{country} \& \tgloss{became} \\
		\end{deptext}
		\depedge{3}{2}{\udlabel{det}}
		\depedge{7}{3}{\udlabel{nmod}}
		\depedge{7}{4}{\udlabel{obj}}
		\depedge{6}{5}{\udlabel{nummod}}
		\depedge{7}{6}{\udlabel{nmod}}
		\depedge{9}{7}{\udlabel{acl}}
		\depedge{9}{8}{\udlabel{amod}}
		\depedge{9}{10}{\udlabel{cop}}
		\deproot[edge unit distance=1.3ex]{9}{\udlabel{root}}
	\end{dependency}
	\caption{Verbal adjectives: verbal adjectives are the head of everything in their own clause and are an \udtag{acl} dependency of the noun they modify.}\label{fig:gpr}
\end{figure}

Verbal adverb clauses in Kazakh act as a clausal verbal adjunct, essentially allowing a whole verb phrase to act as an adverb.  The dependency relation between the verbal adverb and the head verb is \udtag{advcl}, per UD documentation.  An example is provided in figure~\ref{fig:gna}.

\begin{figure}[htbp]
	\centering
	\begin{dependency}[theme = simple, font = \small]
		\begin{deptext}[column sep=0.08cm]
			1960 \& жылдан \& бастап \& Батыс \& үлгісінде \& дами \& бастады. \\
			\gmk{num} \& \gmk{n.abl} \& \gmk{v.tv.gna} \& \gmk{n.nom} \& \gmk{n.px3sp.loc} \& \gmk{v.iv.prc} \& \gmk{vaux.ifi} \\
			\tgloss{1960} \& \tgloss{from year} \& \tgloss{starting} \& \tgloss{the West} \& \tgloss{on its model} \& \tgloss{developing} \& \tgloss{started} \\
		\end{deptext}
		\depedge{2}{1}{\udlabel{amod}}
		\depedge{3}{2}{\udlabel{nmod}}
		\depedge{6}{3}{\udlabel{advcl}}
		\deproot[edge unit distance=1.3ex]{6}{\udlabel{root}}
	\end{dependency}
	\caption{Verbal adverbs: verbal adverbs are the head of everything in their own clause and are an \udtag{advcl} dependency of the verb they are subordinate to.}\label{fig:gna}
\end{figure}

Verbal nouns allow a whole verb phrase to be used as a nominal, and the resulting verbal noun phrase may be used as a subject or object of a verb.  When used as the subject of a verb, verbal noun phrases receive the UD label \udtag{subj}.  When used as the object of a verb, verbal noun phrases receive the UD label \udtag{ccomp}, unless the subject of the subordinate clause is obligatorily identical to the subject of the main clause, in which case it receives \udtag{xcomp}.  An example of a verbal noun phrase used as a \udtag{ccomp} dependency of the main verb is given in figure~\ref{fig:gerdat}.

\begin{figure}[htbp]
	\centering
	\begin{dependency}[theme = simple, font = \small]
		\begin{deptext}[column sep=0.08cm]
			Ол \& терезеден \& Азамат \& пен \& Айгүлдің \& ойнағанына \& қарап \& тұр. \\
			\gmk{prn} \& \gmk{n.abl} \& \gmk{np.nom} \& \gmk{cc} \& \gmk{np.gen} \& \gmk{v.iv.ger.px3sp.dat} \& \gmk{v.tv.prc} \& \gmk{vaux.pres} \\
			\tgloss{she} \& \tgloss{from the window} \& \tgloss{Azamat} \& \tgloss{and} \& \tgloss{Aygül's} \& \tgloss{playing} \& \tgloss{watching} \& \tgloss{is} \\
		\end{deptext}
		\depedge{5}{3}{\udlabel{conj}}
		\depedge{5}{4}{\udlabel{cc}}
		\depedge{6}{5}{\udlabel{subj}}
		\depedge{7}{6}{\udlabel{ccomp}}
		\deproot[edge unit distance=1.3ex]{7}{\udlabel{root}}
	\end{dependency}
	\caption{Verbal nouns: verbal nouns are the head of everything in their own clause and can be a \udtag{ccomp} dependency of the verb they are the complement of.  Note here that the subject noun phrase of the subordinate clause is in genitive case.}\label{fig:gerdat}
\end{figure}

As mentioned earlier, verbal nouns combined with certain case morphology and/or postpositions can function as an adverbial adjunct to a main clause, similar to verbal adverb clauses.  In such instances, they receive the UD label \udtag{advcl}, and example of which is given in figure~\ref{fig:gerabl}.

\begin{figure}[htbp]
	\centering
	\begin{dependency}[theme = simple, font = \small]
		\begin{deptext}[column sep=0.08cm]
			Айгүл \& санап \& біткеннен \& кейін \& айналасына \& қарады. \\
			\gmk{n.nom} \& \gmk{v.iv.prc} \& \gmk{vaux.ger.abl} \& \gmk{post} \& \gmk{n.px3sp.dat} \& \gmk{v.tv.ifi} \\
			\tgloss{Aygül} \& \tgloss{counting} \& \tgloss{having finished} \& \tgloss{after} \& \tgloss{to her surroundings} \& \tgloss{looked} \\
		\end{deptext}
		\depedge{6}{2}{\udlabel{advcl}}
		\depedge{2}{3}{\udlabel{aux}}
		\depedge{3}{4}{\udlabel{post}}
		\depedge{6}{5}{\udlabel{nmod}}
		\deproot[edge unit distance=1.3ex]{6}{\udlabel{root}}
	\end{dependency}
	\caption{Verbal nouns with case/postpositions: verbal nouns with certain case morphology and/or postpositions are the head of everything in their own clause and can be an \udtag{advcl} dependency of the verb they are the complement of.}\label{fig:gerabl}
\end{figure}

% verbal adjectives → acl
% verbal nouns (gerunds) → ccomp, no xcomp
% verbal adverbs → advlc
% * gna_cnd? → advlc?

\section{Evaluation}\label{sec:eval}

In order to test the utility of the treebank in a real-world setting, we trained
and evaluated a number of models using the popular MaltParser tool \parencite{nivre07}.
MaltParser is a toolkit for data-driven dependency parsing, it can learn a parsing
model from treebank data and apply this model to parse unseen sentences. The parser
has a large number of options and parameters that need to be optimised. 
To select the best parser configuration we relied on 
MaltOptimiser \parencite{ballesteros15}. The optimiser was run separately for each of 
the model configurations.

As the treebank takes advantage of the new tokenisation standards in the CoNLL-U format,
and MaltParser only supports CoNLL-X, certain transformations were needed to perform 
the experiments. The corpus was flattened with conjoined tokens receiving a dummy 
surface form. The converted corpus is available alongside the original.\footnote{\url{removed for review}}

% system1:
% nivreeager
% LAS: 0.3341 [0.234 0.288 0.289 0.292 0.333 0.335 0.359 0.378 0.408 0.425]
% UAS: 0.5093 [0.413 0.464 0.465 0.466 0.474 0.499 0.531 0.545 0.604 0.632]

% system2:
% nivreeager
% LAS: 0.234+0.288+0.289+0.292+0.333+0.335+0.359+0.378+0.408+0.425
% UAS: 0.413+0.464+0.465+0.466+0.474+0.499+0.531+0.545+0.604+0.632

% system3:
% nivreeager
% LAS: 0.5529 [0.434 0.446 0.501 0.511 0.515 0.542 0.558 0.572 0.7 0.75]
% UAS: 0.7376 [0.658 0.669 0.675 0.684 0.694 0.723 0.751 0.776 0.825 0.921]

% system4:
% covnonproj
% LAS: 0.6092 [0.519 0.535 0.544 0.569 0.569 0.62 0.621 0.649 0.683 0.783]
% UAS: 0.7575 [0.674 0.686 0.69 0.714 0.727 0.746 0.795 0.803 0.812 0.928]

%             [0.434+0.446+0.501+0.511+0.515+0.542+0.558+0.572+0.7+0.75]
%             [0.658+0.669+0.675+0.684+0.694+0.723+0.751+0.776+0.825+0.921]

\begin{table}[htbp]
	\caption{Preliminary parsing results from MaltParser using different models. The numbers in brackets denote the upper and lower bounds found during cross validation. Adding structural features to the model substantially improves the performance of the parser, although we find no improvement in using full morphosyntactic description (MSD) over using simply the first part-of-speech (POS) tag.}
	\centering
	\begin{tabular}{llrr}
		\toprule
			\textbf{Features}       & \textbf{Algorithm} &\textbf{LAS} [range] & \textbf{UAS} [range] \\
		\midrule
			surface                & nivreeager  & 33.4 [23.4, 42.5] & 50.9 [41.3, 63.2] \\
			surface+lemma          & nivreeager  & 33.4 [23.4, 42.5] & 50.9 [41.3, 63.2] \\
			surface+lemma+POS      & nivreeager  & 55.2 [43.4, 75.0] & 73.7 [65.8, 92.1] \\
			surface+lemma+POS+MSD  & nivreeager  & 55.2 [43.4, 75.0] & 73.7 [65.8, 92.1] \\
		\bottomrule
	\end{tabular}
	\label{table:eval}
\end{table}

To perform 10-fold cross validation we randomised the order of sentences in the corpus
and split it into 10 equally-sized parts. In each iteration we held out one part for testing and used
the rest for training. We calculated the labelled-attachment score (LAS) and 
unlabelled-attachment score (UAS) for each of the models.

The results we obtain are similar to those obtained with similar 
sized treebanks, for example the Irish treebank of \textcite{Lynn12}, who report an LAS of
63.3 and a UAS of 73.3 with the best model.


\section{Future work}\label{sec:future}

%stabilise guidelines

%crosslingual parsing: Tatar, Kumyk, Kyrgyz, Tuvan

%language-specific dependency relations --- (acl:relcl, ...)

Future work will focus on improving the annotation guidelines and the consistency of
annotation in the corpus. We will also study the possibility of deepening the annotation
with Turkic-specific relations. When we have stable annotation guidelines we intend to 
extend the corpus with more texts.
We would also like to work on cross-lingual dependency parsing,
that is, applying a model learnt on the Kazakh treebank to other Turkic languages
such as Tatar, Kumyk and Tuvan. We have morphological analysers for these languages which have
compatible tagsets for morphological features and as such it should be possible to learn a
delexicalised model based on these features. As Turkic syntax is broadly homogenous, this
presents a promising avenue for future work.


\section{Concluding remarks}\label{sec:conclusions}

We have presented the first steps towards a free/open-source dependency treebank for
Kazakh with annotation based on the universal dependencies. The treebank is small, but
provides a base for bootstrapping further. Performance of a state-of-the-art statistical
parser trained on the treebank is comparable to other treebanks of similar size.

\section*{Acknowledgements}

%Tolgonay?, Aida?

%\bibliographystyle{apalike}
%\bibliography{paper}
\begin{small}
\printbibliography
\end{small}
\end{document}
