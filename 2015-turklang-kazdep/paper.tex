\documentclass[a4paper,11pt, onecolumn,twoside]{article}

\usepackage{turklang}


%%%%% Language and font stuff %%%%%

\usepackage{alltt}
\usepackage{polyglossia}
\setdefaultlanguage[variant=australian]{english}

\usepackage{fontspec}
\usepackage{xunicode}

%\usepackage{times}
\setmainfont{Times New Roman}
\newfontfamily\ll[]{Linux Libertine O}
\newfontfamily\udfont[Scale=1.3,Letters=SmallCaps]{Linux Libertine O}
\newfontfamily\fsfont[Scale=MatchLowercase]{FreeSerif}



%%%%% General formatting stuff %%%%%

\usepackage{multirow}
\usepackage[small,bf]{caption} % CHECK IF THIS IS OK
\usepackage[colorlinks=true,citecolor=black,linkcolor=black,urlcolor=blue]{hyperref}

\usepackage{subcaption}
\usepackage{booktabs}

%\usepackage{natbib}
%\bibdata{paper}
%\usepackage[backend=bibtex]{biblatex}
\usepackage[backend=biber,citestyle=apa,style=authoryear,maxbibnames=99,maxcitenames=2]{biblatex}
\bibliography{paper}
\renewcommand*{\nameyeardelim}{\addcomma\space}


%%%%% Stuff for this paper %%%%%
\usepackage{tikz-dependency}
\newcommand{\gmk}[1]{\texttt{#1}}
\newcommand{\gloss}[1]{{\em #1}}
\newcommand{\tag}[1]{\texttt{‹#1›}}
%\newcommand{\udtag}[1]{\texttt{@#1}}
\newcommand{\udtag}[1]{{\ll \textsc{#1}}}
\newcommand{\udlabel}[1]{{\udfont #1}}
\newcommand{\tilda}{{\fsfont ∼}}
%\newcommand{\corrauthor}[1]{\footnotemark\footnotetext{hargle}}


%%%%% Paper metadata %%%%%

\title{Towards a free/open-source universal-dependency treebank for Kazakh}
\name{Francis M. Tyers$^a$ and Jonathan Washington$^b$}
\shortauthor{Tyers \& Washington}
\address{$^a$ HSL-fakultehta, UiT Norgga árktalaš universitehta, N-9015 Tromsø, Norway\\
	$^b$ Departments of Linguistics and Central Eurasian Studies, Indiana University, Bloomington, IN 47405, USA\footnote{Corresponding author.  E-mail address: jonwashi@indiana.edu}}

\keywords{Kazakh; treebank; dependency grammar; universal dependency}

%250-300-word ``extended abstract in English''
\abstract{
This article describes the first steps towards a free/open-source dependency treebank for Kazakh based on universal dependency (UD) annotation standards. The treebank contains 302 sentences and is based on texts from a range of open-source and public domain sources. This ensures its free availability and extensibility. Texts in the treebank are first morphologically analysed and disambiguated and then annotated manually for dependency structure. 
}


%%%%% End preamble %%%%%

\begin{document}

%\thanks{hargle}
\maketitleabstract{}
\thispagestyle{firststyle}

\section{Introduction}

This article describes work towards the development of a dependency treebank for Kazakh, a Turkic language spoken in Central Asia and Europe.
Despite its status as a \emph{core}\footnote{FIXME} Turkic %FIXME
language, little computational-linguistic research has been published on syntactic
parsing for Kazakh.
A valuable resource in the study of syntactic parsing is
a treebank---a corpus of parsed text containing gold-standard syntactic annotation.

Freely available treebanks exist for many languages, such as large languages like Finnish \parencite{haverinen2013tdt,voutilainen11} 
and Polish \parencite{wolinski11} and smaller languages like Irish \parencite{Lynn12}. To our knowledge, however, a treebank exists for only one other Turkic language, Turkish \parencite{Oflazer03}, which is unfortunately not freely available.

In building our treebank we take advantage of existing work done on tokenisation,
morphological analysis and part-of-speech tagging for Kazakh. We also take a pragmatic
and iterative view of development of the treebank, in line with recent work 
on cross-linguistic parsing with universal dependencies \parencite{DeMarneffe14}.

The remainder of the paper is organised as follows. Section~\ref{sec:back} gives some
background linguistic information on Kazakh, and outlines some special challenges in
parsing Kazakh. In Section~\ref{sec:method} we describe the corpus that we annotated and the methodology
used in annotating it. Section~\ref{sec:annotation} gives a sketch of some decisions
we have made with respect to annotation guidelines, referring back to the discussion in
Section~\ref{sec:back}. For reasons of space, these guidelines are not complete, but
present a subset of guidelines which are of particular interest. A small experiment
in statistical dependency parsing using the corpus is presented in Section~\ref{sec:eval},
and in Sections~\ref{sec:future} and \ref{sec:conclusions} we give perspectives
for future work and some concluding remarks.

%\textcite{Atalay03}

\section{Background}\label{sec:back}

\subsection{Kazakh}\label{sec:kazakh}

Kazakh (қазақ тілі), a Turkic language of Central Asia and Europe, is spoken by around 13 million people in Kazakhstan, China, Mongolia, and adjacent areas \parencite{ethnologue}.  While works like \textcite{Balaqayev54} provide decent syntactic overviews of the language, there is little to no work on the syntax of Kazakh within modern theoretical syntactic frameworks.  The authors are familiar with such work on related languages, especially Turkish (e.g., \cite{Kornfilt97} and \cite{GokselKerslake05}); while not directly consulted for this work, these works have contributed to our understanding of Kazakh syntax.

As an agglutinative language with rich morphology and agreement phenomena, Kazakh presents some interesting challenges for computational syntax.  These challenges include the syntactic functions of the various ``case'' morphemes, problems of ``zero derivation'', non-finite clauses, and copulas and copula constructions.  An existing morphological transducer of Kazakh \parencite{Washington14} implements analyses of how these various phenomena occur on the morphological level.  These phenomena will be described in this section, and how they were dealt with in the annotation of the treebank will be described in section~\ref{sec:annotation}. 

In Kazakh, as in most languages with case, there is not a one-to-one relation between ``case'' morphemes and syntactic function (not to mention a wide range of semantic functions).  The main syntactic functions of the traditionally defined cases in Kazakh are summarised in table~\ref{tab:cases}

\begin{table}[htbp]
	\centering
	\caption{Primary syntactic functions of traditionally defined cases in Kazakh.}\label{tab:cases}
	\begin{tabular}{l p{5em} p{8em} p{22em}}
		\toprule
			\textbf{Case} & \textbf{Morpheme} & \textbf{Functions} & \textbf{Examples} \\
		\midrule
			nominative & - & subject,\newline
									attributive,\newline
									indefinite object,\newline
									indefinite genitival 
								& hargle bargle \newline
									jargal barble \newline 
									\\
			accusative & -/NI/ & object,\newline
										embedded subject,\newline
										causative subject
									& foo \newline
										bar \\
			genitive & -/NIң/ & genitival,\newline
										embedded subject \\
			locative & -/DA/ & adverbial
									& \\
			ablative & -/DAн/ & adverbial,\newline
										comparator \\
			dative & -/GA/ & indirect object,\newline
									attributive
								& \\
		\bottomrule
	\end{tabular}
\end{table}

All of the traditionally defined ``cases'' in Kazakh have a variety of uses, .  The morphemes that mark the accusative and genitive cases are only used when the noun phrase is semantically definite (e.g., үйді \gloss{the house} \gmk{acc}, үйдің \gloss{of the house} \gmk{gen}).  This means that indefinite accusative and genitive as well as nominative noun phrases are all unmarked for case, and hence are all ambiguous (e.g., үй \gloss{house}).  The Kazakh transducer marks these all as \tag{nom}, but their use may be disambiguated.  Besides identifying the semantic role of an unmarked noun phrase, there are various agreement properties that make it clear what ``case'' the noun phrase ``truly'' receives: genitive noun phrases must have a corresponding noun phrase with possessive morphology that agrees in person with the genitive noun phrase; and truly nominative noun phrases must have a corresponding predicate that agrees in person with the nominative noun phrase---e.g., a verb or a copula.

Another category that may also be non-overt is the copula.  In the present tense, all but third person copulas have realised forms (e.g., Мен үйде\textbf{мін}. \gloss{I'm at home.}), but the third person copula does not surface (e.g., Ол үйде. \gloss{S/he's at home.}).

Another type of ``zero-derivation'' involves the use of adjectives and nouns.  In Kazakh, adjectives are all attributive, but many may also be used either nominally (i.e., as nouns) or adverbially (i.e., as adverbs), or both.  The morphological transducer internally provides a series of classes; depending on class membership, an adjective may receive the readings \tag{adj}\tag{advl} or \tag{adj}\tag{subst}, besides the default attributive reading of \tag{adj}.  If an adjective may be used substantivally, then it will receive a range of substantive morphology as well.  Similarly, many nouns may be used attributively, besides the default substantive reading.  Besides the nominative (and indefinite accusative/genitive) reading of \tag{nom}, the transducer provides an attributive reading for most nominals of \tag{nom}\tag{attr}.

How each of these issues bears on a dependency analysis will be discussed in section~\ref{sec:annotation}.

% case and function
%% indefinite acc/gen = "nom"
%% attr/subst/advl function of cases
% "zero" derivation
%% attr/subst/etc.
%% copulas?

\subsection{Treebanks} % Related work

A treebank is a parsed corpus of sentences annotated syntactically following a particular
syntactic theory. Two broad groups can be distinguished, phrase-structure treebanks
which annotate constituency strucure, and dependency treebanks which annotate dependency
structure. Some treebanks combine both.

Treebanks can be used directly for linguistic and computational linguistic research, by
performing search queries. For example, to extract a valency lexicon for verbs, or to
study the frequency of various syntactic phenomena such as word order or nominal case usage
and syntactic function.

They can also be used to train statistical parsers which can
be used to annotate previously unseen texts. These parsers can be used in end-user applications
such as machine translation and computer-aided language learning.

According to \textcite{nivre08}, a parser trained on a treebank of only 1,500 sentences 
can provide reasonable parsing accuracy.


\section{Methodology}\label{sec:method}

\subsection{Corpus}

\begin{table}[htbp]
  \centering
  \caption{Composition of the corpus. The corpus covers a range of genres and text types from free and public-domain sources.}
	\begin{small}
		\begin{tabular}{llrrr}
			\toprule
				\textbf{Document} & \textbf{Description} & \textbf{Sentences} & \textbf{Tokens} & \textbf{Avg. length}\\
			\midrule
				UN Declaration on Human Rights & Legal text on human rights & 25 & 409 & 16.3 \\
				Phrasebook                     & Phrases from Wikitravel   & 37 & 205 & 5.5 \\
				Жиырма Бесінші Сөз             & Philosophical text        & 34 & 525 & 15.4 \\ 
				Қожанасырдың тойға баруы       & Folk tale from Wikisource  & 8 & 140 & 17.5 \\
				Ер Төстік                      & Folk tale from Wikisource  & 23 & 203 & 8.8 \\
				Азамат қайда?                  & Story for language learners & 48 & 434 & 9.0 \\
				Футболдан әлем чемпионаты 2014 & Wikipedia article (2014 World Cup) & 14 & 246 & 17.5 \\
				Иран & Wikipedia article (Iran)                                & 111 & 1562 & 14.0 \\
				Радиан & Wikipedia article (Radian)                            & 2 & 17 & 8.5 \\
			\midrule
			\multicolumn{2}{c}{\tilda{}} & 302 & 3741 & 12.3 \\\cmidrule[\heavyrulewidth]{3-5}
		\end{tabular}
	\end{small}
\end{table}


\subsection{Preprocessing}

\begin{figure}
\begin{tiny}
\begin{alltt}
"<Елде>"
        "ел" n loc
"<ағылшын>"
        "ағылшын" n nom
"<үстемдігіне>"
        "үстемдік" n px3sp dat
"<қарсы>"
        "қарсы" post
"<жаңа>"
        "жаңа" adj
"<толқулар>"
        "толқу" n pl nom
"<басталды>"
        "баста" v tv pass ifi p3 pl
"<.>"
        "." sent
\end{alltt}
\end{tiny}
\caption{Example output after preprocessing and manual disambiguation. Each token is annotated for lemma,
   part of speech and morphosyntactic information.}
\end{figure}

Preprocessing the corpus consists of running the text through the Kazakh morphological
analyser \parencite{Washington14}, which also performs tokenisation of multiword units based 
the longest match left-to-right. Tokenisation for Kazakh is a non-trivial task, and so
we do not simply take space as a delimiter. The morphological analyser returns all 
the possible morphological analyses for each word based on a lexicon of around 20,000 lexemes.
After tokenisation and morphological analysis, the text is processed with a constraint-grammar 
based disambiguator for Kazakh consisting of 113 rules which remove inappropriate 
analyses in context. This reduces the average number of analyses per word from around 3.4
to around 1.7.

% tokenisation
% http://universaldependencies.github.io/docs/u/overview/tokenization.html

\subsubsection{Tokens and words}

Tokenisation of the corpus is performed by our morphological analyser. This analyser
performs tokenisation on the basis of a left-to-right longest match algorithm described
in \textcite{garrido02}. Simple tokens such as \emph{толқулар} `riots' are maintained
as a single token, and their lemma and morphological analysis is returned. Multiword
units such as \emph{ауа райы} `weather' and \emph{ата-ананың} `of parents' are combined
into a single token. Abbreviations and numerals which bear case, such as АҚШ-пен `with the USA'
and \emph{90\%-ына} `to the 90\%' are analysed as a single token, as are light-verbs such as
\emph{пайда бол} `to appear' and tense forms written with space like \emph{оқыған жоқ},
the third-person negative past of \emph{оқы} `to learn'.

In some cases a single token is split into two tokens, as with the aorist copula suffixes,
\emph{үйдемін} `I am in the house' is tokenised as үй.{\sc loc} + е.{\sc cop.aor.sg1}. Furthermore
two input tokens may result in three output tokens, \emph{бар ма?} `is there?' would be
tokenised as бар.{\sc adj} + е.{\sc cop.aor.sg1} + ма.{\sc qst}.


% productive suffixes -lı (cite wafflelazor)

\section{Annotation}\label{sec:annotation}

Annotation of the treebank followed an iterative approach. The first stage was to
select a text and preprocess it using the morphological analyser and constraint grammar.
The output of this process was then manually disambiguated by choosing the most appropriate
analysis in context, or adding a new analysis if one was not returned by the morphological
analyser. This latter step was mostly in the case of unknown proper nouns.\footnote{New lexemes
found during the development of the corpus will contribute to the development of the
morphological analyser.}

After the text was disambiguated, it was annotated by a single annotator for dependency
structure relying on the annotation guidelines for universal dependencies.\footnote{\url{http://universaldependencies.github.io/docs/u/overview/syntax.html}}

Where it was not clear how to annotate a particular structure, it was left pending, that
is marked with the {\sc dep} relation. After a text was annotated, graphical representations
of the trees were produced and proofread by a second annotator. The final analysis for a
sentence was decided on during discussions between the two annotators and in some cases
through the universal-dependency issue tracker.\footnote{\url{https://github.com/UniversalDependencies/docs/issues/}}

During the course of annotation, language-specific annotation guidelines for Kazakh are
being produced and published online.\footnote{\url{http://wiki.apertium.org/wiki/Dependency_parsing_for_Turkic}} In 
the remainder of this section we describe a number of Kazakh-specific decisions.

\begin{table}[htbp]
	\centering
	\begin{tabular}{ll}
		\toprule
		\textbf{Label} & \textbf{Description} \\
		\midrule
				\\
 		\bottomrule
	\end{tabular}
	\caption{Universal dependency label set}
\end{table}

\subsection{Copula}

\begin{figure}[htbp]
	\centering

	\begin{subfigure}[b]{0.3\textwidth}
		\centering
		\begin{dependency}[theme = simple, font = \small]
		   \begin{deptext}[column sep=0.3cm]
				Айгүл \& студент \& · \\
				\gmk{np} \& \gmk{n} \& \gmk{cop} \\
				\gloss{Aygül} \& \gloss{student} \& \gloss{is} \\
			\end{deptext}
			\depedge{2}{1}{\udlabel{subj}}
			\depedge{3}{2}{\udlabel{cop}}
		\end{dependency}
		\caption{xxx }\label{fig:compound}
	\end{subfigure}
	\quad
	\begin{subfigure}[b]{0.3\textwidth}
		\centering
		\begin{dependency}[theme = simple, font = \small]
		   \begin{deptext}[column sep=0.3cm]
				Айгүл \& студент \& - \\
				\gmk{np} \& \gmk{-} \& \gmk{-} \\
				\gloss{Aygül} \& \gloss{-} \& \gloss{-}\\
			\end{deptext}
			\depedge{3}{2}{\udlabel{cop}}
		\end{dependency}
		\caption{xxx.}\label{fig:cop2}
	\end{subfigure}
	\quad
	\begin{subfigure}[b]{0.3\textwidth}
		\centering
		\begin{dependency}[theme = simple, font = \small]
		   \begin{deptext}[column sep=0.3cm]
				Айгүл   \& - \& - \\
				\gmk{n} \& \gmk{-} \& \gmk{-} \& \\
				\gloss{Aygül} \& \gloss{-} \& \gloss{-} \\
			\end{deptext}
			\depedge{3}{2}{\udlabel{cop}}
		\end{dependency}
		\caption{xxx.}\label{fig:cop3}
	\end{subfigure}

	\caption{Dependency trees of copula constructions.}
\end{figure}




\subsection{Coordination}

\begin{figure}[htbp]
    \centering

	\begin{dependency}[theme = simple, font = \small]
	   \begin{deptext}[column sep=0.08cm]
%               1         2         3          4              5                6              7                8           9             10
             Барлық \& адамдар \& тумысынан \& азат      \& және         \& қадір-қасиеті \& мен          \& құқықтары  \& тең       \& болып     \& дүниеге \& келеді \\
                    \&         \&           \& \gmk{adj} \& \gmk{cnjcoo} \& \gmk{n}       \& \gmk{cnjcoo} \& \gmk{n.pl} \& \gmk{adj} \& \gmk{cop} \& \& \\
                    \&         \&           \& \gloss{free} \& \gloss{and}  \& \gloss{dignity}  \&  \gloss{and} \& \gloss{rights} \&  \gloss{equal} \& \gloss{being} \& \& \\
		\end{deptext}
		\depedge{9}{4}{\udlabel{conj}}
		\depedge{9}{5}{\udlabel{cc}}
		\depedge{8}{6}{\udlabel{conj}}
		\depedge{8}{7}{\udlabel{cc}}
		\depedge{9}{8}{\udlabel{subj}}
		\depedge{10}{9}{\udlabel{cop}}
		\depedge{12}{10}{\udlabel{advcl}}
	\end{dependency}
	\caption{Coordination}\label{fig:coord}
	
\end{figure}

\subsection{Complex nominals}

% compounds + izafet

There are different ways in which two nominals may occur together to act as a single nominal.  Compounds are formed by an attributive nominal (indistinguishable from the bare / nominative form, but tagged with \tag{attr}) preceding another nominal, as shown in \ref{fig:compound}.  An indefinite genitive construction is formed by an indefinite genitive nominal (indistinguishable from the bare / nominative form, and tagged with \tag{nom}) preceding a nominal that has third-person possessive morphology, as shown in \ref{fig:indefgen}.  A definite genetive construction is formed by a genitive-marked nominal (tagged \tag{gen}) precding a nominal that has third-person possessive morphology, as shown in \ref{fig:defgen}.

%One example of this is indefinite genitives.  In such constructions, a ``modifying'' nominal is morphologically not marked (which means it will be tagged as \tag{nom}, as discussed in §\ref{sec:kazakh}), though should be considered an indefinite genitive.  The modified element is a nominal with third-person possessive morphology, an example of which is provide in figure~\ref{fig:indefgen}.

\begin{figure}[htbp]
	\centering

	\begin{subfigure}[b]{0.3\textwidth}
		\centering
		\begin{dependency}[theme = simple, font = \small]
		   \begin{deptext}[column sep=0.3cm]
				көрші \& елдер \\
				\gmk{n.attr} \& \gmk{n.pl.nom} \\
				\gloss{neighbour} \& \gloss{countries} \\
			\end{deptext}
			\depedge{2}{1}{\udlabel{compound}}
		\end{dependency}
		\caption{Compounding: an attributive nominal depending on a nominal.}\label{fig:compound}
	\end{subfigure}
	\quad
	\begin{subfigure}[b]{0.3\textwidth}
		\centering
		\begin{dependency}[theme = simple, font = \small]
		   \begin{deptext}[column sep=0.3cm]
				әлем \& чемпионаты \\
				\gmk{n.nom} \& \gmk{n.px3sg.nom} \\
				\gloss{world} \& \gloss{championship} \\
			\end{deptext}
			\depedge{2}{1}{\udlabel{nmod}}
		\end{dependency}
		\caption{Indefinite genitive: An indefinite genitive depending on a nominal.}\label{fig:indefgen}
	\end{subfigure}
	\quad
	\begin{subfigure}[b]{0.3\textwidth}
		\centering
		\begin{dependency}[theme = simple, font = \small]
		   \begin{deptext}[column sep=0.3cm]
				Иранның \& экономиясы \\
				\gmk{n.gen} \& \gmk{n.px3sg.nom} \\
				\gloss{Iran's} \& \gloss{economy} \\
			\end{deptext}
			\depedge{2}{1}{\udlabel{nmod}}
		\end{dependency}
		\caption{Definite genitive: A definite genitive depending on a nominal.}\label{fig:defgen}
	\end{subfigure}

	\caption{Dependency trees of complex nominal relations.}
\end{figure}

As seen in the graphs, the compound relationship of an attributive nominal depending on another nominal is labelled \udtag{compound}, and genitive relations are considered \udtag{nmod}, regardless of whether there is a definite or indefinite genitive construction.


\subsection{Non-finite clauses}

% verbal adjectives → acl
% verbal nouns (gerunds) → ccomp, no xcomp
% verbal adverbs → advlc
% * gna_cnd? → advlc?

\section{Evaluation}\label{sec:eval}

In order to test the utility of the treebank in a real-world setting, we trained
and evaluated a number of models using the popular MaltParser tool \parencite{nivre07}.
MaltParser is a toolkit for data-driven dependency parsing, it can learn a parsing
model from treebank data and apply this model to parse unseen sentences. The parser
has a large number of options and parameters that need to be optimised. 
To select the best parser configuration we relied on 
MaltOptimiser \parencite{ballesteros15}. The optimiser was run separately for each of 
the model configurations.

As the treebank takes advantage of the new tokenisation standards in the CoNLL-U format,
and MaltParser only supports CoNLL-X, certain transformations were needed to perform 
the experiments. The corpus was flattened with conjoined tokens receiving a dummy 
surface form. The converted corpus is available alongside the original.\footnote{\url{removed for review}}

% system1:
% nivreeager
% LAS: 0.3341 [0.234 0.288 0.289 0.292 0.333 0.335 0.359 0.378 0.408 0.425]
% UAS: 0.5093 [0.413 0.464 0.465 0.466 0.474 0.499 0.531 0.545 0.604 0.632]

% system2:
% nivreeager
% LAS: 0.234+0.288+0.289+0.292+0.333+0.335+0.359+0.378+0.408+0.425
% UAS: 0.413+0.464+0.465+0.466+0.474+0.499+0.531+0.545+0.604+0.632

% system3:
% nivreeager
% LAS: 0.5529 [0.434 0.446 0.501 0.511 0.515 0.542 0.558 0.572 0.7 0.75]
% UAS: 0.7376 [0.658 0.669 0.675 0.684 0.694 0.723 0.751 0.776 0.825 0.921]

% system4:
% covnonproj
% LAS: 0.6092 [0.519 0.535 0.544 0.569 0.569 0.62 0.621 0.649 0.683 0.783]
% UAS: 0.7575 [0.674 0.686 0.69 0.714 0.727 0.746 0.795 0.803 0.812 0.928]

%             [0.434+0.446+0.501+0.511+0.515+0.542+0.558+0.572+0.7+0.75]
%             [0.658+0.669+0.675+0.684+0.694+0.723+0.751+0.776+0.825+0.921]

\begin{table}[htbp]
	\centering
	\begin{tabular}{llrr}
		\toprule
			\textbf{Features}       & \textbf{Algorithm} &\textbf{LAS} [range] & \textbf{UAS} [range] \\
		\midrule
			surface                & nivreeager  & 33.4 [23.4, 42.5] & 50.9 [41.3, 63.2] \\
			surface+lemma          & nivreeager  & 33.4 [23.4, 42.5] & 50.9 [41.3, 63.2] \\
			surface+lemma+POS      & nivreeager  & 55.2 [43.4, 75.0] & 73.7 [65.8, 92.1] \\
			surface+lemma+POS+MSD  & nivreeager  & 55.2 [43.4, 75.0] & 73.7 [65.8, 92.1] \\
		\bottomrule
	\end{tabular}
	\caption{Preliminary parsing results from MaltParser using different models. The numbers in brackets denote the upper and lower bounds found during cross validation. Adding structural features to the model substantially improves the performance of the parser, although we find no improvement in using full morphosyntactic description (MSD) over using simply the first part-of-speech (POS) tag.}
	\label{table:eval}
\end{table}

To perform 10-fold cross validation we randomised the order of sentences in the corpus
and split it into 10 equally-sized parts. In each iteration we held out one part for testing and used
the rest for training. We calculated the labelled-attachment score (LAS) and 
unlabelled-attachment score (UAS) for each of the models.

The results we obtain are similar to those obtained with similar 
sized treebanks, for example the Irish treebank of \textcite{Lynn12}, who report an LAS of
63.3 and a UAS of 73.3 with the best model.


\section{Future work}\label{sec:future}

%stabilise guidelines

%crosslingual parsing: Tatar, Kumyk, Kyrgyz, Tuvan

%language-specific dependency relations --- (acl:relcl, ...)

\section{Conclusions}\label{sec:conclusions}

\section*{Acknowledgements}

%Tolgonay?, Aida?

%\bibliographystyle{apalike}
%\bibliography{paper}
\printbibliography

\end{document}
