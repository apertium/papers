\documentclass[a4paper,11pt, onecolumn,twoside]{article}

\usepackage{turklang}


%%%%% Language and font stuff %%%%%

\usepackage{polyglossia}
\setdefaultlanguage[variant=australian]{english}

\usepackage{fontspec}
\usepackage{xunicode}

%\usepackage{times}
\setmainfont{Times New Roman}
\newfontfamily\udfont[Scale=1.3,Letters=SmallCaps]{Linux Libertine O}


%%%%% General formatting stuff %%%%%

\usepackage{multirow}
\usepackage[small,bf]{caption} % CHECK IF THIS IS OK
\usepackage[colorlinks=true,citecolor=black,linkcolor=black,urlcolor=blue]{hyperref}

\usepackage{subcaption}

\usepackage{natbib}
\bibdata{paper}


%%%%% Stuff for this paper %%%%%
\usepackage{tikz-dependency}
\newcommand{\gmk}[1]{\texttt{#1}}
\newcommand{\gloss}[1]{#1}
\newcommand{\tag}[1]{\texttt{‹#1›}}
\newcommand{\udtag}[1]{\texttt{@#1}}
\newcommand{\udlabel}[1]{{\udfont #1}}
%\newcommand{\corrauthor}[1]{\footnotemark\footnotetext{hargle}}


%%%%% Paper metadata %%%%%

\title{Towards a free/open-source universal-dependency treebank for Kazakh}
\name{Francis M. Tyers$^a$ and Jonathan Washington$^b$}
\shortauthor{Tyers \& Washington}
\address{$^a$ HSL-fakultehta, UiT Norgga árktalaš universitehta, N-9015 Tromsø, Norway\\
	$^b$ Departments of Linguistics and Central Eurasian Studies, Indiana University, Bloomington, IN 47405, USA\footnote{Corresponding author.  E-mail address: jonwashi@indiana.edu}}

\keywords{Kazakh; treebank; dependency grammar; universal dependency}

%250-300-word ``extended abstract in English''
\abstract{
This article describes the first steps towards a free/open-source dependency treebank for Kazakh based on universal dependency (UD) annotation standards. The treebank contains 302 sentences and is based on texts from a range of open-source and public domain sources. This ensures its free availability and extensibility. Texts in the treebank are first morphologically analysed and disambiguated and then annotated manually for dependency structure. 
}


%%%%% End preamble %%%%%

\begin{document}

%\thanks{hargle}
\maketitleabstract{}
\thispagestyle{firststyle}

\section{Introduction}

\citet{Lynn12}, \citet{Atalay03}, \citet{Oflazer03}, \citet{DeMarneffe14}

\section{Background}

\subsection{Kazakh}\label{sec:kazakh}

Kazakh (қазақ тілі), a Turkic language of Central Asia with around 12 million speakers, hargle bargle. %FIXME

In the process of designing a Kazakh transducer \citep{Washington14}, several decisions were made about how various inflectional morphemes affect the part of speech and syntactic function of various word forms.  These decisions include the relations to other words defined by ``case'' morphemes, and processes of ``zero derivation''.

All of the traditionally defined ``cases'' in Kazakh have a variety of uses.  The morphemes that mark the accusative and genitive cases are only used when the noun phrase is semantically definite.  This means that indefinite accusative and genitive as well as nominative noun phrases are all unmarked for case, and hence are all ambiguous.  Our Kazakh transducer marks these all as \tag{nom}, but their use may be disambiguated.  Besides identifying the semantic role of an unmarked noun phrase, there are various agreement properties that make it clear what ``case'' the noun phrase ``truly'' receives: genitive noun phrases must have a corresponding noun phrase with possessive morphology that agrees in person with the genitive noun phrase; and nominative noun phrases must have a corresponding predicate that agrees in person with the nominative noun phrase, e.g., a verb or a copula.

Copulas may also be non-overt.  In the present tense, all but third person copulas have realised forms, but the third person copula does not surface.

Adjectives in Kazakh are by default attributive, but many may also be used either nominally (i.e., as nouns) or adverbially (i.e., as adverbs), or both.  Our morphological transducer internally provides a series of classes; depending on class membership, an adjective may receive readings \tag{adj}\tag{advl} or \tag{adj}\tag{subst}, besides the default attributive reading of \tag{adj}.  If an adjective may be used substantivally, then it will receive a range of substantive morphology as well.

% case and function
%% indefinite acc/gen = "nom"
%% attr/subst/advl function of cases
% "zero" derivation
%% attr/subst/etc.
%% copulas?

\subsection{Treebanks} % Related work

% what are treebanks used for ?

\section{Methodology}

\subsection{Corpus}

\begin{table}[htbp]
  \centering
\begin{small}
  \begin{tabular}{|l|l|r|r|r|}
    \hline
    \textbf{Document} & \textbf{Description} & \textbf{Sentences} & \textbf{Tokens} & \textbf{Avg. length}\\
    \hline
    UN Declaration on Human Rights & Legal text on human rights & 25 & 409 & 16.3 \\
    Phrasebook                     & Phrases from Wikitravel   & 37 & 205 & 5.5 \\
    Жиырма Бесінші Сөз             & Philosophical text        & 34 & 525 & 15.4 \\ 
    Қожанасырдың тойға баруы       & Folk tale from Wikisource  & 8 & 140 & 17.5 \\
    Ер Төстік                      & Folk tale from Wikisource  & 23 & 203 & 8.8 \\
    Азамат қайда?                  & Story for language learners & 48 & 434 & 9.0 \\
    Футболдан әлем чемпионаты 2014 & Wikipedia article (2014 World Cup) & 14 & 246 & 17.5 \\
    Иран & Wikipedia article (Iran)                                & 111 & 1562 & 14.0 \\
    Радиан & Wikipedia article (Radian)                            & 2 & 17 & 8.5 \\
    \hline
    \multicolumn{2}{c|}{~} & 302 & 3741 & 12.3 \\\cline{3-5}
  \end{tabular}
\end{small}
  \caption{Composition of the corpus. The corpus covers a range of genres and text types from 
    free and public-domain sources.}
\end{table}


\subsection{Preprocessing}

Preprocessing the corpus consists of running the text through the Kazakh morphological
analyser \citep{Washington14}, which also performs tokenisation of multiword units based 
the longest match left-to-right. Tokenisation for Kazakh is a non-trivial task, and so
we do not simply take space as a delimiter. The morphological analyser returns all 
the possible morphological analyses for each word based on a lexicon of around 20,000 lexemes.
After tokenisation and morphological analysis, the text is processed with a constraint-grammar 
based disambiguator for Kazakh consisting of 113 rules which remove inappropriate 
analyses in context. This reduces the average number of analyses per word from around 3.4
to around 1.7.

% tokenisation
% http://universaldependencies.github.io/docs/u/overview/tokenization.html

\subsubsection{Tokens and words}

% simple tokens
% multiword units
% copula
% clitics
% productive suffixes -lı (cite wafflelazor)

\section{Annotation guidelines}

% general procedure

% - disambiguate
% - annotate
% - check the tree structure
% - fix errors
% - check for label consistency
% - fix errors

\begin{table}[htbp]
  \centering
  \begin{tabular}{|l|l|}
    \hline
    \textbf{Label} & \textbf{Description} \\

    \hline
  \end{tabular}
  \caption{Universal dependency label set}
\end{table}

\subsection{Copula}

\subsection{Coordination}

\subsection{Complex nominals}

% compounds + izafet

There are different ways in which two nominals may occur together to act as a single nominal.  Compounds are formed by an attributive nominal (indistinguishable from the bare / nominative form, but tagged with \tag{attr}) preceding another nominal, as shown in \ref{fig:compound}.  An indefinite genitive construction is formed by an indefinite genitive nominal (indistinguishable from the bare / nominative form, and tagged with \tag{nom}) preceding a nominal that has third-person possessive morphology, as shown in \ref{fig:indefgen}.  A definite genetive construction is formed by a genitive-marked nominal (tagged \tag{gen}) precding a nominal that has third-person possessive morphology, as shown in \ref{fig:defgen}.

%One example of this is indefinite genitives.  In such constructions, a ``modifying'' nominal is morphologically not marked (which means it will be tagged as \tag{nom}, as discussed in §\ref{sec:kazakh}), though should be considered an indefinite genitive.  The modified element is a nominal with third-person possessive morphology, an example of which is provide in figure~\ref{fig:indefgen}.

\begin{figure}[htbp]
	\centering

	\begin{subfigure}[b]{0.3\textwidth}
		\centering
		\begin{dependency}[theme = simple, font = \small]
		   \begin{deptext}[column sep=0.3cm]
				көрші \& елдер \\
				\gmk{n.attr} \& \gmk{n.pl.nom} \\
				\gloss{neighbour} \& \gloss{countries} \\
			\end{deptext}
			\depedge{2}{1}{\udlabel{compound}}
		\end{dependency}
		\caption{Compounding: an attributive nominal depending on a nominal.}\label{fig:compound}
	\end{subfigure}
	\quad
	\begin{subfigure}[b]{0.3\textwidth}
		\centering
		\begin{dependency}[theme = simple, font = \small]
		   \begin{deptext}[column sep=0.3cm]
				әлем \& чемпионаты \\
				\gmk{n.nom} \& \gmk{n.px3sg.nom} \\
				\gloss{world} \& \gloss{championship} \\
			\end{deptext}
			\depedge{2}{1}{\udlabel{nmod}}
		\end{dependency}
		\caption{Indefinite genitive: An indefinite genitive depending on a nominal.}\label{fig:indefgen}
	\end{subfigure}
	\quad
	\begin{subfigure}[b]{0.3\textwidth}
		\centering
		\begin{dependency}[theme = simple, font = \small]
		   \begin{deptext}[column sep=0.3cm]
				Иранның \& экономиясы \\
				\gmk{n.gen} \& \gmk{n.px3sg.nom} \\
				\gloss{Iran's} \& \gloss{economy} \\
			\end{deptext}
			\depedge{2}{1}{\udlabel{nmod}}
		\end{dependency}
		\caption{Definite genitive: A definite genitive depending on a nominal.}\label{fig:defgen}
	\end{subfigure}

	\caption{Dependency trees of complex nominal relations.}
\end{figure}

As seen in the graphs, the compound relationship of an attributive nominal depending on another nominal is labelled \udlabel{compound}, and genitive relations are considered \udtag{nmod}, regardless of whether there is a definite or indefinite genitive construction.


\subsection{Non-finite clauses}

\section{Evaluation}


In order to test the utility of the treebank in a real-world setting, we trained
and evaluated a number of models using the popular MaltParser tool \citep{nivre07}.
MaltParser is a toolkit for data-driven dependency parsing, it can learn a parsing
model from treebank data and apply this model to parse unseen sentences. The parser
has a large number of options and parameters that need to be optimised.
To select the best parser configuration we relied on
MaltOptimiser \citep{ballesteros15}. The optimiser recommended the \emph{covnonproj}
parsing algorithm.

As the treebank takes advantage of the new tokenisation standards in the CoNLL-U format,
and MaltParser only supports CoNLL-X, certain transformations were needed to perform
the experiments. The corpus was flattened with conjoined tokens receiving a dummy
surface form. The converted corpus is available alongside the original.\footnote{\url{removed for review}}

% LAS: 0.5529 [0.434 0.446 0.501 0.511 0.515 0.542 0.558 0.572 0.7 0.75]
% UAS: 0.7376 [0.658 0.669 0.675 0.684 0.694 0.723 0.751 0.776 0.825 0.921]

\begin{table}
  \centering
  \begin{tabular}{|l|r|r|}
    \hline
    \textbf{Features} & \textbf{LAS} & \textbf{UAS} \\
    \hline
     - & - & - \\
    \hline
  \end{tabular}
  \caption{Preliminary parsing results from MaltParser using different models}
  \label{table:eval}
\end{table}

To perform 10-fold cross validation we randomised the order of sentences in the corpus
and split it into 10 parts. In each iteration we held out one part for testing and used
the rest for training. We calculated the labelled-attachment score (LAS) and
unlabelled-attachment score (UAS) for each of the models.

The results we obtain for unlabelled attachment are similar to those obtained with similar 
sized treebanks, for example the Irish treebank of \cite{Lynn12}, who report an LAS of
63.3 and a UAS of 73.3 with the best model.


\section{Future work}

%stabilise guidelines

%crosslingual parsing: Tatar, Kumyk, Kyrgyz, Tuvan

%language-specific dependency relations --- (acl:relcl, ...)

\section{Conclusions}

\section*{Acknowledgements}

%Tolgonay?, Aida?

\bibliographystyle{apalike}
\bibliography{paper}

\end{document}
