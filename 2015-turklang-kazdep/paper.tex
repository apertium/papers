\documentclass[a4paper,11pt, onecolumn]{article}

\usepackage{turklang}

\usepackage{polyglossia}
\setdefaultlanguage[variant=australian]{english}

\usepackage{fontspec}
\usepackage{xunicode}

\usepackage{multirow}
\usepackage[colorlinks=true,citecolor=black,linkcolor=black,urlcolor=blue]{hyperref}

\usepackage{natbib}
\bibdata{paper}

%\usepackage{times}
\setmainfont{Times New Roman}

\title{Towards a free/open-source universal-dependency treebank for Kazakh}
\name{Francis M. Tyers$^a$ and Jonathan Washington$^b$}
\address{$^a$ HSL-fakultehta, UiT Norgga árktalaš universitehta, N-9015 Tromsø, Norway\\
	$^b$ Departments of Linguistics and Central Eurasian Studies, Indiana University, Bloomington, IN 47405, USA\footnote{Corresponding author.  E-mail address: jonwashi@indiana.edu}}

\keywords{Kazakh; treebank; dependency grammar; universal dependency}

\abstract{250-300-word ``extended abstract in English''}

\begin{document}

\maketitleabstract{}


\section{Introduction}

\citet{Lynn12}, \citet{Atalay03}, \citet{Oflazer03}, \citet{DeMarneffe14}

\section{Background}

\subsection{Kazakh}

\subsection{Treebanks} % Related work

\section{Methodology}

\subsection{Corpus}

\begin{table}
  \centering
  \begin{tabular}{|l|l|r|r|r|}
    \hline
    \textbf{Document} & \textbf{Description} & \textbf{Sentences} & \textbf{Tokens} & \textbf{Avg. length}\\
    \hline
    UDHR & Legal text on human rights & - & - & - \\
    Phrasebook & Phrases from Wikitravel & 37 & 205 & 5.5 \\
    Жиырма Бесінші Сөз & Philosophical text & 34 & 525 & 15.4 \\ 
    Өлген қазан & Folk tale & 8 & 140 & 17.5 \\
    Азамат къайда? & Story for children and language learners & 48 & 434 & 9.0 \\
    Футболдан әлем чемпионаты 2014  & Wikipedia article (2014 World Cup) & 14 & 246 & 17.5 \\
    Иран & Wikipedia article (Iran) & - & - & - \\
    Радиан & Wikipedia article (Radian) & - & - & - \\
    \hline
  \end{tabular}
  \caption{Composition of the corpus. The corpus covers a range of genres and text types.}
\end{table}


\subsection{Preprocessing}

% tokenisation
\cite{Washington14}

\subsection{Annotation guidelines}

\begin{table}
  \centering
  \begin{tabular}{|l|l|}
    \hline
    \textbf{Label} & \textbf{Description} \\

    \hline
  \end{tabular}
  \caption{Universal dependency label set}
\end{table}

\section{Evaluation}

\section{Future work}

%crosslingual parsing: Tatar, Kumyk, Kyrgyz, Tuvan

\section{Conclusions}

\section*{Acknowledgements}

\bibliographystyle{apa}
\bibliography{paper}

\end{document}
