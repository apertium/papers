\documentclass[a4paper,twocolumn,11pt]{article}

\usepackage{url}

\newcommand{\com}[1]{\marginpar{\scriptsize #1}} 

\title{A free/open-source machine translation system for English to Kazakh}

% Authors:
% Aida, Mikel, Fran, Inari, ... ?

\begin{document}

\section{Introduction}

SOMETHING ABOUT ENGLISH-to-KAZAKH trnslation problem??

This paper describes work in progress of development of machine translation system for English-to-Kazakh, which developed by using Apertium free/open-source machine translation platform (Forcada et al. 2011, http://www.apertium.org). 

\section{The Apertium platform}
Apertium (Forcada et al. 2011, http://www.apertium.org) is a free/open-source rule-based machine translation (MT) platform that was launched in 2005 by the Universitat d’Alacant. Though it was initially aimed at translating between closely related languages, it was later extended to be able to deal with unrelated languages. All of the components of the platform (MT engine, developer’s tools, and linguistic data for an increasing number of language pairs) are licensed under the free/open-source GNU General Public License (GPL, versions 2 and 3) and are available to everyone interested in the website.
FIGURE construction of Apertium
WILL EDIT THIS
	Deformatter. It separates the text to be translated from the formatting tags. Formatting tags are encapsulated as “superblanks” that are placed between words in such a way that the remaining modules see them as regular blanks.
	Morphological analyser. For each surface form (that is, for each lexical unit as it appears in the text), the morphological analyser generates one or more lexical forms composed of: lemma (dictionary or citation form), lexical category (or part-of-speech), and inflection information. The morphological analyser executes a finite-state transducer generated by compiling a morphological dictionary for the source language. Lexical units containing more than one word (multiword lexical units) are analyzed as a single lexical unit .   Morphological analyser uses a finite state transducer based on two-level rules (in the case of Kazakh, apertium-kaz.kaz.lexc, apertium-kaz.kaz.twol). This module therefore separates lexemes and processes morphological analysis, and then returns possible lexical forms.
	Part-of-speech (POS) tagger. Apertium's POS tagger is based on a statistical model based on hidden Markov models which processes the result of the application of  on constraint-grammar rules (Karlsson 2005), which areused to discard some analyses  using simple rules ( written in apertium-kaz.kaz.rlx) based on context. For example, consider the morphological analysis of word қара:
^қара/қара<adj>/қара<adj><advl>/қара<adj><subst><nom>/қара<v><tv><imp><p2><sg>/қара<adj>+е<cop><p3><pl>/қара<adj>+е<cop><p3><sg>/қара<adj><subst><nom>+е<cop><p3><pl>/қара<adj><subst><nom>+е<cop><p3><sg>
This word is ambiguous and has 6 meanings. Many  surface forms are ambiguous, which means that these words have more than one POS and therefore more than one possible translation.   After  this module, all words have only one morphological analysis.
	Lexical transfer. This module uses a bilingual dictionary (apertium-eng-kaz.eng-kaz.dix) which has very simple structure [7]. The module reads each source-language lexical form and finds one or more corresponding target-language lexical forms.  Multiword units are translated as a single word.
	Lexical selection. It uses rules that select for those lexical words having many translations,  one of the translations in the target language according to context. All rules are written in file apertium-eng-kaz.kaz-eng.lrx .
	Structural transfer. This module identifies sequences of lexical forms  (phrases or segments), which need syntactical processing (handling of number, prepositions, etc.) to be translated. It uses  files with rules, which specify  the syntactic transformation as a cascaded process. Transfer rules, which transform lexical-form sequences into a new sequences for the  target language, perform the work in this module.  
  
Figure-1. The Apertium machine translation pipeline 
	 Morphological generator. From the sequence of target-language lexical forms produced by the structural transfer, it generates a corresponding sequence of target language surface forms. The morphological generator executes a finite-state transducer generated by compiling a morphological dictionary for the target language.     
	Post-generator. It takes care of some minor orthographical operations in the target language (for instance, it generates the English form cannot from can and not). This module is generated from file with rules which are very   similar in format to dictionary files. 
	Reformatter. It places format tags back into the text so that its format is preserved.



\section{Linguistic Data(dictionaries,rules for eng-kaz)}

In Apertium is used XML-based formats for linguistic data, it includes bilingual and monolingual dictionaries, structural transfer, lexical rules and rules for part-of-speech tagging.

Dictionaries


\section{Results}

\section{Evaluation}

\com{task-based evaluation (WER, assim, TM-patching?) -FMT}

\com{Compare with GT + other systems -FMT}

\section*{Acknowledgements}

\end{document}
