\documentclass[a4paper,twocolumn,11pt]{article}

\usepackage{url}

\newcommand{\com}[1]{\marginpar{\scriptsize #1}} 

\title{A free/open-source machine translation system for English to Kazakh}

% Authors:
% Aida, Mikel, Fran, Inari, ... ?

\begin{document}

\section{Introduction}

SOMETHING ABOUT ENGLISH-to-KAZAKH trnslation problem??

This paper describes work in progress of development of machine translation system for English-to-Kazakh, which developed by using Apertium free/open-source machine translation platform (Forcada et al. 2011, http://www.apertium.org). 

\section{The Apertium platform}
Apertium (Forcada et al. 2011, http://www.apertium.org) is a free/open-source rule-based machine translation (MT) platform that was built in 2005 by the Universitat d’Alacant. At first, it was initially aimed to translate texts between closely related languages, then it was extended to deal with unrelated languages. This platform has next components: machine translation engine, developer’s tools, and linguistic data for an increasing number of language pairs and they are licensed under the free/open-source GNU General Public License (GPL, versions 2 and 3).

FIGURE construction of Apertium
WILL EDIT THIS
	De-formatter. Separates the text to be translated from the formatting tags.  Formatting tags are encapsulated in brackets so they are treated as “superblanks” that are placed between words in such a way that the remaining modules see them as regular blanks.  
	Morphological analyser. For each surface form the morphological analyser generates one or more lexical forms, which consist of: lemma (dictionary or citation form), lexical category (or part-of-speech), and inflection information. 
For example, the morphological analyser will deliver next sentence "I played" as showed below:
^I/I<prn><subj><p1><mf><sg>$ 
^played/play<vblex><past>/play<vblex><pp>$

In this example each source form has been analysed as one or more lexical forms: "I" is analysed into "I", where lexical category is subject pronoun(<prn><subj>) and first person(p1), could be masculine or feminine(mf), singular(sg); "played" is analysed into lexical verb(vblex) "play" with past simple tense(past) or it could be past participle(pp), so it has two analysis. Each analysis of word are separated by "^" and "" symbols, and for one word each lexical forms are delimited by "/", and tags(<...>) show grammatical attributes of lexical form. 
 For Kazakh language finite state transducer based on two-level rules (in the case of Kazakh, apertium-kaz.kaz.lexc, apertium-kaz.kaz.twol). This module therefore separates lexemes and processes morphological analysis, and then returns possible lexical forms.
	Part-of-speech (POS) tagger. This module chooses one of lexical forms of ambiguous word. As can be seen from the previous example, the morphological analyser could deliver more than one lexical form and choosing wrong one could produce errors in translation. POS tagger is based on a statistical model, which based on hidden Markov models and it has been trained on source language texts, which processes the result of the application of  on constraint-grammar rules (Karlsson 2005), which areused to discard some analyses  using simple rules (written in apertium-eng-kaz.eng-kaz.rlx) based on context. For example, consider the morphological analysis of word "book":
^book/book<n><sg>
/book<vblex><inf>
/book<vblex><pres>
This word has 3 lexical forms, and depends on context, ont of them will be chosen by rules or tagger. After  this module, all words have only one morphological analysis.
	Lexical transfer. It reads each lexical form of source language and delivers a corresponding target language lexical form. This module uses a bilingual dictionary (apertium-eng-kaz.eng-kaz.dix) which has very simple structure(1). For English-Kazakh language pair, each lexical form of English word is translated into Kazakh as follows:
^I<prn><subj><p1><mf><sg> /мен<prn><pers><subj><p1><mf><sg>$ 
^play<vblex><past>/ойна<v><tv><past>$
 Multiword units are translated as a single word.
	Lexical selection. It uses rules for lexical words, which have many translations, to select one of the translations in the target language according to context. All rules are written in file apertium-eng-kaz.eng-kaz.lrx.
	Structural transfer. This module uses pattern matching to identify sequences of lexical forms  (phrases or segments), which need syntactical processing to translate grammatical differences between two languages (handling of number, gender, etc.). It uses  files with rules, which specify  the syntactic transformation such as word reorderings, lexical changes such as changes in prepositions and agreement between target language lexical forms. Transfer rules can produce new sequences for the  target language, for instance, preposition-noun rule is used to built sequence, where for noun is chosen right case, which depends on preposition: 
in garden  -  ^бақша<n><sg><PXD><loc
  
	 Morphological generator. Delivers the sequence of target-language lexical forms, produced by the structural transfer, to a corresponding sequence of target language surface forms. The morphological generator executes a finite-state transducer generated by compiling a morphological dictionary for the target language.     
	Post-generator. It performs some minor orthographical operations in the target language (for instance, it generates the English form cannot from can and not). This module is generated from file with rules which are very   similar in format to dictionary files. 
	Reformatter. It places format tags back into the text so that its format is preserved.



\section{Linguistic Data(dictionaries,rules for eng-kaz)}

In Apertium is used XML-based formats for linguistic data, it includes bilingual and monolingual dictionaries, structural transfer, lexical rules and rules for part-of-speech tagging.

Dictionaries


\section{Results}

\section{Evaluation}

\com{task-based evaluation (WER, assim, TM-patching?) -FMT}

\com{Compare with GT + other systems -FMT}

\section*{Acknowledgements}

\section{References}
1 Сундетова А.М., Кәрібаева А.С., АПЕРТИУМ ПЛАТФОРМАСЫНДАҒЫ АҒЫЛШЫН-ҚАЗАҚ МАШИНАЛЫҚ АУДАРМАШЫ ҮШІН ЕКІТІЛДІ СӨЗДІКТІ ҚҰРУ. Матер. межд. научн.-практ. конф. «Применение информационно-коммуникационных технологий в образовании и науке», посвященной 50-летию Департамента информационно-коммуникационных технологий и 40-летию кафедры «Информационные системы» КазНУ им. аль-Фараби. 22 ноября 2013г. – Алматы: Қазақ Университеті, 2013. – С.53-57.

\end{document}
