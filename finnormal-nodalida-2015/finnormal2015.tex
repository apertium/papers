%
% File nodalida2015.tex
%
% Contact beata.megyesi@lingfil.uu.se
%
% Based on the instruction file for EACL 2014
% which in turn was based on the instruction files for previous 
% ACL and EACL conferences.

\documentclass[11pt]{article}
\usepackage{nodalida2015}
\usepackage{expex}
%\usepackage{times}
\usepackage{mathptmx}
%\usepackage{txfonts}
\usepackage{url}
\usepackage{multirow}
\usepackage{latexsym}
\special{papersize=210mm,297mm} % to avoid having to use "-t a4" with dvips 
%\setlength\titlebox{6.5cm}  % You can expand the title box if you really have to

\title{Automatic conversion of colloquial Finnish to standard Finnish}

\author{}
%
%Francis M. Tyers\\
%  UiT Norgga \'{a}rktala\v{s} universitehta, \\
%  N-9015 Norway \\
%  {\tt francis.tyers@uit.no} \\\And
%  Inari Listenmaa\\
%  Chalmers tekniska h\"{o}gskola, \\
%  Sweden \\
%  {\tt inari@chalmers.se} \\}
%
\date{}

\begin{document}
\maketitle
\begin{abstract}
  This paper presents an unsupervised method for converting between colloquial Finnish
  and standard Finnish. The method relies upon a small number of orthographical rules
  combined with a large language model of standard Finnish for ranking the possible 
  conversions. Aside from this contribution, the paper also presents an evaluation
  corpus consisting of aligned sentences in colloquial Finnish, orthographically-standardised
  colloquial Finnish and standard Finnish. The methods we present outperforms the baseline
  of simply treating colloquial Finnish as standard Finnish and offers promise for the adaptation
  of language-technology tools created for standard Finnish to colloquial Finnish. To this end
  the paper also presents preliminary results which show promise for using normalisation in 
  the machine translation task.
\end{abstract}

\section{Introduction}

Most language technology tools are designed or trained based on standard language 
forms, where they exist. The application of these tools to non-standard
language can cause a substantial decrease in quality. For example in machine translation,
parsing and part-of-speech tagging \cite{eisenstein2013}.

For language-technology researchers working on non-standard forms of language, there are 
two clear options: either create new tools to process non-standard text,
  or create tools to preprocess non-standard text, standardising it to be subsequently processed
 by existing tools.

This paper evaluates a number of methods for converting colloquial Finnish to standard Finnish and describes a parallel corpus for evaluation.

\section{Related work}

There are a number of areas of research related to the task of text normalisation. Text
proofing tools, such as spelling and grammar checkers CITEXXX can be used to encourage adherence
to particular orthographic or grammatical norms. Accent and diacritic restoration --- for example in \newcite{scannell2011} --- is similar
in that it aims to bring text closer to standard orthography in order to facilitate treatment by 
automatic tools. Another related area is machine translation between different written norms of 
the same language. For example between Norwegian Bokm\aa{}l and Norwegian Nynorsk \cite{unhammer2009}.

\newcite{scannell2014} presents a method for normalising pre-standardised text in Irish to 
the modern standard. The method relies on a translation model consisting of word-to-word 
correspondences in addition to spelling rules. Each word-to-word mapping has the same
conditional probability and a penalty is assigned to each spelling rule application. Decoding
works by processing the source sentence word-for-word left-to-right, keeping track of the 
possible `hypothesis' translations and their probabilities, and when the end of 
sentence is reached, the most probable is output.

\subsection{Colloquial Finnish}


\newcite{viinikka2013} describe the common meaning of the terms colloquial (\emph{puhekieli}) and standard (\emph{yleiskieli} or \emph{kirjakieli}) Finnish: standard language is regulated by kielenhuolto and is unified in morphology and vocabulary; colloquial language shows local and idiolectal variation, and has structures that are characteristic to spoken variety,  such as discourse particles and incomplete clauses.



Colloquial Finnish differs from standard Finnish in some key features: 

Morphology shows dialectal variation, shortening and merging of morphemes, e.g. \emph{en min\"{a}} `not I' $\rightarrow$ \emph{emm\"{a}} (`I don't').

Syntax: due to the spontaneous nature of colloquial language, syntactic structures are common for spoken language. 
Prototypically, colloquial language is spoken and spontaneous, and standard is written, planned and structured. We illustrate the differences with the following example from our data set. Sentence 1 is the original colloquial version. The gloss shows the actual word-by-word translation, and the translation shows similar style and register in English.

\ex
\begingl
\gla seiskakin oli vaan silleen et fonotaksista p\"{a}\"{a}ttelin //
\glb seven-{\sc also} was just like.that that phonotactics.{\sc ela} I.deduced //
\glft `also the seventh, it was like, I just deduced it from phonotactics' //
\endgl
\xe


For the partial version, we did not change the syntactic structure, but changed only morphology and vocabulary. On the lexical level, the word \emph{seiska} `number 7' is colloquial style, and in the standard translation it is replaced by the ordinal \emph{seitsem\"{a}s} `seventh'.  Other changes in the partial version are common morphological or phonological, phenomena, such as reducing the diphthong in \emph{vaan} $\rightarrow$ \emph{vain} or shortening in  \emph{et} $\rightarrow$ \emph{että}. The original sentence and the partial translation are shown below, aligned word by word.

\ex
\begingl
\gla seiskakin oli vaan silleen et fonotaksista p\"{a}\"{a}ttelin //
\glb seitsem\"{a}skin oli vain sill\"{a}\#lailla ett\"{a} fonotaksista p\"{a}\"{a}ttelin //
%\glb seventh-{\sc also} was just like.that that phonotactics.{\sc ela} I.deduced //
% \glft `also the seventh, it was like, I just deduced it from phonotactics' //
\endgl
\xe

 The syntactic structure of the original sentence is markedly spoken; the word \emph{seiska} is topicalised, and the main information ``deduced from phonotactics'' is in a subordinate clause. 
The translation into  is shorter and more precise, leaving just the main information.

\ex
\begingl
\gla p\"{a}\"{a}ttelin seitsem\"{a}nnenkin fonotaksista //
\glb I.deduced seventh-{\sc also} phonotactics.{\sc ela} //
\glft `I deduced also the seventh from phonotactics' //
\endgl
\xe




% ``Yleens\"{a} kirjakieli ymm\"{a}rret\"{a}\"{a}n samaksi kuin yleiskieli: kielenhuollon suositusten mukaiseksi standardoiduksi kieleksi, jonka sanasto on yleistajuista. Puhekieli puolestaan miellet\"{a}\"{a}n kirjakielt\"{a} arkisemmaksi ja tyylilt\"{a}\"{a}n vapaamuotoisemmaksi kielimuodoksi, jonka ei tarvitse noudattaa kirjakielen s\"{a}\"{a}ntöj\"{a} (esim. mie, h\"{a}nen koira, me kirjotetaan vs. min\"{a}, h\"{a}nen koiransa, me kirjoitamme).''


\section{Corpus}

\begin{table}
  \centering
 \begin{tabular}{|l|r|r|r|}
    \hline
    \multirow{2}{*}{\textbf{Section}} & \multicolumn{3}{|c|}{\textbf{Tokens}} \\\cline{2-4}
                                      & \texttt{train} & \texttt{test} & \texttt{dev} \\
    \hline
    Colloquial                           & 5,103 & 1,012 & 1,003 \\ 
    Normalised                           & 5,105 & 1,012 & 1,003 \\ 
    Standardised                         & 4,982 & 991 & 1,000 \\
    \hline
 \end{tabular} 
  \caption{Statistics on sentences from the parallel corpus.}
  \label{table:corpsize}
\end{table}

\begin{table*}
  \centering
  \begin{tabular}{|l|l|l|}
  \hline
  \textbf{Colloquial} & \textbf{Normalised} & \textbf{Standardised} \\
  \hline
  tai emm\"{a} tii\"{a} olikse erikseen & tai en\#min\"{a} tied\"{a} oliko\#se erikseen & tai en min\"{a} tied\"{a} oliko erikseen \\
  joku nuorisoalennus                             & jokin nuorisoalennus & nuorisoalennus \\
  \hline
   toistaseks tullu        & toistaiseksi tullut  & toistaiseksi on tullut  \\
   kaks kysymyst\"{a}       & kaksi kysymyst\"{a} & kaksi kysymyst\"{a} \\
  \hline
  ja sit 2009 just ennenku & ja sitten 2009 juuri ennen\#kuin  & ja sitten 2009 juuri ennen kuin  \\ 
menin Japaniin & menin Japaniin & menin Japaniin
 \\
  \hline
  \end{tabular}
  \caption{Example sentences from the parallel corpus. The \texttt{\#} mark represents a missing
    word boundary.}
  \label{table:corpexample}
\end{table*}

Our evaluation corpus was created by manually translating texts in colloquial Finnish
to standard Finnish.\footnote{The corpus is freely available and published under the 
CreativeCommons {\sc cc-by-sa} 3.0 licence.} The texts were extracts from internet 
relay chat (IRC) conversations. We performed the conversion process in two steps, the first
step involved simple orthographic normalisation, for 
example \emph{oon} $\rightarrow$ \emph{olen} `I am'. Syntactic and stylistic conversions 
were not applied at this stage. The second conversion step normalised the text
both orthographically and syntactically/stylistically. Table~\ref{table:corpexample} presents
an excerpt from each of the three parts of the corpus.

The corpus was split into three parts, testing, development and training. The testing
and development portions contain approximately 1,000 words each, with the remaining approximately 5,000 words for training
supervised models.\footnote{The corpus is split into 14 files of 500 words each. Files 01--02 
  were used for development; 03--04 for testing and 05--14 for training.} Table~\ref{table:corpsize} gives statistics on the number of words in each section.

\section{Experiments}

\subsection{Unsupervised normalisation}

For the unsupervised normalisation we applied a set of regular-expression based 
replace rules to the input text to produce all the possible candidate sentences 
in standard Finnish and then used a target-language model to rank the possible 
candidates. The candidate with the highest rank was selected as the normalised sentence.
For the target-language model we used the Finnish side of the English--Finnish EuroParl
parallel corpus \cite{koehn2005}.

We developed two sets of rules for the unsupervised normalisation approach:

\begin{itemize}
  \item 273 rules from \newcite{karlsson2008}'s grammar of Finnish (\S95--97). The 
    rules took around one hour to implement.
  \item 98 rules written by examining the development corpus, these rules also
    took approximately one hour to implement.
\end{itemize}

\begin{table*}
  \centering
  \begin{tabular}{|lrl|l|}
    \hline
    \textbf{Rules:} & & 1. m\"{a} $\rightarrow$ min\"{a} \\
     ~              & & 2. oon $\rightarrow$ olen \\
     ~              & & 3. (?+)nkaa $\rightarrow$ \textbackslash1n kanssa \\
    \hline
  \end{tabular}
~\\
~\\
  \begin{tabular}{|l|r|l|l|}
    \hline
    \textbf{Input:} & \multicolumn{3}{l|}{M\"{a} oon Inarinkaa `I am with Inari'.} \\
    \hline
    \multirow{2}{*}{Step 1} & & M\"{a} oon Inarinkaa & apply rule 1 \\
                            & & Min\"{a} oon Inarinkaa & \\
    \hline
    \multirow{2}{*}{Step 2} & & M\"{a} oon Inarinkaa & apply rule 2 \\
                            &   & Min\"{a} oon Inarinkaa & \\
                            &   & M\"{a} olen Inarinkaa & \\
                            &   & Min\"{a} olen Inarinkaa & \\
    \hline
    \multirow{2}{*}{Step 3} &   & M\"{a} oon Inarinkaa   & apply rule 3\\
                            &   & Min\"{a} oon Inarinkaa  &\\
                            &   & M\"{a} olen Inarinkaa  &\\
                            &   & Min\"{a} olen Inarinkaa  &\\
                            &   & M\"{a} oon Inarin kanssa  &\\
                            &   & Min\"{a} oon Inarin kanssa  &\\
                            &   & M\"{a} olen Inarin kanssa  &\\
                            &   & Min\"{a} olen Inarin kanssa  &\\
    \hline
    \multirow{2}{*}{Step 4} & -4.5811 &  Min\"{a} olen Inarin kanssa & rank candidates \\
                            & -7.8174 &  Min\"{a} oon Inarin kanssa & \\
                            & -8.0941 &  M\"{a} olen Inarin kanssa & \\
                            & -8.8651&  M\"{a} oon Inarin kanssa & \\
                            & -9.2045 &  Min\"{a} olen Inarinkaa & \\
                            & -12.4408 &  Min\"{a} oon Inarinkaa & \\
                            & -12.7176 &  M\"{a} olen Inarinkaa & \\
                            & -13.4885 &  M\"{a} oon Inarinkaa & \\

    \hline
    \textbf{Output:} & \multicolumn{3}{l|}{Min\"{a} olen Inarin kanssa} \\
    \hline
  \end{tabular}
  \caption{Example trace of the unsupervised normalisation method. Rules are applied in order to each of 
     the possible candidate translations in turn. The candidates are then ranked using an $n$-gram language model 
     of standard Finnish and either an $n$-best list
     or the best candidate is output.}
  \label{table:unsup-trace}
\end{table*}

\subsection{Statistical machine translation}


The statistical-machine translation approaches were implemented using the 
Moses toolkit \cite{koehn2007}. The training set up was that used for
the baseline system in the  in the WMT shared tasks
on machine translation.\footnote{\url{http://www.statmt.org/wmt11/baseline.html}}

The target-language model corpus used was the same as in the unsupervised experiments. 

We trained models based on two approaches, the first being phrase-based machine translation
CITEXXX and the second on character-based machine translation \cite{nakov2012,TiedemannEAMT2009}.

For both approaches we trained two systems, the first used the \emph{normalised}
part of the corpus as the target language; the second used the \emph{standardised} 
part of the corpus as the target language.

The idea behind this was that the \emph{normalised} part of the corpus would be 

\subsubsection{Character-based}

\newcite{nakov2012} present a method of statistical machine translation on the character level between related languages that takes advantage of phrase-based machine translation architecture. The method relies on preprocessing the input and output by inserting spaces in between the characters of words, for example the string `m\"{a} meen Helsinkiin' would become `m \"{a} \$ m e e n \$ H e l s i n k i i n' with a unigram model, or `m\"{a} \"{a}\$ \$m me ee en n\$ \$H He el ls si in nk ki ii in' with a bigram model.

After preprocessing, the corpora are processed as with the phrase-based system, with the difference that the language model order is increased from 5 to 10-grams.

\subsection{End-to-end translation}

The idea of the end-to-end translation experiment was to evaluate how well the different 
normalisation strategies worked in combination with another language technology tool. To
evaluate this, we took the test corpus and translated it to English. 

We then translated the corpus with each of the best performing methods of each class, and 
calculated the {\sc bleu} score.

\section{Results}

\begin{table}
  \centering
  \begin{tabular}{|l|r|r|r|}
     \hline
    \textbf{System} & \textbf{PER} & \textbf{WER} & \textbf{BLEU} \\
     \hline
     Baseline & 46.12 & 48.04 & 26.31 \\
     \hline
     rules-1 & 38.27 & 41.19 & 32.65 \\
     rules-2 & 38.17 & 35.25 & 36.41 \\
     rules-c & 36.56 & 39.68 & 34.68 \\
     \hline
     CBMT-cn & 43.09 & 48.34 & 33.55 \\
     CBMT-cs & 46.22 & 52.27 & 29.21 \\
     PBMT-cn & 28.05 & 35.42 & 48.37 \\
     PBMT-cs & 27.95 & 36.13 & 46.76 \\
     \hline
  \end{tabular}
  \caption{Results for the normalisation task}
  \label{table:results-norm}
\end{table}

Tables~\ref{table:results-norm} and \ref{table:results-trad} presents the results of the experiments.

\begin{table}
  \centering
  \begin{tabular}{|l|r|r|r|}
     \hline
    \textbf{System} & \textbf{PER} & \textbf{WER} & \textbf{BLEU} \\
     \hline
     Colloquial & - & - & 11.95 \\
     Normalised & - & - & 21.81 \\
     Standardised & - & - & 17.62 \\
     \hline
     rules-2 & - & - & 12.28 \\
     cbmt-cn & - & - & 12.35 \\
     pbmt-cn & - & - & 17.90 \\
     \hline
  \end{tabular}
  \caption{Results for the translation task}
  \label{table:results-trad}
\end{table}

\section{Future work}

% expand test corpus -- up to 10,000 words
% add other colloquial genres
% apply to other NLP tasks
% weight rules 
% 

\section{Conclusions}

We have presented a parallel corpus of colloquial Finnish and standard Finnish --
to our knowledge the first of its kind -- and an evaluation of methods for 
converting colloquial Finnish to standard Finnish. 

We have shown that converting from colloquial Finnish to standard Finnish substantially
helps with the machine translation task.

%The main contribution of this paper is a parallel corpus of colloquial Finnish
%and standard Finnish

\section*{Acknowledgments}

Joonas Kylm\"{a}l\"{a}

% If you use BibTeX with a bib file named eacl2014.bib, 
% you should add the following two lines:
\bibliographystyle{acl}
\bibliography{finnormal2015.bib}

\end{document}
