%
% File nodalida2015.tex
%
% Contact beata.megyesi@lingfil.uu.se
%
% Based on the instruction file for EACL 2014
% which in turn was based on the instruction files for previous 
% ACL and EACL conferences.

\documentclass[11pt]{article}
\usepackage{nodalida2015}
%\usepackage{times}
\usepackage{mathptmx}
%\usepackage{txfonts}
\usepackage{url}
\usepackage{latexsym}
\special{papersize=210mm,297mm} % to avoid having to use "-t a4" with dvips 
%\setlength\titlebox{6.5cm}  % You can expand the title box if you really have to

\title{Automatic conversion of colloquial Finnish to standard Finnish}

\author{}
%
%Francis M. Tyers\\
%  UiT Norgga \'{a}rktala\v{s} universitehta, \\
%  N-9015 Norway \\
%  {\tt francis.tyers@uit.no} \\\And
%  Inari Listenmaa\\
%  Chalmers tekniska h\"{o}gskola, \\
%  Sweden \\
%  {\tt inari@cse.chalmers.se} \\}
%
\date{}

\begin{document}
\maketitle
\begin{abstract}
  This paper presents an unsupervised method for converting between colloquial Finnish
  and standard Finnish. The method relies upon a small number of orthographical rules
  combined with a large language model of standard Finnish for ranking the possible 
  conversions. Aside from this contribution, the paper also presents an evaluation
  corpus consisting of aligned sentences in colloquial Finnish, orthographically-standardised
  colloquial Finnish and standard Finnish. The methods we present outperforms the baseline
  of simply treating colloquial Finnish as standard Finnish and offers promise for the adaptation
  of language-technology tools created for standard Finnish to colloquial Finnish.
\end{abstract}

\section{Introduction}

Most language technology tools are designed or trained based on standard language 
forms, where they exist. The application of these tools to non-standard
language can cause a substantial decrease in quality. For example in machine translation,
or part-of-speech tagging. CITEXXX

For language-technology researchers working on non-standard forms of language, there are 
two clear options: either create new tools to process non-standard text,
  or create tools to preprocess non-standard text, standardising it to be subsequently processed
 by existing tools.

This paper presents a method for converting colloquial Finnish to standard Finnish and 
a parallel corpus for evaluation.

\section{Related work}

There are a number of areas of research related to the task of text normalisation. Text
proofing tools, such as spelling and grammar checkers CITEXXX can be used to encourage adherence
to particular orthographic or grammatical norms. Accent and diacritic restoration CITEXXX is similar
in that it aims to bring text closer to standard orthography in order to facilitate treatment by 
automatic tools. Another related area is machine translation between different written norms of 
the same language. For example between Norwegian Bokmål and Norwegian Nynorsk CITEXXX.

\newcite{scannell2014} presents a method for normalising pre-standardised text in Irish to 
the modern standard. The method relies on 

\subsection{Colloquial Finnish}

 \emph{suomen puhekieli}

\section{Corpus}

\begin{table}
  \centering
 \begin{tabular}{|l|r|}
    \hline
    \textbf{Section} & \textbf{Tokens} \\
    \hline
    \texttt{train} & 2,538 \\
    \texttt{test} & 1,012 \\
    \texttt{dev} & 1,003 \\
    \hline
 \end{tabular} 
  \caption{Statistics on sentences from the parallel corpus.}
  \label{table:corpsize}
\end{table}

\begin{table*}
  \centering
  \begin{tabular}{|l|l|l|}
  \hline
  \textbf{Colloquial} & \textbf{Normalised} & \textbf{Standardised} \\
  \hline
  mut siis, oli siell\"{a} jotain vaikka     & mutta siis, oli siell\"{a} jotain vaikka     & mutta oli siell\"{a} jotain     \\
  puheteknologiamatskuja.  & puheteknologiamateriaaleja. & puheteknologiamateriaaleja.  \\
  \hline
  oon tosi hy\"{o}dyllinen assari     & ole todella hy\"{o}dyllinen assistentti & ole todella hy\"{o}dyllinen assistentti \\
  t\"{a}n vikan minuutin.           & t\"{a}m\"{a}n viimeisen minuutin.            & t\"{a}m\"{a}n viimeisen minuutin. \\
  \hline
  \ldots & \ldots & \ldots \\
  \hline
  \end{tabular}
  \caption{Example sentences from the parallel corpus.}
  \label{table:corpexample}
\end{table*}

Our evaluation corpus was created by manually translating texts in colloquial Finnish
to standard Finnish.\footnote{The corpus is freely available and published under the 
CreativeCommons {\sc cc-by-sa} 3.0 licence.} The texts were extracts from internet 
relay chat (IRC) conversations. We performed the conversion process in two steps, the first
step involved simple orthographic normalisation, for 
example \emph{oon} $\rightarrow$ \emph{olen} `I am'. Syntactic and stylistic conversions 
were not applied at this stage. The second conversion step normalised the text
both orthographically and syntactically/stylistically. Table~\ref{table:corpexample} presents
an excerpt from each of the three parts of the corpus.

The corpus was split into three parts, testing, development and training. The testing
and development portions contain 1,000 words each, with the remaining X,000 for training
supervised models.\footnote{The corpus is split into XX files of 500 words each. Files 01--02 
  were used for development; 03--04 for testing and 05--XX for training.}

\section{Methodology}

\subsection{Unsupervised normalisation}

For the unsupervised normalisation we applied a set of regular-expression based 
replace rules to the input text to produce all the possible candidate sentences 
in standard Finnish and then used a target-language model to rank the possible 
candidates. The candidate with the highest rank was selected as the normalised sentence.
For the target-language model we used the Finnish side of the English--Finnish EuroParl
parallel corpus \cite{koehn2005}.

We developed two sets of rules for the unsupervised normalisation approach:

\begin{itemize}
  \item 273 rules from \newcite{karlsson2008}'s grammar of Finnish (\S95--97). The 
    rules took around one hour to implement.
  \item 98 rules written by examining the development corpus, these rules also
    took approximately one hour to implement.
\end{itemize}

%Rules: mä -> minä, oon -> olen, ?*nkaa -> ?*n kanssa
%
%Input: Mä oon Tomminkaa
%Step 1: Mä oon Tomminkaa
%        Minä oon Tomminkaa
%Step 2: Mä oon Tomminkaa
%        Minä oon Tomminkaa
%        Mä olen Tomminkaa
%        Minä olen Tomminkaa
%Step 3: Mä oon Tomminkaa
%        Minä oon Tomminkaa
%        Mä olen Tomminkaa
%        Minä olen Tomminkaa
%        Mä oon Tommin kanssa
%        Minä oon Tommin kanssa
%        Mä olen Tommin kanssa
%        Minä olen Tommin kanssa
%Rank:
%	-13.4885 ← minä olen Tommin kanssa
%	-12.7176 ← minä oon Tommin kanssa
%	-12.4408 ← mä olen Tommin kanssa
%	-9.2045 ← mä oon Tommin kanssa
%	-8.8651 ← minä olen Tomminkaa
%	-8.0941 ← minä oon Tomminkaa
%	-7.8174 ← mä olen Tomminkaa
%	-4.5811 ← mä oon Tomminkaa
%Output: Minä olen Tommin kanssa	

\subsection{Statistical machine translation}


The statistical-machine translation approaches were implemented using the 
Moses toolkit \cite{koehn2007}. The training set up was that used for
the baseline system in the  in the WMT shared tasks
on machine translation.\footnote{\url{http://www.statmt.org/wmt11/baseline.html}}

The target-language model used was the same as in the unsupervised experiments. 

For both approaches we trained two systems, the first used the \emph{normalised}
part of the corpus as the target language; the second used the \emph{standardised} 
part of the corpus as the target language.

\subsubsection{Phrase-based}

\subsubsection{Character-based}

\cite{nakov2012}

\section{Results}

\begin{table}
  \centering
  \begin{tabular}{|l|r|r|r|}
     \hline
    \textbf{System} & \textbf{PER} & \textbf{WER} & \textbf{BLEU} \\
     \hline
     Baseline & 46.12 & 48.04 & 26.31 \\
     \hline
     rules-1 & 38.27 & 41.19 & 32.65 \\
     rules-2 & 38.17 & 35.25 & 36.41 \\
     rules-c & 36.56 & 39.68 & 34.68 \\
     \hline
     CBMT$^1$-cn & - & - & - \\
     CBMT$^1$-cs & 46.22 & 52.27 & 29.21 \\
     CBMT$^2$-cn & - & - & - \\
     CBMT$^2$-cs & - & - & - \\
     PBMT-cn & 28.05 & 35.42 & 48.37 \\
     PBMT-cs & 27.95 & 36.13 & 46.76 \\
     \hline
  \end{tabular}
  \label{table:results}
  \caption{Results...}
\end{table}

\section{Future work}

% weight rules 
% use system in pipeline (e.g. translation)
% 

\section{Conclusions}

We have presented a parallel corpus of colloquial Finnish and standard Finnish --
to our knowledge the first of its kind -- and a method for converting colloquial
Finnish to standard Finnish.

%The main contribution of this paper is a parallel corpus of colloquial Finnish
%and standard Finnish

\section*{Acknowledgments}

Joonas Kyl\"{a}m\"{a}

% If you use BibTeX with a bib file named eacl2014.bib, 
% you should add the following two lines:
\bibliographystyle{acl}
\bibliography{finnormal2015.bib}

\end{document}
