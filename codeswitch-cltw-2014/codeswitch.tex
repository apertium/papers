%
% File coling2014.tex
%
% Contact: jwagner@computing.dcu.ie
%%
%% Based on the style files for ACL-2014, which were, in turn,
%% Based on the style files for ACL-2013, which were, in turn,
%% Based on the style files for ACL-2012, which were, in turn,
%% based on the style files for ACL-2011, which were, in turn, 
%% based on the style files for ACL-2010, which were, in turn, 
%% based on the style files for ACL-IJCNLP-2009, which were, in turn,
%% based on the style files for EACL-2009 and IJCNLP-2008...

%% Based on the style files for EACL 2006 by 
%%e.agirre@ehu.es or Sergi.Balari@uab.es
%% and that of ACL 08 by Joakim Nivre and Noah Smith

\documentclass[11pt]{article}
\usepackage{coling2014}
\usepackage{times}
\usepackage{url}
\usepackage{latexsym}

%\setlength\titlebox{5cm}

% You can expand the titlebox if you need extra space
% to show all the authors. Please do not make the titlebox
% smaller than 5cm (the original size); we will check this
% in the camera-ready version and ask you to change it back.


\title{Subsegmental language detection in Celtic language text}

\author{Akshay \\
  Affiliation / Address line 1 \\
  Affiliation / Address line 2 \\
  Affiliation / Address line 3 \\
  {\tt email@domain} \\\And
  Francis M. Tyers \\
  Giellatekno / CLEAR \\
  UiT Norgga \'arktala\v{s} universitehta  \\
  9017 Romsa (Norway) \\
  {\tt francis.tyers@uit.no} \\}

\date{}

\begin{document}
\maketitle
\begin{abstract}
  This paper describes an experiment to detect the language of subsegments
  of text in three Celtic languages: Breton, Irish and Welsh. 
\end{abstract}

\section{Introduction}
\label{intro}

\cite{lyu2006}

%
% The following footnote without marker is needed for the camera-ready
% version of the paper.
% Comment out the instructions (first text) and uncomment the 8 lines
% under "final paper" for your variant of English.
% 
\blfootnote{
    %
    % for review submission
    %
    \hspace{-0.65cm}  % space normally used by the marker
    Place licence statement here for the camera-ready version, see
    Section~\ref{licence} of the instructions for preparing a
    manuscript.
    %
    % % final paper: en-uk version (to license, a licence)
    %
    % \hspace{-0.65cm}  % space normally used by the marker
    % This work is licensed under a Creative Commons 
    % Attribution 4.0 International Licence.
    % Page numbers and proceedings footer are added by
    % the organisers.
    % Licence details:
    % \url{http://creativecommons.org/licenses/by/4.0/}
    % 
    % % final paper: en-us version (to licence, a license)
    %
    % \hspace{-0.65cm}  % space normally used by the marker
    % This work is licenced under a Creative Commons 
    % Attribution 4.0 International License.
    % Page numbers and proceedings footer are added by
    % the organizers.
    % License details:
    % \url{http://creativecommons.org/licenses/by/4.0/}
}

The following instructions are directed to authors of papers submitted
to COLING-2014 or accepted for publication in its proceedings. All
authors are required to adhere to these specifications. Authors are
required to provide a Portable Document Format (PDF) version of their
papers. \textbf{The proceedings are designed for printing on A4
  paper.}

Authors from countries in which access to word-processing systems is
limited should contact the publication chairs,
Joachim Wagner, Liadh Kelly and Lorraine Goeuriot
(\texttt{firstname.lastname @computing.dcu.ie} respectively),
as soon as possible.

We will make additional instructions available at
\url{http://www.coling-2014.org/call-for-papers.php}. Please check
this website regularly.



\section*{Acknowledgements}

We thank Kevin Scannell... 

% include your own bib file like this:
\bibliographystyle{acl}
\bibliography{codeswitch}


\end{document}
