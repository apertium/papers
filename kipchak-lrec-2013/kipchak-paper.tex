\documentclass{article}

\usepackage{fontspec}
\usepackage{xunicode}
\usepackage{xltxtra}
\usepackage{natbib}
\usepackage{multirow}
\usepackage{fullpage}
\usepackage{multirow}

\usepackage{booktabs}

\setromanfont{DejaVu Serif}


\bibdata{kipchak-paper}


\title{Finite-state morphological transducers for three Kypchak languages}

\author{Jonathan North Washington \\
Departments of Linguistics and Central Eurasian Studies\\
Indiana University\\
Bloomington, IN 47405 (USA)\\
\texttt{jonwashi@indiana.edu} \and
Ilnar Salimzyanov  \\
Institut für Maschinelle Sprachverarbeitung \\
Universität Stuttgart\\
Stuttgart (Germany) \\
\texttt{ilnar@ilnar} \and 
Francis M. Tyers\\
Departament de Llenguatges i Sistemes Informàtics \\  
Universitat d'Alacant\\
E-03071 Alacant (Spain)\\
\texttt{ftyers@dlsi.ua.es} 
}

\begin{document}

\maketitle

\begin{abstract}
Hargle, bargle.
\end{abstract}

\section{Introduction}

The Northwestern branch of Turkic is often referred to as the Kypchak branch, and can be divided into three subbranches.  Kumyk is a member of the Western Kypchak group, Tatar is a member of the Northern Kypchak group, and Kazakh is a member of the south Kypchak group \citep[82-83]{histofturkic}.  The geographic distribution of the languages is shown in map \ref{}.

\cite{washington2012}
\cite{salimzyanov2013}
\cite{bekmanova2013}

\section{Languages}

\subsection{Kazakh}

\subsection{Tatar}

\subsection{Kumyk}

\cite{bammatov1960}

\section{Methodology}

\section{Evaluation}

\begin{table}
\begin{center}
\begin{tabular}{lrrr}
		\hline
\multirow{2}{*}{\textbf{Part of speech}} & \multicolumn{3}{|c|}{\textbf{Number of stems}} \\
                        & Kazakh & Tatar & Kumyk \\
		\hline
		Noun & - & - & - \\
		Verb & - & - & - \\
		Adjective & - & - & - \\
		Proper noun & - & - & - \\
		Adverb & - & - & - \\
		Numeral & - & - & - \\
		Conjunction & - & - & - \\
		Postposition & - & - & - \\
		Pronoun & - & - & - \\
		Determiner & - & - & - \\
		\hline
		Total: & - & - & - \\
		\hline
\end{tabular}
\end{center}

\end{table}

\begin{table}
\begin{center}
\begin{tabular}{lr}
\textbf{Language} & - \\
\hline
Kazakh & - \\
Tatar & - \\
Kumyk & - \\
\hline
\end{tabular}
 \caption{Naïve coverage}
 \label{table:coverage}
\end{center}
\end{table}

\begin{table}
\begin{center}
\begin{tabular}{lrr}
\textbf{Language} & \textbf{Precision} & \textbf{Recall} \\
\hline
Kazakh & - &  - \\
Tatar & - & - \\
Kumyk & - & - \\
\hline
\end{tabular}
 \caption{Precision and recall}
 \label{table:coverage}
\end{center}
\end{table}

\begin{table}
\begin{center}
\begin{tabular}{lrr}
\textbf{Corpus} & \textbf{Words} & \textbf{Coverage} \\
\hline
wikipedia & 850K &  - \\
Äwezov & 155K & - \\
RFERL 2010 & 3.2M & - \\
bible & 577K & - \\
quran & 107K & - \\
\hline
\end{tabular}
 \caption{Corpora used for coverage of Kazakh}
 \label{table:corpora-kaz}
\end{center}
\end{table}

\begin{table}
\begin{center}
\begin{tabular}{llrr}
\toprule
\textbf{Language} & \textbf{Corpus} & \textbf{Words} & \textbf{Coverage} \\
\midrule
\multirow{6}{*}{Kazakh} & wikipedia 2011& 850K &  - \\
	& Äwezov & 155K & - \\
	& RFERL 2010 & 3.2M & - \\
	& bible & 577K & - \\
	& quran & 107K & - \\\cline{2-4}
	& average & - & 90.5\% \\
\midrule
\multirow{5}{*}{Tatar} & wikipedia 2013 & 128K &  - \\
	& news 2005-2011 & 4.6M & - \\
	& new testament & 137K & - \\
	& quran & 165K & - \\
	& Aytmatov & 5K & - \\\cline{2-4}
	& average & - & 89.0\% \\
\midrule
\multirow{4}{*}{Kumyk} & yoldash & 287K &  - \\
	& new testament & 154K & - \\
	& book of Genesis & 28K & - \\\cline{2-4}
	& average & - & 88.0\% \\
\bottomrule
\end{tabular}
 \caption{Corpora used for coverage tests}
 \label{table:corpora}
\end{center}
\end{table}



\section{Future work}

%Code switching
%More languages 

\section{Conclusions}


\bibliographystyle{apalike}

\bibliography{kipchak-paper}

\end{document}
