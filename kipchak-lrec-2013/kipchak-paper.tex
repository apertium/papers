\documentclass[a4paper,11pt,twocolumn]{article}

\usepackage{lrec2012}

\usepackage{polyglossia}
\setdefaultlanguage[variant=australian]{english}

\usepackage{fontspec}
\usepackage{xunicode}
\usepackage{xltxtra}
\usepackage{natbib}
\usepackage{multirow}
%\usepackage{fullpage}
\usepackage{multirow}

\usepackage{booktabs}

\setromanfont{Times New Roman}


\bibdata{kipchak-paper}


\title{Finite-state morphological transducers for three Kypchak languages}

\author{Jonathan North Washington \\
Departments of Linguistics and Central Eurasian Studies\\
Indiana University\\
Bloomington, IN 47405 (USA)\\
\texttt{jonwashi@indiana.edu} \and
Ilnar Salimzyanov  \\
Institut für Maschinelle Sprachverarbeitung \\
Universität Stuttgart\\
Stuttgart (Germany) \\
\texttt{ilnar@ilnar} \and 
Francis M. Tyers\\
Departament de Llenguatges i Sistemes Informàtics \\  
Universitat d'Alacant\\
E-03071 Alacant (Spain)\\
\texttt{ftyers@dlsi.ua.es} 
}


\name{Jonathan North Washington$^\dagger$, Ilnar Salimzyanov$^\ddagger$, Francis M. Tyers$^\star$}

\address{$^\dagger$Departments of Linguistics and Central Eurasian Studies\\
Indiana University\\
Bloomington, IN 47405 (USA)\\
\texttt{jonwashi@indiana.edu} \and
$^\ddagger$Institut für Maschinelle Sprachverarbeitung \\
Universität Stuttgart\\
Stuttgart (Germany) \\
\texttt{ilnar@ilnar} \and 
$^\star$Departament de Llenguatges i Sistemes Informàtics \\  
Universitat d'Alacant\\
E-03071 Alacant (Spain)\\
\texttt{ftyers@dlsi.ua.es} 
}

\abstract{Hargle, bargle.\\
\Keywords{Kazakh, Tatar, Kumyk, morphology, transducer}}

\begin{document}

\maketitleabstract{}

\section{Introduction}

The Northwestern branch of Turkic is often referred to as the Kypchak branch, and can be divided into three subbranches.  Kumyk is a member of the Western Kypchak group, Tatar is a member of the Northern Kypchak group, and Kazakh is a member of the south Kypchak group \citep[82-83]{histofturkic}.  The geographic distribution of the languages is shown in map \ref{}.

\cite{washington2012}
\cite{salimzyanov2013}
\cite{bekmanova2013}

\section{Languages}

\subsection{Kazakh}

\subsection{Tatar}

\subsection{Kumyk}

\cite{bammatov1960}
\cite{olmesov2000}

\section{Methodology}

% Noun morphotactics -- is basically equivalent in all three, except for archiphonemes
% Adjective categorisation
% Adverb categorisation
% "Irregular" harmony, caused by orthography
% Non-finite verb form categorisation "converbs"

% Handling numerals and acronyms.

\subsection{Development effort}
% how long it took
% make a graph for the kumyk one

\subsection{Statistics}

\begin{table}
\begin{center}
\begin{tabular}{|l|rrr|}
		\hline
\multirow{2}{*}{\textbf{Part of speech}} & \multicolumn{3}{|c|}{\textbf{Number of stems}} \\ \cline{2-4}
                        & Kazakh & Tatar & Kumyk \\
		\hline
		Noun & - & - & - \\
		Verb & - & - & - \\
		Adjective & - & - & - \\
		Proper noun & - & - & - \\
		Adverb & - & - & - \\
		Numeral & - & - & - \\
		Conjunction & - & - & - \\
		Postposition & - & - & - \\
		Pronoun & - & - & - \\
		Determiner & - & - & - \\
		\hline
		Total: & - & - & - \\
		\hline
\end{tabular}
 \caption{Number of stems in each of the categories}
 \label{table:coverage}
\end{center}

\end{table}

\section{Evaluation}

We have evaluated the morphological analysers in two ways. The first was by calculating the naïve coverage\footnote{Naïve coverage refers to the percentage of surface forms in a given corpora that receive at least one analysis.  Forms counted by this measure may have other analyses which are not delivered by the transducer.} and mean ambiguity 
on freely available corpora. The second was by performing an evaluation of precision and recall on some 
smaller, hand-validated test sets.

\subsection{Corpora}

% For kazakh+tatar tested the coverage of the analysers over 3 separate domains: encyclopaedic text, news and religion, 
% as there was no wikipedia for kumyk we just tested news and religion. The corpora were obtained from:
% Wikipedia:
% RFE/RL
% Yoldaš: 
% New testaments:

% kaz: kkwiki-20131006-pages-articles.xml.bz2
% tat:

\begin{table}
\begin{center}
\begin{tabular}{llrr}
\toprule
\textbf{Language} & \textbf{Corpus} & \textbf{Words} & \textbf{Coverage} \\
\midrule
\multirow{4}{*}{Kazakh} & Wikipedia 2013 &  -  &  - \\
	& RFE/RL 2010 & 3.2M & - \\
	& Bible & 577K & - \\\cline{2-4}
	& Average & - & 90.5\% \\
\midrule
\multirow{4}{*}{Tatar} & Wikipedia 2013 & 128K &  - \\
	& RFE/RL 2005--2011 & 4.6M & - \\
	& New Testament & 137K & - \\\cline{2-4}
	& Average & - & 89.0\% \\
\midrule
\multirow{3}{*}{Kumyk} & Yoldaš & 287K &  - \\
	& New Testament & 154K & - \\\cline{2-4}
	& Average & - & 90.1\% \\
\bottomrule
\end{tabular}
 \caption{Corpora used for naïve coverage tests}
 \label{table:corpora}
\end{center}
\end{table}



\begin{table}
\begin{center}
\begin{tabular}{lrr}
\textbf{Language} & \textbf{Precision} & \textbf{Recall} \\
\hline
Kazakh & - &  - \\
Tatar & - & - \\
Kumyk & - & - \\
\hline
\end{tabular}
 \caption{Precision and recall}
 \label{table:coverage}
\end{center}
\end{table}



\section{Future work}

%Code switching - +example
%More languages: Nogai, Bashkir, Karakalpak, Karachay-Balkar

\section{Conclusions}

\section*{Acknowledgements}

We would like to thank the Google Code-in (2011) for supporting the development 
of the Kazakh transducer, and in particular the effort by Nathan Maxson. We 
would also like to thank the Google Summer of Code (2012) for supporting the 
development of both the Kazakh and the Tatar transducers. 

\bibliographystyle{lrec2012}
\bibliography{kipchak-paper}

\end{document}
